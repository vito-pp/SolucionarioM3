% archivo con las configuraciones del documento y los paquetes usados

\usepackage{tocloft} % para cambiar formato de \tableofcontents
\usepackage{titlesec} % para cambiar formato de los titulos
\usepackage{amsmath, amssymb, amsthm, mathtools, esint} % Paquetes de simbolos matematicos
\usepackage{xcolor}
\usepackage[skip=20pt, indent=0pt]{parskip} 
\usepackage[margin=3cm]{geometry}
\usepackage[spanish, shorthands=off]{babel} % Disable shorthands to prevent conflicts
\usepackage{chngcntr} % para cambiar los contadores
\usepackage{hyperref} % para hyperlinks asociados al documento
\usepackage{pgfplots} % para graficos 
\pgfplotsset{compat=1.18}
\usepackage{tikz} % ambiente para graficar 
\usepackage{float} 
\usepgfplotslibrary{fillbetween}
\usetikzlibrary{patterns, shapes}
\usetikzlibrary{arrows.meta, decorations.pathmorphing, positioning}

% para graficar flechas en curvas cerradas
\xdef\Rad{2}
\newcommand{\ARW}[2][]{%
    \foreach \ang in {#2}{%
        \draw[#1] (\ang:\Rad)--(\ang+1:\Rad) ;
    }
}
% simbolo final de teorema, definicion, corolario, etc.
\newcommand{\final}{\hfill$\diamond$}
\newcommand{\finalmath}{\tag*{$\diamond$}}
% numeros reales
\renewcommand{\Re}{\mathbb {R}}
% derivadas parciales
\newcommand{\partialx}{\frac{\partial}{\partial x}}
\newcommand{\partialy}{\frac{\partial}{\partial y}}
\newcommand{\partialz}{\frac{\partial}{\partial z}}
% numeros romanos
\newcommand*{\rom}[1]{\expandafter\@slowromancap\romannumeral #1@}
% interior de un conjunto
\newcommand{\interior}[1]{%
    {\kern0pt#1}^{\mathrm{o}}%
}

%%%%%%%%%%%%%%%%%%%%%%%%%%%%
\renewcommand{\contentsname}{Indice}
\renewcommand{\listfigurename}{Lista de Figuras}
\renewcommand{\listtablename}{Lista de Tablas}
\renewcommand{\bibname}{Bibliograf\'{\i}a}
\renewcommand{\indexname}{Indice}
\renewcommand{\figurename}{Figura}
\renewcommand{\tablename}{Tabla}
\renewcommand{\partname}{Parte}
\renewcommand{\chaptername}{Cap\'{\i}tulo}
\renewcommand{\appendixname}{Ap\'endice}
\renewcommand{\abstractname}{Resumen}
%%%%%%%%%%%%%%%%%%%%%%%%%%%%

%%%%%%%%%%%%%%%%%%%%%%%%%%%%%%%%
%  Teoremas y proposiciones
%%%%%%%%%%%%%%%%%%%%%%%%%%%%%%%%%
\theoremstyle{definition} % Para que el texto no aparezca en italicas
\newtheorem{question}{Ejercicio}
\newtheorem{solution}{Solución}
\newtheorem{theorem}{Teorema}[section]
\newtheorem{corollary}{Corolario}[section]
\newtheorem{definition}{Definic\'on}[section]
\newtheorem{propertie}{Propiedad}[section]
\newtheorem{example}{Ejemplo}[section]

% Provide commands for \autoref
\providecommand*{\questionautorefname}{Ejercicio}
\providecommand*{\solutionautorefname}{Solución}
\providecommand*{\theoremautorefname}{Teorema}
\providecommand*{\corollaryautorefname}{Corolario}
\providecommand*{\definitionautorefname}{Definición}
\providecommand*{\propertieautorefname}{Propiedad}
\providecommand*{\exampleautorefname}{Ejemplo}

% Para que se resetee el contador en cada seccion
\counterwithin*{solution}{subsubsection}
\counterwithin*{question}{subsubsection}

\spanishdecimal{.}

% ajuste formato capitulos
\titleformat{\chapter}[display]
  {\normalfont\fontsize{14}{16}\selectfont\bfseries}{\vspace{-4cm}}{0pt}{}[\vspace{-1cm}]

% ajuste formato secciones
\titleformat{\section}
  {\normalfont\fontsize{12}{14}\selectfont\bfseries}{\thesection}{1em}{}

% ajuste formato indice
\renewcommand{\cfttoctitlefont}{\normalfont\Large\bfseries}
\renewcommand{\cftbeforetoctitleskip}{-.5cm}