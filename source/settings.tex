% archivo con las configuraciones del documento y los paquetes usados

\usepackage{tocloft} % para cambiar formato de \tableofcontents
\usepackage{titlesec} % para cambiar formato de los titulos
\usepackage{amsmath, amsthm, mathtools, esint} % Paquetes de simbolos matematicos
\usepackage{amsfonts}
\usepackage{amssymb}
\usepackage{xcolor}
\usepackage[skip=20pt, indent=0pt]{parskip} 
\usepackage[margin=3cm]{geometry}
\usepackage[spanish, shorthands=off]{babel} % Disable shorthands to prevent conflicts
\usepackage{chngcntr} % para cambiar los contadores
\usepackage{hyperref} % para hyperlinks asociados al documento
\usepackage{pgfplots} % para graficos 
\pgfplotsset{compat=1.18}
\usepackage{tikz} % ambiente para graficar 
\usepackage{float} 
\usepgfplotslibrary{fillbetween}
\usetikzlibrary{patterns, shapes}
\usetikzlibrary{arrows.meta, decorations.pathmorphing, positioning}

% para graficar flechas en curvas cerradas
\xdef\Rad{2}
\newcommand{\ARW}[2][]{%
    \foreach \ang in {#2}{%
        \draw[#1] (\ang:\Rad)--(\ang+1:\Rad) ;
    }
}
% simbolo final de teorema, definicion, corolario, etc.
\newcommand{\final}{\hfill$\diamond$}
\newcommand{\finalmath}{\tag*{$\diamond$}}
% numeros reales
\renewcommand{\Re}{\mathbb {R}}
% derivadas parciales
\newcommand{\partialx}{\frac{\partial}{\partial x}}
\newcommand{\partialy}{\frac{\partial}{\partial y}}
\newcommand{\partialz}{\frac{\partial}{\partial z}}
% numeros romanos
\newcommand*{\rom}[1]{\expandafter\@slowromancap\romannumeral #1@}
% interior de un conjunto
\newcommand{\interior}[1]{%
    {\kern0pt#1}^{\mathrm{o}}%
}

%%%%%%%%%%%%%%%%%%%%%%%%%%%%
\renewcommand{\contentsname}{Indice}
\renewcommand{\listfigurename}{Lista de Figuras}
\renewcommand{\listtablename}{Lista de Tablas}
\renewcommand{\bibname}{Bibliograf\'{\i}a}
\renewcommand{\indexname}{Indice}
\renewcommand{\figurename}{Figura}
\renewcommand{\tablename}{Tabla}
\renewcommand{\partname}{Parte}
\renewcommand{\chaptername}{Cap\'{\i}tulo}
\renewcommand{\appendixname}{Ap\'endice}
\renewcommand{\abstractname}{Resumen}
%%%%%%%%%%%%%%%%%%%%%%%%%%%%

%%%%%%%%%%%%%%%%%%%%%%%%%%%%%%%%
%  Teoremas y proposiciones
%%%%%%%%%%%%%%%%%%%%%%%%%%%%%%%%%
\theoremstyle{definition} % Para que el texto no aparezca en italicas
\newtheorem{question}{Ejercicio}
\newtheorem{solution}{Solución}
\newtheorem{theorem}{Teorema}[section]
\newtheorem{corollary}{Corolario}[section]
\newtheorem{definition}{Definici\'on}[section]
\newtheorem{propertie}{Propiedad}[section]
\newtheorem{example}{Ejemplo}[section]
\newtheorem{obs}{Observaci\'on}[section]
% \newthorem{demo}{Demostraci\'on}[section]

% Provide commands for \autoref
\providecommand*{\questionautorefname}{Ejercicio}
\providecommand*{\solutionautorefname}{Solución}
\providecommand*{\theoremautorefname}{Teorema}
\providecommand*{\corollaryautorefname}{Corolario}
\providecommand*{\definitionautorefname}{Definición}
\providecommand*{\propertieautorefname}{Propiedad}
\providecommand*{\exampleautorefname}{Ejemplo}

% Para que se resetee el contador en cada seccion
\counterwithin*{solution}{subsubsection}
\counterwithin*{question}{subsubsection}

\spanishdecimal{.}

% ajuste formato capitulos
\titleformat{\chapter}[display]
  {\normalfont\fontsize{14}{16}\selectfont\bfseries}{\vspace{-4cm}}{0pt}{}[\vspace{-1cm}]

% ajuste formato secciones
\titleformat{\section}
  {\normalfont\fontsize{12}{14}\selectfont\bfseries}{\thesection}{1em}{}

% ajuste formato indice
\renewcommand{\cfttoctitlefont}{\normalfont\Large\bfseries}
\renewcommand{\cftbeforetoctitleskip}{-.5cm}

%========================--paquetes de Pablo Bucello--=========================%

% \ProvidesPackage{pablo}

% \usepackage{amsmath}
% \usepackage{amsthm}
% \usepackage{amssymb}
% \usepackage{amsfonts}
% \usepackage{amsopn}
% \usepackage{mathtools}
\usepackage{physics}
% \usepackage{tikz}
% \usetikzlibrary{patterns}
% % \usepackage{enumitem}
% \usepackage{enumerate}
% \usepackage[spanish,es-noshorthands]{babel}
% % \usepackage[english]{babel}
% \usepackage[latin1]{inputenc}
% \usepackage{url}



\newcommand{\field}[1]{\mathbb{#1}}

\newcommand{\N}{\field{N}}
\newcommand{\Z}{\field{Z}}
\newcommand{\Q}{\field{Q}}
\newcommand{\I}{\field{I}}
\newcommand{\R}{\field{R}}
\newcommand{\C}{\field{C}}

\newcommand{\Rn}[1]{\R^#1}
\newcommand{\Cn}[1]{\C^#1}
\newcommand{\Rnp}[2]{\R^{#1 \times #2}}
\newcommand{\Cnp}[2]{\C^{#1 \times #2}}

% \newcommand{\ip}[2]{< #1 , #2 >}
\newcommand{\dist}[2]{dist \left( #1 , #2 \right)}

% \newcommand{\abs}[1]{\left\lvert#1\right\rvert}
% \newcommand{\norm}[1]{\left\lVert#1\right\rVert}

\newcommand{\conj}[1]{\overline{#1}}

\newcommand{\then}{\Rightarrow}

\newcommand{\dif}[2]{\dfrac{\partial #1}{\partial #2}}
\newcommand{\diff}[2]{\frac{\partial^2 #1}{\partial #2^2}}
\newcommand{\tdif}[2]{\frac{d #1}{d #2}}
\newcommand{\tdiff}[2]{\frac{d^2 #1}{d #2^2}}
% \newcommand{\grad}{\nabla}

\DeclareMathOperator{\gen}{gen}
\DeclareMathOperator{\col}{Col}
\DeclareMathOperator{\nul}{Nul}
\DeclareMathOperator{\rg}{rg}
\DeclareMathOperator{\proy}{proy}
% \DeclareMathOperator{\tr}{tr}
\DeclareMathOperator{\diag}{diag}
\DeclareMathOperator{\im}{Im}
% \DeclareMathOperator{\interior}{int}
\DeclareMathOperator{\frontera}{front}
\DeclareMathOperator{\front}{\partial}
\DeclareMathOperator{\dis}{d}

% \theoremstyle{plain}
%   \newtheorem{theorem}{Teorema}
%   \newtheorem{corolario}{Corolario}
%   \newtheorem{lema}{Lema}
%   \newtheorem{propiedad}{Propiedad}

% \theoremstyle{definition}  
%   \newtheorem{definition}{Definici�n}

% \theoremstyle{remark}  
%   \newtheorem{exmp}{Ejemplo}
%   \newtheorem{obs}{Observaci�n}

% \numberwithin{equation}{section}
% \numberwithin{theorem}{section}
% \numberwithin{corolario}{section}
% \numberwithin{exmp}{section}
% \numberwithin{lema}{section}
% \numberwithin{definition}{section}
% \numberwithin{propiedad}{section}
% % \numberwithin{proposicion}{section}
% \numberwithin{obs}{section}

% \renewcommand{\labelenumi}{\textbf{(\Roman{enumi})}}
\renewcommand{\thefootnote}{(\arabic{footnote})}
