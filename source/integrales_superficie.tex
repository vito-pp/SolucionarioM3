\begin{definition}
    Sean $D\subseteq\Rn{2}$ y $\boldsymbol{\Sigma}:D\to\Rn{3}$ de clase $\mathcal{C}^1$ y continua en todo $D$. Sea define la \textbf{superficie} de $S$ como $S=\text{Im}(\boldsymbol{\Sigma})$ y se dice que $\boldsymbol{\Sigma}$ es una \textbf{parametrizaci\'on} de $S$.
\end{definition}

\begin{definition}
    Sea $\boldsymbol{\Sigma}:D\to\Rn{3}$ una parametrizaci\'on de $S$. Se define a $\Sigma$ como una \textbf{parametrizaci\'on regular} en $(u_0, v_0)$ si
    \[
        \boldsymbol{\Sigma}_u(u_0,v_0)\times\boldsymbol{\Sigma}(u_0,v_0)\neq \boldsymbol{0}.
    \]
    En tal caso, $S$ se llama superficie regular en $\boldsymbol{\Sigma}(u_0,v_0)$. Si $S$ es regular entonces admite una parametrizaci\'on regular en todo el interior de su dominio.
\end{definition}

\begin{definition}
    Sea $S\subseteq\Rn{3}$. Se define a $S$ como \textbf{orientable} si existe un campo vectorial $\boldsymbol{\eta}:S\to\Rn{3}$, continuo talque $\boldsymbol{\eta}(p)$ es ortogonal a $S$ en $p$ y $\norm{\boldsymbol{\eta}(p)}=1\;\forall p\in S$. Al campo $\boldsymbol{\eta}$ se lo llama \textbf{orientaci\'on} de $S$.
\end{definition}

\textcolor{red}{Grafico superficie orientable y no orientable (cinta Möbius)}

\begin{obs}
    Sea $S$ una superficie orientable. Entonces $S$ admite dos orientaciones posibles: $\boldsymbol{\eta}$ y $-\boldsymbol{\eta}$.
\end{obs}

\begin{definition}
    Sea $S\subseteq\Rn{3}$ una superficie orientada por $\boldsymbol{\eta}$. Sea $\boldsymbol{\Sigma}:D\subseteq\Rn{2}\to S$ una parametrizaci\'on continua de clase $\mathcal{C}^1$ y regular de $S$.
    \begin{itemize}
        \item Si \[
            \frac{\boldsymbol{\Sigma}_u\times\boldsymbol{\Sigma}_v}{\norm{\boldsymbol{\Sigma}_u\times\boldsymbol{\Sigma}_v}}(u,v)=\boldsymbol{\eta}\left(\boldsymbol{\Sigma}(u,v)\right),
        \]  
        decimos que $\boldsymbol{\Sigma}$ \textbf{preserva} la orientaci\'on de $S$.
        \item Si \[
            \frac{\boldsymbol{\Sigma}_u\times\boldsymbol{\Sigma}_v}{\norm{\boldsymbol{\Sigma}_u\times\boldsymbol{\Sigma}_v}}(u,v)=-\boldsymbol{\eta}\left(\boldsymbol{\Sigma}(u,v)\right),
        \]
        decimos que $\boldsymbol{\Sigma}$ \textbf{invierte} la orientaci\'on de $S$.  
    \end{itemize}
\end{definition}

\begin{definition}
    Sea $S\subseteq\Rn{3}$ una superficie orientada por $\boldsymbol{\eta}$. Sea $\mathbf{F}:S\to\Rn{3}$ un campo continuo en $S$. Se define la integral de superficie de $\mathbf{F}$ sobre $S$ y se onta por $\iint_S\mathbf{F}\;d\mathbf{S}$ a
    \[
        \iint_S \mathbf{F}\;d\mathbf{S}=\iint_S \mathbf{F}\cdot\boldsymbol{\eta}\;dS.
    \]
\end{definition}

\textcolor{red}{Propiedad?} Sea $\boldsymbol{\Sigma}:D\subseteq\Rn{2}\to S\subseteq\Rn{3}$ una parametrizaci\'on de $S$ tal que $\boldsymbol{\Sigma}$ preserve la orientaci\'on, es decir
\[
    \frac{\boldsymbol{\Sigma}_u\times\boldsymbol{\Sigma}_v}{\norm{\boldsymbol{\Sigma}_u\times\boldsymbol{\Sigma}_v}}(u,v)=\boldsymbol{\eta}\left(\boldsymbol{\Sigma}(u,v)\right).
\]
Entonces, por definici\'on, 
\begin{align*}
    \iint_S \mathbf{F}\cdot d\mathbf{S}&=\iint_S \mathbf{F}\cdot\boldsymbol{\eta}\;dS=\iint_D (\mathbf{F}\cdot\boldsymbol{\eta})\circ\boldsymbol{\Sigma}\:\norm{\boldsymbol{\Sigma}_u\times\boldsymbol{\Sigma}_v}\;dudv\\[.2cm]
    &=\iint_D (\mathbf{F}\circ\boldsymbol{\Sigma})\cdot(\boldsymbol{\eta}\circ\boldsymbol{\Sigma})\norm{\boldsymbol{\Sigma}_u\times\boldsymbol{\Sigma}_v}\;dudv=\iint_D \mathbf{F}\circ\boldsymbol{\Sigma}\cdot\left(\boldsymbol{\Sigma}_u\times\boldsymbol{\Sigma}_v\right)dudv.
\end{align*}

\begin{obs}
    Sea $S$ una superficie con orientaci\'on $\boldsymbol{\eta}$. Sea $\boldsymbol{\Sigma}:D\subseteq\Rn{2}\to S\subseteq\Rn{3}$, una parametrizaci\'on de $S$ que invierte la orientaci\'on. Entonces
    \[
        \iint_S \mathbf{F}\cdot d\mathbf{S} = \iint_S \mathbf{F}\cdot\eta\;dS=-\iint_D \mathbf{F}\circ\boldsymbol{\Sigma}\cdot(\boldsymbol{\Sigma}_u\times\boldsymbol{\Sigma}_v)\:dudv.
    \]
\end{obs}

