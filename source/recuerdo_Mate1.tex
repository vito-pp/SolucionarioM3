\begin{definition}[Función en $\mathbb{R}$]
    Se llama función a la relación $f: A \subseteq \mathbb{R} \rightarrow B \subseteq \mathbb{R}$ tal que a cada elemento de
    $A$ le corresponde uno y solo un elemento de $B$.
    Si tomamos como ejemplo a: $f: A \subseteq \mathbb{R} \rightarrow B \subseteq \mathbb{R} / f(x)=2x-1$,
    \begin{itemize}
        \item $f$ es el nombre de la función.
        \item $A$ es el dominio.
        \item $B$ es el codominio.
        \item $x$ es la variable.
        \item $2x-1$ es la fórmula de la función.
    \end{itemize}
\end{definition}
\begin{definition}[Dominio de una función en $\mathbb{R}$]
    Se llama dominio de una función \newline$f: A \subseteq \mathbb{R} \rightarrow B \subseteq \mathbb{R}$
    al conjunto de entrada del mismo. Es decir, $\text{Dom}(f)=A$.
\end{definition}
\begin{definition}[Codominio de una función en $\mathbb{R}$]
    Se llama codominio de una función \newline $f: A \subseteq \mathbb{R} \rightarrow B \subseteq \mathbb{R}$
    al conjunto al que se restringen las salidas del mismo. Es decir que $\text{Cod}(f)=B$.
    Puede que no todos los elementos de $B$ correspondan a una evaluación de la función.
\end{definition}
\begin{definition}[Imagen de una función en $\mathbb{R}$]
    Se llama imagen de una función \newline $f: A \subseteq \mathbb{R} \rightarrow B \subseteq \mathbb{R}$ 
    al conjunto 
    \begin{equation*}
        \text{Im}(f) = \left\{ y\in B: \exists x\in A / f(x)=y \right\}    
    \end{equation*}
\end{definition}
\begin{definition}[Gráfica de una función en $\mathbb{R}$]
    La gráfica de una función $f: A \subseteq \mathbb{R} \rightarrow B \subseteq \mathbb{R}$ es
    el conjunto
    \begin{equation*}
        \text{Graf}(f) = \left\{ (x,y)\in \mathbb{R}: x \in A \land y=f(x)\right\}    
    \end{equation*}
\end{definition}
\begin{example}
    Si se tiene la función ejemplo $f: A \subseteq \mathbb{R} \rightarrow B \subseteq \mathbb{R} / f(x)=2x-1$.
    Se puede armar la siguiente tabla de valores:
    \begin{table}[H]
        \centering
        \begin{tabular}{|c|c|}
            \hline
            $x$ & $y=f(x)$ \\ \hline
            -2  & 4        \\ \hline
            -1  & 1        \\ \hline
            0   & 0        \\ \hline
            1   & 1        \\ \hline
            2   & 4        \\ \hline
            \end{tabular}
      \end{table}
    Luego, el siguiente conjunto de puntos, se encuentran incluidos en la
    gráfica de la función (que cabe aclarar, tiene infinitos elementos):
    \begin{equation*}
        \left\{ (-2,4),(-1,1),(0,0),(1,1),(2,4) \right\}\subset \text{Graf}(f)
    \end{equation*}
    De hecho, si se fuera a posicionar los puntos de $\text{Graf}(f)$, se obtendría
    lo que se entiende normalmente si se habla de la gráfica de la función. De hecho,
    graficando la función como uno lo haría normalmente, en realidad
    se están dibujando los infinitos puntos pertenecientes
    al $\text{Graf}(f)$ sobre el plano.
    \begin{center}
        \begin{tikzpicture}
            \begin{axis}[
                axis lines = middle,
                xmin = -2.5, xmax = 2.5,
                ymin = 0, ymax = 4.5,
                xlabel = {$x$},
                ylabel = {$y$},
                domain = -2:2,
                samples = 100,
                xtick={-2,-1,1,2},
                ytick={1,2,3,4},
                clip=false,
                ]
            
                % Define the curves
                \addplot[name path=A, blue, thick, domain=-2:2] {x^2};
                \fill[red] (0,0) circle (4pt);
                \fill[red] (-2,4) circle (4pt);
                \fill[red] (-1,1) circle (4pt);
                \fill[red] (1,1) circle (4pt);
                \fill[red] (2,4) circle (4pt);
            \end{axis}
        \end{tikzpicture}
        \end{center}
\end{example}
\begin{definition}[Sobreyectividad, Inyectividad y Biyectividad]
    Se tiene una función:
    \begin{equation*}
        f: A \subseteq \mathbb{R} \rightarrow B \subseteq \mathbb{R}
    \end{equation*}
    La función se dice sobreyectiva si se cumple que su codominio es igual a su imagen: 
    \begin{equation*}
        \text{Cod}(f)=\text{Img}(f)
    \end{equation*}
    Es decir que cada elemento del conjunto de salida es la evaluación de la función en al menos un elemento del
    dominio.

    La función se dice inyectiva si a cada elemento en el conjunto imagen le corresponde un único
    valor del dominio. Es decir, que la función es inyectiva cuando se cumple:
    \begin{equation*}
        a=b \Leftrightarrow f(a)=f(b)
    \end{equation*}
    Tal que $a,b\in \text{Dom}(f)$.

    Finalmente, la función es sobreyectiva cuando es simultáneamente sobreyectiva e inyectiva.
\end{definition}