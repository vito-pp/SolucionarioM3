\begin{definition}
   Se llama \textit{función} a la regla que asigna a cada elemento de un conjunto
    $A$  un y s\'olo un elemento de un conjunto $B$. Tomaremos, en esta secci\'on,  el caso  donde $A, B \subseteq \mathbb{R}.$   La manera habitual de denotar una funci\'on $f$ es $$f: A \subseteq \mathbb{R} \rightarrow B \subseteq \mathbb{R}.$$  A los conjuntos $A$ y $B$  se los llama, respectivamente, \textit{dominio}  y \textit{codominio} de la funci\'on y  la notaci\'on usual para los mismos  es  $A=\text{Dom}(f)$ y $B=\text{Cod}(f)$.
\end{definition}


\begin{definition} Dada una funci\'on $f: A \subseteq \mathbb{R} \rightarrow B \subseteq \mathbb{R},$  se llama \textit{imagen} de una función y se nota  $\text{Im}(f)$, a
    al conjunto 
    \begin{equation*}
        \text{Im}(f) = \left\{ y\in B: \exists\;x\in A\:|\:f(x)=y \right\}.    
    \end{equation*}
\end{definition}

\begin{definition}\label{grafm1}
   Dada una función $f: A \subseteq \mathbb{R} \rightarrow B \subseteq \mathbb{R}$ se define la \textit{gráfica} de $f$, y se nota  $\text{Graf}(f)$, al conjunto
    \begin{equation*}
        \text{Graf}(f) = \left\{ (x,y)\in \mathbb{R}^{2}: x \in A \land y=f(x)\right\}.    
    \end{equation*}
\end{definition}


\textcolor{red}{Aca hay un lio en los ejemplos, la tabla y el grafico no corresponden a la formula. Hacer dos ejemplos completo por separado. }


\begin{example}
   Sea $$f:  [0,\infty]  \subseteq \mathbb{R} \rightarrow  \mathbb{R} \:|\:  f(x)=2x-1$$ tenemos que, $f$ es el nombre de la función, $x$ es la variable independiente de $f$  y $2x-1$ su  fórmula, es decir,  el valor que la asigna $f$ a cada $x \in  [0,\infty]. $ Adem\'as,   $\text{Dom}(f)=  [0,\infty]$,   $\text{Cod}(f)=  \mathbb{R}$ y $\text{Im}(f)= [-1,\infty]$ 
 \end{example}



\begin{example}  Sea $$f: A  \subseteq \mathbb{R} \rightarrow  \mathbb{R} \:|\:  f(x)=..........$$ 

,   armar la siguiente tabla de valores:
    \begin{table}[H]
        \centering
        \begin{tabular}{|c|c|}
            \hline
            $x$ & $y=f(x)$ \\ \hline
            -2  & 4        \\ \hline
            -1  & 1        \\ \hline
            0   & 0        \\ \hline
            1   & 1        \\ \hline
            2   & 4        \\ \hline
            \end{tabular}
      \end{table}
    Luego, el siguiente conjunto de puntos, se encuentran incluidos en la
    gráfica de la función (que cabe aclarar, tiene infinitos elementos).
    \begin{equation*}
        \left\{ (-2,4),(-1,1),(0,0),(1,1),(2,4) \right\}\subset \text{Graf}(f)
    \end{equation*}
  
  \textcolor{red}{hacer dos dibujos, el primero solo con los puntos rojos y el segundo con los mismos puntos mas el trazo. Decir algo asi como: por conocimiemto previo sabemos que la manera de unir los puntos es a trav\'es de..... }
  
    De hecho, si se fuera a posicionar los puntos de $\text{Graf}(f)$, se obtendría
    lo que se entiende normalmente si se habla de la gráfica de la función. De hecho,
    graficando la función como uno lo haría normalmente, en realidad
    se están dibujando los infinitos puntos pertenecientes
    al $\text{Graf}(f)$ sobre el plano.
    \begin{center}
        \begin{tikzpicture}
            \begin{axis}[
                axis lines = middle,
                xmin = -2.5, xmax = 2.5,
                ymin = 0, ymax = 4.5,
                xlabel = {$x$},
                ylabel = {$y$},
                domain = -2:2,
                samples = 100,
                xtick={-2,-1,1,2},
                ytick={1,2,3,4},
                clip=false,
                ]
            
                % Define the curves
                \addplot[name path=A, blue, thick, domain=-2:2] {x^2};
                \fill[red] (0,0) circle (4pt);
                \fill[red] (-2,4) circle (4pt);
                \fill[red] (-1,1) circle (4pt);
                \fill[red] (1,1) circle (4pt);
                \fill[red] (2,4) circle (4pt);
            \end{axis}
        \end{tikzpicture}
        \end{center}
\end{example}
\begin{definition}
  
  Sea la funci\'on $f: A \subseteq \mathbb{R} \rightarrow B \subseteq \mathbb{R}$. Diremos que
   $f$ es \textit{sobreyectiva} si se cumple que su codominio es igual a su imagen, esto es
   \begin{equation*}
        \text{Cod}(f)=\text{Im}(f).
    \end{equation*}   Es decir que cada elemento del conjunto de salida es la evaluación de la función en al menos un elemento del dominio.

 Diremos que $f$ es   \textit{inyectiva} si a cada elemento en el conjunto imagen le corresponde un único
    valor del  conjunto dominio, esto es,   si  $a,b\in \text{Dom}(f)$, entonces:
     \begin{equation*}
        a=b \iff f(a)=f(b),
    \end{equation*} 

    Finalmente, diremos que $f$  es \textit{biyectiva} cuando es simultáneamente \textit{sobreyectiva} e \textit{inyectiva} a la vez.
\end{definition}