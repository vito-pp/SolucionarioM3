\begin{definition}
    Sea $f:U\subset\Rn{n}\to\R$, $U$ abierto, diferenciable en $U$. Se define el \textbf{gradiente} de $f$, y si nota $\grad f(\mathbf{x})$, evaluado en $\mathbf{x}$ como el vector en el espacio $\Rn{n}$ dado por 
    \[
        \grad f(\mathbf{x})=\left(\frac{\partial f(\mathbf{x})}{\partial x_1},\frac{\partial f(\mathbf{x})}{\partial x_2},\:\ldots,\: \frac{\partial f(\mathbf{x})}{\partial x_n}\right),
    \]
    donde $\mathbf{x}=(x_1, x_2,\ldots, x_n)$.   
\end{definition}
\begin{definition}
    Sea $\mathbf{F}:U\subset\Rn{n}\to\Rn{n}$, $U$ abierto, de clase $\mathcal{C}^1(U)$. Se define la \textbf{divergencia} de $\mathbf{F}$ como
    \[
        \grad\cdot \mathbf{F}(\mathbf{x})=\frac{\partial f_1(\mathbf{x})}{\partial x_1}+\frac{\partial f_2(\mathbf{x})}{\partial x_2}+\:\ldots+\: \frac{\partial f_n(\mathbf{x})}{\partial x_n},
    \]
    donde $\mathbf{F}=(f_1,f_2,\ldots,f_n),$ con cada $f_i:\Rn{n}\to\R\;(i=1,\ldots,n).$
\end{definition}
\begin{definition}
    Sea $\mathbf{F}:U\subset\Rn{3}\to\Rn{3}$, $U$ abierto, de clase $\mathcal{C}^1(U)$. Se define el \textbf{rotor} de $\mathbf{F}$ como 
    \[
        \grad\times\mathbf{F}(\mathbf{x})=\left(\frac{\partial f_3(\mathbf{x})}{\partial y}-\frac{\partial f_2(\mathbf{x})}{\partial z},\;\frac{\partial f_1(\mathbf{x})}{\partial z}-\frac{\partial f_3(\mathbf{x})}{\partial x},\;\frac{\partial f_2(\mathbf{x})}{\partial x}-\frac{\partial f_1(\mathbf{x})}{\partial y}\right),
    \]   
    donde $\mathbf{F}(x,y,z)=(f_1(x,y,z),\:f_2(x,y,z),\:f_3(x,y,z))$.
    Y para el caso en el que $\mathbf{F}:U\subset\Rn{2}\to\Rn{n}$, con las mismas condiciones, el \textbf{rotor} es
    \[
        \grad\times\mathbf{F}(\mathbf{x})=\left(\frac{\partial f_2(\mathbf{x})}{\partial x}-\frac{\partial f_1(\mathbf{x})}{\partial y}\right).
    \]
\end{definition}

\begin{obs}
    Estas f\'ormulas son m\'as faciles de recordar si pensamos a ``nabla'' $\grad$ como el operador tal que 
    \[
        \grad = (\frac{\partial}{\partial x_1},\frac{\partial}{\partial x_2},\:\ldots,\: \frac{\partial}{\partial x_n}).  
    \]
    Por lo que, pensando a $\grad$ como un vector, la divergencia es el producto escalar de $\grad$ con un campo vectorial y para el caso en $\Rn{3}$ es el producto vectorial. Esto \'ultimo es
    \begin{align*}
        \grad\times\mathbf{F}&= 
        \begin{vmatrix}
        \mathbf{i} & \mathbf{j} & \mathbf{k} \\
        \partialx & \partialy & \partialz \\
        f_1 & f_2 & f_3
        \end{vmatrix} \\[.2cm]
        &=\left(\frac{\partial f_3}{\partial y}-\frac{\partial f_2}{\partial z},\;\frac{\partial f_1}{\partial z}-\frac{\partial f_3}{\partial x},\;\frac{\partial f_2}{\partial x}-\frac{\partial f_1}{\partial y}\right).
    \end{align*}
\end{obs}
\begin{definition}
    Sea $f:U\subset\Rn{n}\to\R$, $U$ abierto, un campo escalar de clase $\mathcal{C}^2(U)$. Se define el \textbf{laplaciano} de $f$, y se lo nota $\Delta f$ o $\grad^2 f$, a
    \[
        \Delta f=\grad^2f=\grad\cdot(\grad f).  
    \]
    Y para el caso de un campo vectorial $\mathbf{F}:U\subset\Rn{n}\to\Rn{n}$ de clase $\mathcal{C}^2(\Rn{n})$, el \textbf{laplaciano} escalar
    \[
        \Delta\mathbf{F}=(\Delta f_1,\Delta f_2,\ldots,\Delta f_n). 
    \]
\end{definition}

\textcolor{red}{AGREGAR EJEMPLOS}

\begin{definition}
    Sea un campo vectorial $\mathbf{F}$. Si $\grad\times\mathbf{F}=0$, lo llamamos \textbf{irrotacional}. Si $\grad\cdot\mathbf{F}=0$, lo llamamos \textbf{solenoidal}.
\end{definition}
\begin{definition}
    Sea un campo escalar $f$ o vectorial $\mathbf{F}$ vectorial, lo llamamos \textbf{arm\'onico} si $\Delta f=0$ o $\Delta \mathbf{F}=0$.
\end{definition}
\begin{propertie}
    Los operadores vectoriales son aplicaciones lineales. Sean $f:U\subseteq\Rn{n}\to\R$ diferenciable y $\mathbf{F}:U\subseteq\Rn{n}\to\Rn{n}$ diferenciable. Sean $\alpha,\;\beta\in\R$:
    \begin{enumerate}
        \item \(\grad(\alpha f(\mathbf{x})+\beta g(\mathbf{x}))=\alpha\grad f(\mathbf{x})+\beta\grad g(\mathbf{x}),\quad\forall\mathbf{x}\in U\)
        \item \(\grad\cdot(\alpha \mathbf{F}(\mathbf{x})+\beta \mathbf{G}(\mathbf{x}))=\alpha\grad\cdot\mathbf{F}(\mathbf{x})+\beta\grad\cdot\mathbf{G}(\mathbf{x}),\quad\forall\mathbf{x}\in U\)
        \item \(\grad\times(\alpha \mathbf{F}(\mathbf{x})+\beta \mathbf{G}(\mathbf{x}))=\alpha\grad\times\mathbf{F}(\mathbf{x})+\beta\grad\times\mathbf{G}(\mathbf{x}),\quad\forall\mathbf{x}\in U\)
    \end{enumerate}
\end{propertie}
\begin{propertie}
    \textbf{Regla de Leibniz para el producto}. Sea $h:\Rn{n}\to\R$ un campo escalar tal que $h\in \mathcal{C}^1$.
    \begin{enumerate}
        \item Sea $\mathbf{F}:U\subseteq\Rn{n}\to\Rn{n}$ de clase $\mathcal{C}^1(U)$, entonces $\grad\cdot(h\mathbf{F})=h\grad\cdot\mathbf{F}+\grad h\cdot\mathbf{F}$
        \item Sea $\mathbf{F}:U\subseteq\Rn{3}\to\Rn{3}$ (o $\Rn{2}$) de clase $\mathcal{C}^1(U)$, entonces $\grad\times(h\mathbf{F})=h\grad\times\mathbf{F}+\grad h\times\mathbf{F}$
    \end{enumerate}
\end{propertie}
\begin{propertie}
    Productos:
    \begin{enumerate}
        \item $\grad(fg)=f\grad g+g\grad f$
        \item $\grad\cdot(\mathbf{F}\times\mathbf{G})=\grad\times\mathbf{F}\cdot\mathbf{G}-\mathbf{F}\cdot\grad\times\mathbf{G}$
    \end{enumerate}
\end{propertie}