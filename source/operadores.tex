
\textcolor{red}{Unificar lo siguiente:  lo nombres en capitulos anteriores no estan en negritas sino que estan con textit }

\begin{definition}
    Sea $f:U\subset\Rn{n}\to\R$ con $U$ abierto y $f$ diferenciable en $U$.   Se define el \textbf{gradiente} de $f$ evaluado en $\mathbf{x}\in U$, y se nota $\grad f(\mathbf{x})$,  como el vector en el espacio $\Rn{n}$ dado por 
    \[
        \grad f(\mathbf{x})=\left(\frac{\partial f(\mathbf{x})}{\partial x_1},\frac{\partial f(\mathbf{x})}{\partial x_2},\:\ldots,\: \frac{\partial f(\mathbf{x})}{\partial x_n}\right),
    \]
  \end{definition}


\begin{definition}
    Sea $\mathbf{F}:U\subset\Rn{n}\to\Rn{n}$ con  $U$ abierto y $\mathbf{F}$ de clase $\mathcal{C}^1(U)$. Se define la \textbf{divergencia} de $\mathbf{F}$ como
    \[
        \grad\cdot \mathbf{F}(\mathbf{x})=\frac{\partial f_1(\mathbf{x})}{\partial x_1}+\frac{\partial f_2(\mathbf{x})}{\partial x_2}+\:\ldots+\: \frac{\partial f_n(\mathbf{x})}{\partial x_n}.
    \]
    donde $\mathbf{F}=(f_1,f_2,\ldots,f_n),$ con cada $f_i :U\subset\Rn{n} \to\R\;(i=1,\ldots,n).$
\end{definition}


\begin{definition}
    Sea $\mathbf{F}:U\subset\Rn{3}\to\Rn{3}$ con $U$ abierto y $\mathbf{F}$ de clase $\mathcal{C}^1(U)$. Se define el \textbf{rotor} de $\mathbf{F}$  evaluado en $\mathbf{x}\in U$, y se nota  $\grad\times\mathbf{F}(\mathbf{x})$,   como el vector en el espacio $\Rn{3}$ dado por 
    \[
        \grad\times\mathbf{F}(\mathbf{x})=\left(\frac{\partial f_3(\mathbf{x})}{\partial y}-\frac{\partial f_2(\mathbf{x})}{\partial z},\;\frac{\partial f_1(\mathbf{x})}{\partial z}-\frac{\partial f_3(\mathbf{x})}{\partial x},\;\frac{\partial f_2(\mathbf{x})}{\partial x}-\frac{\partial f_1(\mathbf{x})}{\partial y}\right),
    \]   
    donde $\mathbf{F}(x,y,z)=(f_1(x,y,z),\:f_2(x,y,z),\:f_3(x,y,z))$.
  
  
    Sea $\mathbf{F}:U\subset\Rn{2}\to\Rn{2}$   con $U$ abierto y $\mathbf{F}$ de clase $\mathcal{C}^1(U)$. Se define el \textbf{rotor} de $\mathbf{F}$  evaluado en $\mathbf{x}\in U$, y se nota  $\grad\times\mathbf{F}(\mathbf{x})$,   como el escalar dado por
    \[
        \grad\times\mathbf{F}(\mathbf{x})=\left(\frac{\partial f_2(\mathbf{x})}{\partial x}-\frac{\partial f_1(\mathbf{x})}{\partial y}\right).
    \]
     donde $\mathbf{F}(x,y)=(f_1(x,y),\:f_2(x,y))$
\end{definition}



\textcolor{red}{Unificar lo siguiente:  lo nombres en capitulos anteriores no estan en negritas sino que estan con textit: laplaciano. irrotacional etc }



\begin{definition}
    Sea $f:U\subset\Rn{n}\to\R$ con  $U$ abierto y $f$ de clase $\mathcal{C}^2(U)$.  Se define el \textbf{laplaciano} de $f$, y se lo nota $\Delta f$ o $\grad^2 f$, a
    \[
        \Delta f=\grad^2f=\grad\cdot\grad f.  
    \]
    Y para el caso de un campo vectorial $\mathbf{F}:U\subset\Rn{n}\to\Rn{n}$ de clase $\mathcal{C}^2(\Rn{n})$, el \textbf{laplaciano} escalar
    \[
        \Delta\mathbf{F}=(\Delta f_1,\Delta f_2,\ldots,\Delta f_n). 
    \]
\end{definition}


\begin{obs}
    Estas f\'ormulas son m\'as f\'aciles de recordar si pensamos a nabla ($\grad$) como   el siguiente operador 
    \[
        \grad = (\frac{\partial}{\partial x_1},\frac{\partial}{\partial x_2},\:\ldots,\: \frac{\partial}{\partial x_n}).  
    \]
    De esta manera,   la divergencia es el producto escalar de $\grad$  con un campo vectorial  y para el caso del rotor en $\Rn{3}$ es el producto cruz  con un campo vectorial. Es decir, 
    
    $$ \grad . \mathbf{F} =  \sum_{i=1}^{n} \frac{\partial f_i }{\partial x_i}  $$
     $$     \grad\times\mathbf{F}= 
                \left(\frac{\partial f_3}{\partial y}-\frac{\partial f_2}{\partial z},\;\frac{\partial f_1}{\partial z}-\frac{\partial f_3}{\partial x},\;\frac{\partial f_2}{\partial x}-\frac{\partial f_1}{\partial y}\right).$$ 
    
\end{obs}

\textcolor{red}{Agregar a la obs lo mismo para el  laplaciano }



\begin{example}
    Sea $f(x,y)=\sin{(x^2+y^2)}$. Calcular $\Delta f(x,y)$.

    Primero calculamos el gradiente de $f$.
    \[
        \grad f(x,y) = (2x\cos{(x^2+y^2)}, 2y\cos{(x^2+y^2)})
    \]
    Luego le aplicamos la divergencia.
    \begin{align*}
        \grad\cdot\grad f (x,y) &= \left(\frac{\partial}{\partial x}, \frac{\partial}{\partial y}\right)\cdot (2x\cos{(x^2+y^2)}, 2y\cos{(x^2+y^2)})\\[.2cm]
        &= 2\cos{(x^2+y^2)}-4x^2\sin{(x^2+y^2)}+2\cos{(x^2+y^2)}-4y^2\sin{(x^2+y^2)}\\[.2cm]
        &=4\cos{(x^2+y^2)}-4(x^2+y^2)\sin{(x^2+y^2)}=\Delta f
    \end{align*}
 \end{example}

\begin{definition}
    Sea un campo vectorial $\mathbf{F}:U\subseteq\Rn{n}\to\Rn{n}$.  Si $\grad\times\mathbf{F}=\mathbf{0}$ se llama  \textbf{irrotacional}.
  \end{definition}  
  
  
  \begin{definition} Sea un campo vectorial $\mathbf{F}:U\subseteq\Rn{n}\to\Rn{n}$.
     Si $\grad\cdot\mathbf{F}=0$  se llama  \textbf{solenoidal}.
\end{definition}
\begin{definition}
    Sea un campo escalar $f$ o un campo vectorial $\mathbf{F}$. Si $\Delta f=0$ o $\Delta \mathbf{F}=0$  se llama  \textbf{arm\'onico}.
\end{definition}

\begin{propertie}
    Sean $f:U\subseteq\Rn{n}\to\R$  y $\mathbf{F}:U\subseteq\Rn{n}\to\Rn{n}$  ambas diferenciables en $U$ y sean $\alpha,\;\beta\in\R$:
    \begin{itemize}
        \item[1.] \(\grad(\alpha f(\mathbf{x})+\beta g(\mathbf{x}))=\alpha\grad f(\mathbf{x})+\beta\grad g(\mathbf{x}),\quad\forall\mathbf{x}\in U\) 
     \end{itemize}   
 
\:\:\:\:  Adem\'as, si ambas son de clase $C^{1}(U)$ 
       \begin{itemize}
        \item[2.]  \(\grad\cdot(\alpha \mathbf{F}(\mathbf{x})+\beta \mathbf{G}(\mathbf{x}))=\alpha\grad\cdot\mathbf{F}(\mathbf{x})+\beta\grad\cdot\mathbf{G}(\mathbf{x}),\quad\forall\mathbf{x}\in U\)
        \item[3.] \(\grad\times(\alpha \mathbf{F}(\mathbf{x})+\beta \mathbf{G}(\mathbf{x}))=\alpha\grad\times\mathbf{F}(\mathbf{x})+\beta\grad\times\mathbf{G}(\mathbf{x}),\quad\forall\mathbf{x}\in U\)
    \end{itemize}
\end{propertie}
\begin{propertie}
    \textbf{Regla de Leibniz para el producto}. Sea $h:\Rn{n}\to\R$ un campo escalar tal que $h\in \mathcal{C}^1$.
    \begin{enumerate}
        \item Sea $\mathbf{F}:U\subseteq\Rn{n}\to\Rn{n}$ de clase $\mathcal{C}^1(U)$, entonces $\grad\cdot(h\mathbf{F})=h\grad\cdot\mathbf{F}+\grad h\cdot\mathbf{F}$
        \item Sea $\mathbf{F}:U\subseteq\Rn{3}\to\Rn{3}$ (o $\Rn{2}$) de clase $\mathcal{C}^1(U)$, entonces $\grad\times(h\mathbf{F})=h\grad\times\mathbf{F}+\grad h\times\mathbf{F}$
    \end{enumerate}
\end{propertie}
\begin{propertie} \textcolor{red}{hipotesis}
    Productos:
    \begin{enumerate}
        \item $\grad(fg)=f\grad g+g\grad f$
        \item $\grad\cdot(\mathbf{F}\times\mathbf{G})=\grad\times\mathbf{F}\cdot\mathbf{G}-\mathbf{F}\cdot\grad\times\mathbf{G}$
    \end{enumerate}
\end{propertie}