\documentclass[10pt, a4paper]{report}

\usepackage{tocloft} % para cambiar formato de \tableofcontents
\usepackage{titlesec} % para cambiar formato de los titulos
\usepackage{amsmath, amssymb, amsthm, mathtools, esint} % Paquetes de simbolos matematicos
\usepackage[skip=20pt, indent=0pt]{parskip} 
\usepackage[margin=3cm]{geometry}
\usepackage[spanish]{babel}
\usepackage{chngcntr} % para cambiar los contadores
\usepackage{hyperref} % para hyperlinks asociados al documento
\usepackage{pgfplots} % para graficos 
\pgfplotsset{compat=1.18}
\usepgfplotslibrary{fillbetween}
\usetikzlibrary{patterns, shapes}

\xdef\Rad{2}
\newcommand{\ARW}[2][]{%
    \foreach \ang in {#2}{%
        \draw[#1] (\ang:\Rad)--(\ang+1:\Rad) ;
    }
}

%%%%%%%%%%%%%%%%%%%%%%%%%%%%
\renewcommand{\contentsname}{Indice}
\renewcommand{\listfigurename}{Lista de Figuras}
\renewcommand{\listtablename}{Lista de Tablas}
\renewcommand{\bibname}{Bibliograf\'{\i}a}
\renewcommand{\indexname}{Indice}
\renewcommand{\figurename}{Figura}
\renewcommand{\tablename}{Tabla}
\renewcommand{\partname}{Parte}
\renewcommand{\chaptername}{Cap\'{\i}tulo}
\renewcommand{\appendixname}{Ap\'endice}
\renewcommand{\abstractname}{Resumen}
%%%%%%%%%%%%%%%%%%%%%%%%%%%%

%%%%%%%%%%%%%%%%%%%%%%%%%%%%%%%%
%  Teoremas y proposiciones
%%%%%%%%%%%%%%%%%%%%%%%%%%%%%%%%%
\theoremstyle{definition} % Para que el texto no aparezca en italicas
\newtheorem{question}{Ejercicio}
\newtheorem{solution}{Solución}
\newtheorem{theorem}{Teorema}[section]
\newtheorem{corollary}{Corolario}[section]
\newtheorem{definition}{Definic\'on}[section]
\newtheorem{propertie}{Propiedad}[section]
\newtheorem{example}{Ejemplo}[section]

% Provide commands for \autoref
\providecommand*{\questionautorefname}{Ejercicio}
\providecommand*{\solutionautorefname}{Solución}
\providecommand*{\theoremautorefname}{Teorema}
\providecommand*{\corollaryautorefname}{Corolario}
\providecommand*{\definitionautorefname}{Definición}
\providecommand*{\propertieautorefname}{Propiedad}
\providecommand*{\exampleautorefname}{Ejemplo}

% Para que se resetee el contador en cada seccion
\counterwithin*{solution}{subsubsection}
\counterwithin*{question}{subsubsection}

% \newtheorem{teo}{Teorema}[section]
% \newtheorem{prop}[teo]{Proposici\'on}
% \newtheorem{lema}[teo]{Lema}
% \newtheorem{obs}[teo]{Observaci\'on}
% \newtheorem{exmpl}[teo]{Ejemplo}
% \newtheorem{coro}[teo]{Corolario}
% \newtheorem{comen}[teo]{Comentario}
% \newtheorem{conje}[teo]{Conjetura}
% \theoremstyle{definition}
% \newtheorem{definition}[teo]{Definici\'on}
% \newtheorem{propertie}[teo]{Propiedad}

% numeros reales
\renewcommand{\Re}{\mathbb {R}}
% derivadas parciales
\newcommand{\partialx}{\frac{\partial}{\partial x}}
\newcommand{\partialy}{\frac{\partial}{\partial y}}
\newcommand{\partialz}{\frac{\partial}{\partial z}}
% numeros romanos
\newcommand*{\rom}[1]{\expandafter\@slowromancap\romannumeral #1@}

\spanishdecimal{.}

% ajuste formato capitulos
\titleformat{\chapter}[display]
  {\normalfont\fontsize{14}{16}\selectfont\bfseries}{\vspace{-4cm}}{0pt}{}[\vspace{-1cm}]

% ajuste formato secciones
\titleformat{\section}
  {\normalfont\fontsize{12}{14}\selectfont\bfseries}{\thesection}{1em}{}

% ajuste formato indice
\renewcommand{\cfttoctitlefont}{\normalfont\Large\bfseries}
\renewcommand{\cftbeforetoctitleskip}{-.5cm}

\begin{document}
%===================================================
%                     CARATULA
%===================================================
    \begin{center}

        \thispagestyle{empty}  % Saca la numeracion a la primer pagina

        \includegraphics[scale=0.2]{logoitba2.png}
        \medskip
        
        \textbf{\large{INSTITUTO  TECNOL\'OGICO DE BUENOS AIRES}\normalsize}
        
        \smallskip
        
        \textbf{\large  Departamento de Ciencias Exactas y Naturales}
        
        \smallskip
        \textbf{\large \'Area  Matem\'atica}
      
        \vspace{2 cm}
        
        \textbf{\Large \noindent \textbf{Parciales Resueltos.}}
             
        \vspace{1.5cm}
        
        \large  Material de lectura para alumnos........  \normalsize
    \end{center}
    
    \newpage

%===================================================
%                   INTRODUCCION
%===================================================
    \setcounter{page}{1}
    \begin{center}
        \textbf{ \noindent \textbf{Parciales Resueltos}}
    \end{center}
    \vspace{3cm}
      
    (Pequena explicacion).  Este documento presentaremos distintos ejercicios de parcial....
    \newpage

%===================================================
%                      INDICE
%===================================================
    \tableofcontents
        \pagenumbering{arabic}
        \setcounter{page}{1}

%===================================================
%                CONTENIDO TEORICO
%===================================================
    \chapter{Contenido teórico. Primera parte}

    \chapter{Contenido Teórico. Segunda parte}
        \section{Operadores}
            \begin{definition}
    Si $f:U\subset\Re^n\to\Re$, $U$ abierto, es diferenciable el \textbf{gradiente} de $f$ en $\mathbf{x}$ es el vector en el espacio $\Re^n$ dado por 
    \[
        \text{grad}\:f=\nabla f=(\frac{\partial f}{\partial x_1},\frac{\partial f}{\partial x_2},\:\ldots,\: \frac{\partial f}{\partial x_n}),
    \]
    donde $\mathbf{x}=(x_1, x_2,\ldots, x_n)$.
\end{definition}
\begin{definition}
    Sea $\mathbf{F}:U\subset\Re^n\to\Re^n$, $U$ abierto, de clase $C^1(\Re^n)$. Entonces la \textbf{divergencia} de $\mathbf{F}$ se define como
    \[
        \text{div}\:\mathbf{F}=\nabla\cdot \mathbf{F}=  \frac{\partial F_1}{\partial x_1}+\frac{\partial F_2}{\partial x_2}+\:\ldots+\: \frac{\partial F_n}{\partial x_n},
    \]
    donde $\mathbf{F}=(F_1,F_2,\ldots,F_n),$ con cada $F_i:\Re^n\to\Re\;(i=1,\ldots,n)$.
\end{definition}
\begin{definition}
    Sea $\mathbf{F}:U\subset\Re^3\to\Re^n$, $U$ abierto, de clase $C^1(\Re^3)$. Entonces el \textbf{rotor} de $\mathbf{F}$ se define como 
    \[
        \text{rot}\:\mathbf{F}=\nabla\times\mathbf{F}=\left(\frac{\partial F_3}{\partial y}-\frac{\partial F_2}{\partial z},\;\frac{\partial F_1}{\partial z}-\frac{\partial F_3}{\partial x},\;\frac{\partial F_2}{\partial x}-\frac{\partial F_1}{\partial y}\right),
    \]   
    donde $\mathbf{F}(x,y,z)=(F_1(x,y,z),\:F_2(x,y,z),\:F_3(x,y,z))$.
    Y para el caso en el que $\mathbf{F}:U\subset\Re^2\to\Re^n$, con las mismas condiciones, el \textbf{rotor} es
    \[
        \text{rot}\:\mathbf{F}=\nabla\times\mathbf{F}=\left(\frac{\partial F_2}{\partial x}-\frac{\partial F_1}{\partial y}\right).
    \]
\end{definition}
    Estas f\'ormulas son m\'as faciles de recordar si pensamos a ``nabla'' $\nabla$ como el operador tal que 
    \[
        \nabla = (\frac{\partial}{\partial x_1},\frac{\partial}{\partial x_2},\:\ldots,\: \frac{\partial}{\partial x_n}).  
    \]
    Por lo que, pensando a $\nabla$ como un vector, la divergencia es el producto escalar de $\nabla$ con un campo vectorial y para el caso en $\Re^3$ es el producto vectorial. Esto \'ultimo es
    \begin{align*}
        \nabla\times\mathbf{F}&= 
        \begin{vmatrix}
        \mathbf{i} & \mathbf{j} & \mathbf{k} \\
        \partialx & \partialy & \partialz \\
        F_1 & F_2 & F_3
        \end{vmatrix} \\[.2cm]
        &=\left(\frac{\partial F_3}{\partial y}-\frac{\partial F_2}{\partial z},\;\frac{\partial F_1}{\partial z}-\frac{\partial F_3}{\partial x},\;\frac{\partial F_2}{\partial x}-\frac{\partial F_1}{\partial y}\right)\\[.2cm]
        &=\text{rot}\:\mathbf{F}.
    \end{align*}
\begin{definition}
    Sea $f:U\subset\Re^n\to\Re$ un campo escalar de clase $C^2(\Re^n)$. Entonces se define el \textbf{laplaciano} de $f$ como
    \[
        \Delta f=\nabla^2f=\nabla\cdot(\nabla f)=\text{div}\:\text{grad}\:\mathbf{F}.  
    \]
    Y para el caso de un campo vectorial $\mathbf{F}:U\subset\Re^n\to\Re^n$ de clase $C^2(\Re^n)$, el \textbf{laplaciano} escalar
    \[
        \Delta\mathbf{F}=(\Delta F_1,\Delta F_2,\ldots,\Delta F_n).  
    \]
\end{definition}
\begin{definition}
    Para un campo vectorial $\mathbf{F}$, si $\nabla\times\mathbf{F}=0$, lo llamamos \textbf{irrotacional}. Y si $\nabla\cdot\mathbf{F}=0$, lo llamamos \textbf{solenoidal}.
\end{definition}
\begin{definition}
    Tanto como para un campo escalar como vectorial, decimos que la funci\'on es \textbf{arm\'onica} si $\Delta f=0$ o $\Delta \mathbf{F}=0$.
\end{definition}
\begin{propertie}
    Los operadores vectoriales son aplicaciones lineales, $\alpha,\;\beta\in\Re$:
    \begin{enumerate}
        \item \(\nabla(\alpha f+\beta g)=\alpha\nabla f+\beta\nabla g\)
        \item \(\nabla\cdot(\alpha \mathbf{F}+\beta \mathbf{G})=\alpha\nabla\cdot\mathbf{F}+\beta\nabla\cdot\mathbf{G}\)
        \item \(\nabla\times(\alpha \mathbf{F}+\beta \mathbf{G})=\alpha\nabla\times\mathbf{F}+\beta\nabla\times\mathbf{G}\)
    \end{enumerate}
\end{propertie}
\begin{propertie}
    \textbf{Regla de Leibniz para el producto}. Sea $h:\Re^n\to\Re$ un campo escalar tal que $h\in C^1$.
    \begin{enumerate}
        \item Sea $\mathbf{F}:\Re^n\to\Re^n$ de clase $C^1$, entonces $\nabla\cdot(h\mathbf{F})=h\nabla\cdot\mathbf{F}+\nabla h\cdot\mathbf{F}$
        \item Sea $\mathbf{F}:\Re^3\to\Re^3$ de clase $C^1$, entonces $\nabla\times(h\mathbf{F})=h\nabla\times\mathbf{F}+\nabla h\times\mathbf{F}$
    \end{enumerate}
\end{propertie}
\begin{propertie}
    Productos:
    \begin{enumerate}
        \item $\nabla(fg)=f\nabla g+g\nabla f$
        \item $\nabla\cdot(\mathbf{F}\times\mathbf{G})=\nabla\times\mathbf{F}\cdot\mathbf{G}-\mathbf{F}\cdot\nabla\times\mathbf{G}$
    \end{enumerate}
\end{propertie}
        \section{Curvas}
            \begin{definition}
    Definimos una curva $\Gamma\in\Re^n$ como la imagen de una trayectoria $\boldsymbol{\sigma}:I\subseteq\Re\to\Re^n$ continua y de clase $C^1$ a trozos. Si $I=[a,b]$ a los puntos $\boldsymbol{\sigma}(a)$ y $\boldsymbol{\sigma}(b)$ se los llama extremos de la curva.\final
\end{definition}

\begin{center}
\begin{tikzpicture}

    % Real number line on the left
    \draw[semithick,-Stealth] (-2,0) -- (2,0);
    \node[label=below:$a$] at (-1.25,0) {$[$}; 
    \node[label=below:$b$] at (1.25,0) {$]$}; 
    \node[label=right:$\mathbb{R}$] at (2,0) {};
    
    % Curved arrow in the middle
    \draw[semithick,->] (2,1) to[out=45,in=135,looseness=0.8] (8,1);
    \node[label=below:$\boldsymbol{\sigma}$] at (5,2) {};
    
    % Plane R^2 on the right
    \begin{scope}[xshift=8cm]
        \draw[semithick] (0,0) .. controls (1,2) and (2,-1) .. (3,1);
        \node at (0,0) [circle,fill,inner sep=1pt] {};
        \node at (3,1) [circle,fill,inner sep=1pt] {};
        \node[below left] at (0,0) {$\boldsymbol{\sigma}(a)$};
        \node[above right] at (3,1) {$\boldsymbol{\sigma}(b)$};
    \end{scope}

\end{tikzpicture}
\end{center}

\begin{center}
\begin{tikzpicture}

    % Real number line on the left
    \draw[semithick,-Stealth] (-2,0) -- (2,0);
    \node[label=below:$a$] at (-1.25,0) {$[$}; 
    \node[label=below:$b$] at (1.25,0) {$]$}; 
    \node[label=right:$\mathbb{R}$] at (2,0) {};

    % Curved arrow in the middle
    \draw[semithick,->] (2,1) to[out=45,in=135,looseness=0.8] (8,1);
    \node[label=below:$\boldsymbol{\sigma}$] at (5,2) {};

    % Plane R^2 on the right
    \begin{scope}[xshift=8cm, yshift=0.4cm]
        \draw[semithick] 
            (0,0) .. controls (4.5,-1) and (-1.5,-2.5) .. (3,1);
        \node at (0,0) [circle,fill,inner sep=1pt] {};
    \node at (3,1) [circle,fill,inner sep=1pt] {};
    \node[below left] at (0,0) {$\boldsymbol{\sigma}(a)$};
    \node[above right] at (3,1) {$\boldsymbol{\sigma}(b)$};
    \end{scope}

\end{tikzpicture}
\end{center}

\textcolor{red}{desp agrego las leyendas}

Si $\boldsymbol{\sigma}(a)=\boldsymbol{\sigma}(b)$ entonces $\Gamma$ se llama curva cerrada.

\begin{center}
\begin{tikzpicture}

    % Real number line on the left
    \draw[semithick,-Stealth] (-2,0) -- (2,0);
    \node[label=below:$a$] at (-1.25,0) {$[$}; 
    \node[label=below:$b$] at (1.25,0) {$]$}; 
    \node[label=right:$\mathbb{R}$] at (2,0) {};
    
    % Curved arrow in the middle
    \draw[semithick,->] (2,1) to[out=45,in=135,looseness=0.8] (8,1);
    \node[label=below:$\boldsymbol{\sigma}$] at (5,2) {};
    
    % Plane R^2 on the right
    \begin{scope}[xshift=8.2cm, yshift=0.5cm]
        \draw[semithick]
            plot[smooth,tension=1.2] coordinates {(1,0.6) (3,0) (2.5,-1.5) (1,-1) (0,0) (1,0.6)};
    
            % \foreach \point in {(1,0.6), (3,0), (2.5,-1.5), (1,-1), (0.2,0.1), (1,0.6)} {
            % \node at \point [circle,fill,inner sep=1pt] {};
            % }
        \node at (0,0) [circle,fill,inner sep=1pt] {};
        \node[below left] at (0,0) {$\boldsymbol{\sigma}(a)=\boldsymbol{\sigma}(b)$};
    \end{scope}

\end{tikzpicture}
\end{center}

Adem\'as, si $\boldsymbol{\sigma}$ es inyectiva en $I$, salvo tal vez en sus bordes, entonces se llama a $\Gamma$ curva simple.

\begin{example}
    Sea la trayectoria $\boldsymbol{\sigma}:[0,2\pi]\to\Gamma$, tal que $\boldsymbol{\sigma}(t)=(\cos t, \sen t)$.

    \begin{center}
    \begin{tikzpicture}

        % Real number line on the left
        \draw[semithick,-Stealth] (-2,0) -- (2,0);
        \node[label=below:$0$] at (-1.25,0) {$[$}; 
        \node[label=below:$2\pi$] at (1.25,0) {$]$}; 
        \node[label=right:$\mathbb{R}$] at (2,0) {};
    
        % Curved arrow in the middle
        \draw[->, semithick] (2,1) to[out=45,in=135,looseness=0.8] (6,1);
        \node[label=below:$\boldsymbol{\sigma}$] at (4,1.7) {};
    
        % Plane R^2 on the right
        \begin{scope}[xshift=8.5cm]
            % Circle
            \draw[semithick] (0,0) circle [radius=2];
            % Axes
            \draw[-Stealth, thin] (-2.5,0) -- (2.5,0) node[right] {$x$};
            \draw[-Stealth, thin] (0,-2.5) -- (0,2.5) node[above] {$y$};
            \node at (-2,-2) [align=center, below] {
                $\Gamma=\text{Im}(\boldsymbol{\sigma})$
            };
            \node at (2,0) [circle,fill,inner sep=1pt] {};
            \node[below left] at (2,0) {$\boldsymbol{\sigma}(0)=\boldsymbol{\sigma}(2\pi)$};
        \end{scope}
    
    \end{tikzpicture}
    \end{center}
    
\end{example}

\begin{definition}
    Continuidad a trozos. [...] \textcolor{red}{No est\'a en el Tromba y no s\'e como definirlo formalmente}.\final
\end{definition}

Cada curva simple $\Gamma$ tiene asociadas dos orientaciones o sentidos posibles. Si los puntos $P$ y $Q$ son los extremos de la curva entonces podemos considerar a $\Gamma$ con orientaci\'on desde $P$ hacia $Q$ o bien desde $Q$ hacia $P$.

\begin{definition}
    Parametrizaci\'on de una curva. Una parametrizaci\'on de una curva simple $\Gamma\in\Re^n$ es una trayectoria $\boldsymbol{\sigma}:I\subseteq\Re\to\Re^n$ continua, de clase $C^1$ a trozos, inyectiva en $\text{int}\:(I)$ y $\text{Im}\:(\boldsymbol{\sigma})=\Gamma$.\final
\end{definition}

\begin{definition}
    Una paremetrizaci\'on $\boldsymbol{\sigma}:I\subset\Re\to\Re^n$ se llama regular si $\boldsymbol{\sigma}'(t)\neq0$.\final
\end{definition}
        \section{Integrales Dobles} 
            \begin{definition}
    Sea $D\subseteq\Re^2$. $D$ se llama conjunto de tipo 1 si
    \[
        D=\{(x,y)\in\Re^2: a\leq x\leq b;\;\phi_1(x)\leq y\leq \phi_2(x),\;\text{con }\phi_1,\;\phi_2\text{ continuas}\}.
    \]
\end{definition}
\begin{definition}
    Sea $D\subseteq\Re^2$. $D$ se llama conjunto de tipo 2 si
    \[
        D=\{(x,y)\in\Re^2: c\leq y\leq d;\;\psi_1(y)\leq x\leq \psi_2(y),\;\text{con }\psi_1,\;\psi_2\text{ continuas}\}.
    \]
\end{definition}
\begin{definition}
    Sea $D\subseteq\Re^2$. $D$ se llama conjunto de tipo 1 y tipo 2 a la vez.
\end{definition}
\begin{definition}
    A \'estos conjuntos (tipo 1, 2, 3) se los llama elementales.
\end{definition}
\begin{corollary} \label{col:fubini}
    \textcolor{red}{corolario de que teorema es?}
    \begin{enumerate}
        \item Sea $D\subseteq\Re^2$ un conjunto de tipo 1. Supongamos que $f:D\to\Re$ es continua y las $\phi_i$ son continuas. Entonces
        \[
            \iint_D f=\int_a^b\left(\int_{y=\phi_1(x)}^{y=\phi_2(x)}f(x,y)\:dy\right)dx.  
        \]
        \item Sea $D\subseteq\Re^2$ un conjunto de tipo 2. Supongamos que $f:D\to\Re$ es continua y las $\psi_i$ son continuas. Entonces
        \[
            \iint_D f=\int_a^b\left(\int_{x=\psi_1(y)}^{x=\psi_2(y)}f(x,y)\:dx\right)dy.  
        \]
        \item Si $D$ es de tipo 3 y $f,\;\phi_i,\psi_i\;$ son continuas, entonces 
        \[
            \iint_D f=\int_a^b\left(\int_{y=\phi_1(x)}^{y=\phi_2(x)}f(x,y)\:dy\right)dx=\int_a^b\left(\int_{x=\psi_1(y)}^{x=\psi_2(y)}f(x,y)\:dx\right)dy. 
        \]
    \end{enumerate}
\end{corollary}
\begin{example}
    Se quiere calcular la integral $\iint_D f$, donde $f(x,y)=x^2y$, mientras que $D$ es la regi\'on del plano encerrado entre $y=x^3$, $y=x^2$, con $x\in[0,1]$. \textcolor{red}{agregar figura}

    En este caso 
    \[
        D=\{(x,y)\in\Re^2:0\leq x\leq1;\;x^3\leq y\leq x^2\}.  
    \]
    Luego $D$ es de tipo 1. Notar que la \'unica manera de demostrar que un conjunto es elemental es dando su descrici\'on impl\'icita. Por el \autoref{col:fubini}
    \begin{gather*}
        \iint_D x^2y\:dA=\int_0^1\left(\int_{x^3}^{x^2}\:dy\right)dx=\int_0^1\left(x^2\frac{1}{2}y^2\Big\lvert_{x^3}^{x^2}\right)dx\\
        =\int_0^1\frac{1}{2}x^2\left((x^2)^2-(x^3)^2\right)=\int_0^1\frac{1}{2}x^6-\frac{1}{2}x^8\;dx\\=\frac{1}{14}x^7-\frac{1}{18}x^9\Big\lvert_0^1=\frac{1}{14}-\frac{1}{18}.
    \end{gather*}
\end{example}
\begin{definition} %cambie la palabra region --> conjunto
    Se define el \'area de un conjunto $D\subset\Re^n$ como la integral, si existe, de la funci\'on 1. Es decir,
    \[
        \text{A}(D)=\int_D1\:dA.  
    \]
\end{definition}
\begin{theorem}
    \textbf{Teorema del valor medio para integrales dobles}. Suponer que $f:D\to\Re$ es continua y $D$ es un conjunto elemental. Entonces para alg\'un punto $(x_0,y_0)$ en $D$, tenemos
    \[
        \int_D f(x,y)\:dA=f(x_0,y_0)\text{A}(D).
    \]
\end{theorem}
\begin{theorem} % lo escribi como lo tenia en mis apuntes
    Sea $f:D\subseteq\Re^n\to\Re^2$ un campo escalar integrable. Sea $T:D^*\subseteq\Re^n\to D$ continua de clase $C^1$ en int($D^*$), inyectiva en int($D^*$) y $T(D^*)=D$. Entonces:
    \begin{enumerate}
        \item 
        \[
            \iint_D f\:dA=\iint_{D^*}f\circ T|\text{det}(\boldsymbol{D}_T)|\:dA\quad(n=2)
        \]
        \item 
        \[
            \iiint_D f\:dV=\iiint_{D^*}f\circ T|\text{det}(\boldsymbol{D}_T)|\:dV\quad(n=3)
        \]
    \end{enumerate}
    % supuse que ya esta definida la matriz diferencial D_T
    Se suele notar a 
    \[
        |\text{det}(\boldsymbol{D}_T)|=J(T)
    \]
    y se lo llama \textbf{jacobiano} de la matriz diferencial $\boldsymbol{D}_T$.
\end{theorem}
        \section{Integrales Triples}
            \textcolor{red}{definir paralelep\'ipedo rectangular B en R3. }
\textcolor{red}{definir particion de un  paralelep\'ipedo rectangular. quien es $\Delta V_{ijk}$  quien es $V_{ijk}$. Explicar la notacion que aparece en la primer definicion }
\textcolor{red}{se podria hacer algun dibujo} 


\begin{definition}
Dada una funci\'on continua $f:U\to\R$, donde $U$ es un paralelep\'ipedo rectangular (una caja) en $\Rn{3}$, se define la suma de Reimann de $f$ sobre $U$, partiendo los tres lados de $U$ en $n$ partes iguales, a
\[
    S_n=\sum_{i=0}^{n-1}\sum_{j=0}^{n-1}\sum_{k=0}^{n-1}f(c_{ijk})\Delta V_{ijk},
\]  
donde $c_{ijk}\in U_{ijk}$, el $ijk$-\'esimo paralelep\'ipedo rectangular en la partici\'on de $U$, y $\Delta V_{ijk}$ es el volumen de $U_{ijk}$.
\end{definition}

\begin{definition} 
    Sean $U=[a,b]\times[c,d]\times[p,q]$ y $f:U\to\R$ una funci\'on.  Diremos que $f$ es integrable sobre $U$ si existe y es finito el  l\'imite $\lim_{n\to\infty}S_n$, en tal caso  el valor de dicho l\'imite  se llama \textbf{integral triple} de $f$ sobre $U$ y se nota 
    \[
          \iiint_U f\:dV=\iiint_U f(x,y,z)\:dxdydz.
    \]
\end{definition}

\begin{obs}
\textcolor{red}{agregar condiciones sobre f para la  existencia del limite}
\end{obs}


Para extender esta noci\'on de integral a un conjunto acotado m\'as general, esto es, conjuntos que puedan ser encerrados por una caja, definimos lo siguiente. 

\begin{definition}
Dada $f:U\to\R$, se define la funci\'on $f^*$ tal que
\[
    f^*(x,y,z)=
    \begin{dcases*}
        f(x,y,z) & si $(x,y,z)\in W$ \\[.2cm]
        0        & si $(x,y,z)\notin W.$
    \end{dcases*}
\]
Entonces si $U$ es una caja que contiene a $W$ y $\partial W$ est\'a formada por las gr\'aficas de un n\'umero finito de funciones continuas, $f^*$ ser\'a integrable.  Se define
\[
    \iiint_W f\:dV=\iiint_U f^*\:dV.  
\]
\end{definition}

\begin{obs} 
   Esta definici\'on es independiente de la selecci\'on de $U$.
\end{obs}

\begin{propertie}
    Sean $f,\;g$ dos funciones integrables en una regi\'on $W\subset\Rn{3}$, entonces:
    \begin{enumerate}
        \item[i.] $\alpha f+\beta g$ es integrable en $W$, $\forall\;\alpha,\;\beta\in\R$ y adem\'as
        \[
            \iiint_W \left(\alpha f+\beta g\right)dV=\alpha\iiint_W f\:dV+\beta\iiint_W g\:dV.
        \]
        \item[ii.] El producto $fg$ es integrable en $W$.
        \item[iii.] Si $|g(x,y,z)|\geq k>0\;\forall(x,y,z)\in W$, el cociente $f/g$ es integrable en $W$.
        \item[iv.] Si $f\geq 0 $ en $W$, $\iiint_W f\:dV\geq0$.
        \item[v.]Si $f\leq g$ en $W$, $\iiint_W f\:dV\leq\iiint_W g\:dV.$
        \item[vi.]Si $|f|$ es integrable en $W$, entonces 
        \[
            \left|\iiint_W f\:dV\right|\leq\iiint_W|f|\:dV.  
        \]    
        \item[vii.] Si $W=W_1\cup W_2, y \;W_1\cap W_2\neq\varnothing,$ es una partici\'on de $W$, $f$ es integrable en $W$ $\iff$ $f$ es integrable en $W_1$ y $W_2$. En este caso 
        \[
            \iiint_W f\:dV=\iiint_{W_1} f\:dV+\iiint_{W_2}f\:dV.    
        \]
    \end{enumerate}
\end{propertie}

\begin{definition}
    Se define el volumen de un conjunto $W\subset\Rn{3}$  y se nota $\text{Vol}(W)$, a
    \[
        \text{Vol}(W)=\iiint_W1\:dV.
    \]
\end{definition}

\begin{theorem}
    \textbf{Teorema del valor medio}. Sea $f:W\subseteq\Rn{3}\to\Rn{3}$ continua y $W$ es un conjunto elemental. Entonces existe $(x_0,y_0,z_0)$ en $W$ tal que 
    \[
        \iiint_W f\:dV=f(x_0,y_0,z_0)\text{Vol}(W).
    \]
\end{theorem}

\begin{definition}  \textcolor{red}{No queda claro quien es el conjunto D.  }
    \textbf{Conjunto elemental en }$\Rn{3}$.
    Sea $W\subseteq\Rn{3}$. Sean $\phi_1,\;\phi_2:[a,b]\to\R$ funciones y sean $\gamma_1,\;\gamma_2:D\to\R$ funciones continuas.  Un conjunto $W\subseteq\Rn{3}$ se llama elemental, si puede ser escrito  \textcolor{red}{de algunas de las siguientes 6 maneras: es largo pero escribirlas todas, ojo con la notacion} como el conjunto de puntos $(x,y,z)$ que satifacen
    \begin{equation*}
        x\in[a,b], \qquad \phi_1(x)\leq y\leq\phi_2(x), \qquad \gamma_1(x,y)\leq z\leq\gamma_2(x,y). 
    \end{equation*}
    O bien, intercambiando el orden de las variables.
\end{definition}


 \textcolor{red}{Ilustracion}


\begin{theorem}
    \textbf{Teorema del cambio de variables para integrales triples}. Sea $f:W\subseteq\Rn{3}\to\R$  un campo escalar integrable. Sea $\mathbf{T}:W^*\subseteq\Rn{3}\to W\subseteq\Rn{3}$ continua, de clase $\mathcal{C}^1$ en $\interior{(W^*)}$,  inyectiva en $\interior{(W^*)}$ y tal que $\mathbf{T}(W^*)=W$. Entonces
    \[
        \iiint_W f\:dV=\iiint_{W^*}(f\circ\mathbf{T})J_\mathbf{T}\:dV.
    \]
\end{theorem}

        \section{Integrales de L\'inea}
            \textcolor{red}{No hace falta la parte de recuerdo de mate 1 con el teorema de valor medio?}

\textcolor{red}{Falta definicion integral de linea sobre trayectorias}

\begin{definition}
    Sea $\Gamma$ una curva simple en $\Re^n$ y sea $\boldsymbol{\sigma}:I\subseteq\Re\to\Re^n$ una parametrizaci\'on regular de $\Gamma$. Entonces
    \[
        \text{long}\:\Gamma=\text{L}(\Gamma)=\sum_{i=1}^{n}\int_{I_i}\|\boldsymbol{\sigma}'(t)\|dt,    
    \]
    donde $\boldsymbol{\sigma}\lvert_{I_i}$ es de clase $C^1$.
\end{definition}

\begin{definition}
    Sea $\Gamma$ una cirva simple en $\Re^n$ y sea $\boldsymbol{\sigma}:[a,b\subseteq\Re\to\Re^n]$ una parametrizaci\'on regular de $\Gamma$. Sea $f:A\subseteq\Re^n\to\Re$ continua sobre $\Gamma$. Se define la integral de l\'inea del campo $f$ sobre la curva $\Gamma$, y se nota por $\int_{\Gamma}f\:ds$ a
    \[
        \int_{\Gamma}f\:ds=\sum_{i=1}^{n}\int_{I_i}f\circ\boldsymbol{\sigma}(t)\|\boldsymbol{\sigma}'(t)\|\:dt,  
    \]
    donde $\boldsymbol{\sigma}\lvert_{I_i}$ es de clase $C^1$.
\end{definition}

\textbf{Aplicaci\'on.}
Cuando $f(x,y)>0$ esta integral tiene una interpretaci\'on geom\'etrica como el ``\'area de una valla". Podemos construir una ``valla" cuya base sea la imagen de $\boldsymbol{\sigma}$ y altura $f(x,y)$ en $(x,y)$. Si $\boldsymbol{\sigma}$ recorre s\'olo una vez la imagen de $\boldsymbol{\sigma}$, la integral $\int_{\boldsymbol{\sigma}}f(x,y)\:ds$ representa el \'area de una lado de la valla.

\textcolor{red}{ilustracion de la aplicacion}

\begin{definition}
    Integral de un campo vectorial a lo largo de una trayectoria.
    Sean $\mathbf{F}:A\subseteq\Re^n\to\Re^n$ un campo vectorial y $\boldsymbol{\sigma}:[a,b]\to A\subseteq\Re^n$ una trayectorial $C^1$ a trozos tales que $\mathbf{F}\circ\boldsymbol{\sigma}:[a,b]\to\Re^n$ es una funci\'on continua. Se define la \textbf{integral de} $\mathbf{F}$ \textbf{a lo largo de} $\boldsymbol{\sigma}$ como
    \[
        \int_{\boldsymbol{\sigma}}\mathbf{F}\cdot d\mathbf{s}=\int_a^b\mathbf{F}\circ\boldsymbol{\sigma}(t)\cdot\boldsymbol{\sigma}'(t)\:dt.  
    \]
\end{definition}

\textbf{Notaci\'on.} $\int_{\boldsymbol{\sigma}}\mathbf{F}\cdot d\mathbf{s}=\int_{\boldsymbol{\sigma}}\mathbf{F}\cdot\hat{\mathbf{t}}\:ds$, donde $\hat{\mathbf{t}}$ es versor tangencial al campo $\mathbf{F}$ en todo punto.

\begin{definition}
    Integral de un campo vecrorial a lo largo de una curva simple.
    Sea $\mathbf{F}:A\subseteq\Re^n\to\Re^n$ un campo vectorial continuo sobre una curva simple oriantada $\Gamma\subset A$. Sea $\boldsymbol{\sigma}:[a,b]\to A\subseteq\Re^n$ una parametrizaci\'on de $\Gamma$ que preserva su oriantaci\'on. Se define la \textbf{integral de} $\mathbf{F}$ \textbf{sobre la curva simple} $\boldsymbol{\Gamma}$ como
    \[
        \int_{\Gamma}\mathbf{F}\cdot d\mathbf{s}=\int_{\boldsymbol{\sigma}}\mathbf{F}\cdot d\mathbf{s}  
    \]
\end{definition}

\textbf{Notaci\'on.} Si $\Gamma$ es cerrada, se suele notar a la integral como
\[
    \oint_{\Gamma}\mathbf{F}\cdot d\mathbf{s},  
\]
y se la llama integral de circulaci\'on de $\mathbf{F}$ sobre $\Gamma$.

        \section{Integrales de Superficie}
        \section{Campos Conservativos}
            \begin{definition}
    \textcolor{red}{Definicion conjunto conexo...}
\end{definition}

\begin{theorem} \label{thm:t1}
    Sean $A\subseteq\Re^n$ abierto y conexo y $f:A\subseteq\Re^n\to\Re$ de clase $C^1$. Sea $\boldsymbol{\sigma}:[a,b]\to A\subseteq\Re^n$ una trayectoria $C^1$ a trozos. Entonces
    \[
       \int_{\boldsymbol{\sigma}}\nabla f\cdot d\mathbf{s}=f(\boldsymbol{\sigma}(b))-f(\boldsymbol{\sigma}(a)).
    \]    
\end{theorem}

\begin{corollary}
    En las condiciones del \autoref{thm:t1}, $\int_{\boldsymbol{\sigma}}\nabla f\cdot d\mathbf{s}$ no depende de $\boldsymbol{\sigma}$, s\'olo depende de los puntos inicla y final de la trayectoria. Si $\boldsymbol{\sigma}_1$, $\boldsymbol{\sigma}_2$ son trayectorias $C^1$ a trozos tales que $\boldsymbol{\sigma}_1:[a,b]\to A\subseteq\Re^n$ y $\boldsymbol{\sigma}_2:[a,b]\to A\subseteq\Re^n$, entonces
    $$\int_{\boldsymbol{\sigma}_1}\nabla f\cdot d\mathbf{s}=\int_{\boldsymbol{\sigma}_2}\nabla f\cdot d\mathbf{s}.$$
\end{corollary}

\begin{corollary}
    En las condiciones del \autoref{thm:t1}, si $\mathbf{F}:A\subseteq\Re^n\to\Re^n$ es un campo vectorial para el cual $\exists f:A\subseteq\Re^n\to\Re$ clase $C^1$ tal que 
    $$\nabla f(\mathbf{x})=\mathbf{F}(\mathbf{x})\quad\forall\:\mathbf{x}\in A,$$
    entonces para todo par de puntos $\mathbf{p}_0,\:\mathbf{p}_1\in A$
    $$\int_{C_1}\mathbf{F}\cdot d\mathbf{s}=\int_{C_2}\mathbf{F}\cdot d\mathbf{s},$$
    para cualquier par de curvas $C_1,\;C_2\subset A$ que vayan de $\mathbf{p}_0$ a $\mathbf{p}_1$ 
\end{corollary}

\begin{corollary}
    En las condiciones del \autoref{thm:t1}, si $\mathbf{F}=\nabla f$ Entonces
    $$\oint_C\mathbf{F}\cdot d\mathbf{s}=0,$$
    para toda curva $C$ simple cerrada contenida en $A$.
\end{corollary}

\begin{theorem}
    Sean $A\subseteq\Re^n$ abierto y conexo, $\mathbf{F}:A\subseteq\Re^n\to\Re^n$ continuo tal que $\int_C \mathbf{F}\cdot d\mathbf{s}$ es independiente del camino en $A$. Entonces $f:A\subseteq\Re^n\to\Re$ tal que
    $$f(\mathbf{x})=\int_{\mathbf{x}_0}^{\mathbf{x}}\mathbf{F}\cdot d\mathbf{s},$$
    con $\mathbf{x}_0\in A$, cumple que: 
    \begin{enumerate}
        \item $f$ es $C^1$.
        \item $\nabla f(\mathbf{x})=\mathbf{F}(\mathbf{x})\quad\forall\:\mathbf{x}\in A.$
    \end{enumerate}
\end{theorem}

\begin{theorem}
    Sean $A\subseteq\Re^n$ abierto y conexo, $\mathbf{F}:A\subseteq\Re^n\to\Re$ continuo. Las siguientes condiciones sobre $\mathbf{F}$ son equivalentes:
    \begin{enumerate}
        \item $\exists f:A\subseteq\Re^n\to\Re$ de clase $C^1$ tal que $\mathbf{F}=\nabla f.$ 
        \item La integral de $\mathbf{F}$ es independiente del camino en $A$.
        \item $\oint_C\mathbf{F}\cdot d\mathbf{s}=0$ para toda curva cerrada simple $C$ en $A$.
    \end{enumerate}
\end{theorem}
        \section{Teoremas Integrales}
            \begin{definition}
    Sea $\boldsymbol{\Sigma}:D\subseteq\Rn{2}\to S\subseteq\Rn{3}$ una parametrizaci\'on inyectiva para todo $D$ en $S$. Se define el \textbf{borde} o \textbf{frontera} de $S$, y se lo nota $\partial S$, a la curva cerrada simple en $\Rn{3}$ tal que 
    \[
        \partial S=\boldsymbol{\Sigma}(\partial D).\finalmath
    \]
\end{definition}

\begin{obs} 
    Si $\partial D$ est\'a orientada de manera antihoraria, entonces $\boldsymbol{\Sigma}$ \textbf{induce} una orientaci\'on sobre $\partial S$. Adem\'as $\boldsymbol{\Sigma}$ \textbf{induce} una orientaci\'on sobre $S$.
\end{obs}

\begin{definition}
    Sea $D$ un conjunto conexo (en $\Rn{2}$ o en $\Rn{3}$). $D$ se llama \textbf{simplemente conexo} si toda curva cerrada simple $C\subset D$ es el borde de una regi\'on $R$ totalmente contenida en $D$.\final
\end{definition}

\begin{theorem}
    \textbf{Teorema de Green}. Sea $D\in\Rn{2}$ una regi\'on simplemente conexa y sea $\partial D$ la frontera de $D$, con orientaci\'on antihoraria. Y Sea $\mathbf{F}:D\to\Rn{2}$ de clase $C^1$. Entonces
    \[
        \oint_{\partial D}\mathbf{F}\cdot d\mathbf{s} = \iint_D \grad\times\mathbf{F}\cdot d\mathbf{A}.\finalmath
    \]
\end{theorem}

\begin{obs} 
    Otra manera de enunciar el teorema de Green es usando un cambio en la notaci\'on vectorial. Pensando el producto escalar $\mathbf{F}\cdot d\mathbf{s}$ como $P\:dx+Q\:dy$, donde $\mathbf{F}=(P,Q)$ y sabiendo que el rotor en $\Rn{2}$ es la componente $z$ del rotor en $\Rn{3}$; reescribimos el teorema de Green como
\[
        \int_{\partial D}P\:dx+Q\:dy=\iint_D\left(\frac{\partial Q}{\partial x}-\frac{\partial P}{\partial y}\right)dxdy.
\]
\end{obs}

\begin{obs}
    \textbf{Extensi\'on de Green.} El teorema de Green es aplicable a regiones a\'un m\'as generales que simplemente conexas. De hecho, a cualquier regi\'on en $\Rn{2}$ cuya frontera se pueda descomponer en un n\'umero finito de curvas cerradas simples orientadas se le puede aplicar el teorema de Green. La idea es ``recorrer" dicha frontera pasando por todas las curvas que la confroman. 
    
    Por ejemplo, sea $D=D_1\cup D_2\cup D_3$ una regi\'on en $\Rn{2}$ no simplemente conexa, como muestra la figura a continuaci\'on.
    
    \begin{center}
    \begin{tikzpicture}
        % Define the exterior boundary with clockwise arrows
        \fill[blue!60!white!40] (0,0) circle (4);
        \foreach \angle in {315, 270, 225, 180, 135, 90, 45, 0} {
            \draw[thick, ->] ({4*cos(\angle+10)}, {4*sin(\angle+10)}) 
                arc[start angle=\angle+10, end angle=\angle-45+10, radius=4];
        }
        
        \draw[thick, dashed] (-3.96, 0.5) -- (3.96, 0.5);
        \draw[thick, dashed] (-3.7, -1.5) -- (3.7, -1.5);
        
        % Inner circle 1 with counterclockwise arrows
        \fill[white] (-1,0.5) circle (1.5);
        \foreach \angle in {0, 90, 180, 270} {
            \draw[thick, ->] ({-1+1.5*cos(\angle+80)}, {0.5+1.5*sin(\angle+80)}) 
            arc[start angle=\angle+80, end angle=\angle+170, radius=1.5];
            }
            
            % Inner circle 2 with counterclockwise arrows
        \fill[white] (1.5,-1.5) circle (0.7);
        \foreach \angle in {0, 90, 180, 270} {
            \draw[thick, ->] ({1.5+0.7*cos(\angle+80)}, {-1.5+0.7*sin(\angle+80)}) 
                arc[start angle=\angle+80, end angle=\angle+170, radius=0.7];
                }

        % Labels for regions
        \node at (0,3.3) {$D_1$};
        \node at (0.6,-0.5) {$D_2$};
        \node at (0,-3) {$D_3$};
        \node at (-2, 4) {$C$};
        \node at (-1,1.2) {$C_1$};
        \node at (1.5,-1.5) {$C_2$};
        \node at (-3.2, 1) {$\underleftarrow{\;\;L_1\;\;}$};
        \node at (-3.2, 0) {$\overrightarrow{\:-L_1\:}$};
        \node at (2.2, 1) {$\underleftarrow{\;\;L_2\;\;}$};
        \node at (2.2, 0) {$\overrightarrow{\:-L_2\:}$};
        \node at (-2.5, -1) {$\underleftarrow{\;\;L_3\;\;}$};
        \node at (-2.5, -2) {$\overrightarrow{\:-L_3\:}$};
        \node at (2.8, -1) {$\underleftarrow{\;\;L_4\;\;}$};
        \node at (2.8, -2) {$\overrightarrow{\:-L_4\:}$};
    \end{tikzpicture}
    \end{center}

    El borde de $D$ es $\partial D=C\cup C_1\cup C_2$. Y las orientaciones est\'an definidas tal que $C$ sea horaria y $C_1$ y $C_2$ sean antihorarias. Para poder aplicar el teorema de Green, se puede pensar en subdividir la regi\'on $D$ en tres regiones que s\'i sean simplemente conexas $D_1$, $D_2$ y $D_3$. Dichas regiones las delimitan las curvas $L_1$, $L_2$, $L_3$ y $L_4$. Notar que al ``recorrer'' cada una de las regiones subdivididas, las integrales sobre $L_1$, $L_2$, $L_3$ y $L_4$ se ``cancelan'' entre las regiones lim\'itrofes ya que poseen orientación opuesta. 
    
    Por lo tanto, si $\mathbf{F}$ es un campo vectorial definido en $D$, aplicando el teorema de Green.
    \begin{align*}
        \oint_{\partial D} \mathbf{F}\cdot d\mathbf{s}=&\oint_{C}\mathbf{F}\cdot d\mathbf{s}+\oint_{C_1}\mathbf{F}\cdot d\mathbf{s}+\oint_{C_2}\mathbf{F}\cdot d\mathbf{s}\\[.2cm]
        =&\oint_{\partial D_1}\mathbf{F}\cdot d\mathbf{s}+\oint_{\partial D_2}\mathbf{F}\cdot d\mathbf{s}+\oint_{\partial D_1}\mathbf{F}\cdot d\mathbf{s}\\[.2cm]
        =&\iint_{ D_1}\grad\times\mathbf{F}\cdot d\mathbf{A}+\iint_{ D_2}\grad\times\mathbf{F}\cdot d\mathbf{A}+\iint_{ D_1}\grad\times\mathbf{F}\cdot d\mathbf{A}\\[.2cm]
        =&\iint_D \grad\times\mathbf{F}\cdot d\mathbf{A}
    \end{align*}

\end{obs}

\begin{theorem}
\textbf{Teorema de la divergencia en el plano}. Sea $D\subset\Rn{2}$ una regi\'on donde valga el teorema de Green y sea $\partial D$ su frontera. Denotaremos por $\mathbf{n}$ el versor unitario normal a $\partial D$ en todo punto. Si $\boldsymbol{\sigma}:[a,b]\to\Rn{2},\;\boldsymbol{\sigma}(t)=(x(t),y(t))$ es una parametrizaci\'on orientada de manera positiva de $\partial D$, $\mathbf{n}$ est\'a dado por 
\[
    \mathbf{n}=\frac{(y'(t),-x'(t))}{\sqrt{[x'(t)]^2+[y'(t)]^2}}.
\]
Sea $\mathbf{F}$ un campo vectorial $C^1$ en $D$. Entonces
\[
    \int_{\partial D}\mathbf{F}\cdot\mathbf{n}\:ds=\iint_D\grad\cdot\mathbf{F}\:dA.\finalmath
\]
\end{theorem}

\begin{theorem}
\textbf{Teorema de Stokes}. Sea $S\subset\Rn{3}$ una superficie parametrizada por $\boldsymbol{\Sigma}:D\subset\Rn{3}to S$, donde las orientaciones de $S$ y $\partial S$ fueron inducidas por $\boldsymbol{\Sigma}$. Si $\mathbf{F}:$ es un campo vectorial $C^1$ en $S$. Entonces
\[
    \oint_{\partial S}\mathbf{F}\cdot d\mathbf{s}=\iint_S\grad\times\mathbf{F}\cdot d\mathbf{A}.\finalmath
\]
\end{theorem}

\begin{theorem}
\textbf{Teorema de la divergencia o de Gauss}. Sea $S\subset\Rn{3}$ una superficie que encierra un volumen $\Omega$ orientada de manera exterior, esto es $S=\partial \Omega$. Sea $\mathbf{F}:\Omega\to\Rn{3}$ un campo vectorial clase $C^1$, entonces
\[
    \oiint_{\partial \Omega}\mathbf{F}\cdot d\mathbf{S}=\iiint_{\Omega}\grad\cdot\mathbf{F}\:dV.\finalmath
\]
\end{theorem}

%===================================================
%               EJERCICIOS RESUELTOS
%===================================================
    \chapter{Primeros parciales resueltos}
        \section{Fecha 29 de septiembre de 2023}
            \begin{question}
    Calcule, si existe, el siguiente l\'imite.

    \[
        \lim_{(x,y)\to(0,0)}\sen\left(\frac{\pi(x^{2}-y^{2})}{2\sqrt{x^{2}+y^{2}}}\right) \cos\left(\frac{x^{2}-y^{2}}{x^{2}+y^{2}}\right)
    \]
\end{question}

\begin{question}
    Sea $f:\Rn{2}\to\R$ un campo escalar diferenciable y sea $g:\Rn{2}\to\Rn{2}$ dada por: $$g(x,y)=\left(e^{xy}-1,\;\sen(\pi x+\pi y)\right).$$   \noindent Sabiendo que el gr\'afico $h=f\circ g$ en el punto $(1,0,h(1,0))$ tiene plano tangente de ecuaci\'on $$z-1=\pi x+(\pi+1) y,$$  hallar $f_{\mathbf{v}}(0,0)$ para la direcci\'on ${\mathbf{v}}=\left(\frac{1}{\sqrt{2}},-\frac{1}{\sqrt{2}}\right)$.
\end{question}

\begin{question}
    Considere el campo escalar $f:\mathbb{R}^{2}\rightarrow\mathbb{R}$ dado por:  $$f(x,y)=e^{\sen(x)+y^{2}}$$
    \noindent Se pide aproximar $f(-0\text{.}1,\:0\text{.}2)$ mediante un polinomio de grado dos adecuado.
\end{question}

\begin{question}
    Dado el campo escalar $\displaystyle f(x,y)=\cos(y)+\sen(x)$. Encuentre los puntos cr\'iticos de dicho campo en el dominio $\Omega$ y clasif\'iquelos como extremos locales, donde:  \[\Omega=\left\{(x,y)\in\mathbb{R}^{2}: -\pi<x<\pi,\;\;-\pi<y<\pi \right\}\]
\end{question}

\newpage

\begin{solution}
    Notemos que la estructura del límite pedido está dada por el producto de dos funciones acotadas,  por lo tanto, con mostrar que una de las dos tiende a cero bastar\'a  para decir que el límite de dicho producto es cero.

    Utilizando la siguiente desigualdad para el argumento del seno
    \[ \\[2pt]
        0 \leq \bigg| \frac{\pi(x^{2}-y^{2})}{2\sqrt{x^{2}+y^{2}}} \bigg| \leq
        \frac{2(x^2+y^2)}{\sqrt{x^2+y^2}} = 2 \sqrt{x^2+y^2},
    \]
    junto al teorema de intercalaci\'on tenemos que
    \[
        \lim_{(x,y)\to(0,0)}\bigg| \frac{\pi(x^{2}-y^{2})}{2\sqrt{x^{2}+y^{2}}}\bigg|=0.
    \]
    Recordando que $|\sen(x)| \leq |x| \; \; \forall x \in \mathbb{R}$ podemos concluir que
    \[
        \lim_{(x,y)\to(0,0)}\sen\left(\frac{\pi(x^{2}-y^{2})}{2\sqrt{x^{2}+y^{2}}}\right)=0.
    \]
    Por \'ultimo, recordando que $|\cos(x)| \leq 1 \; \; \forall x \in \mathbb{R}$  tenemos que
    \[
        \lim_{(x,y)\to(0,0)}\sen\left(\frac{\pi(x^{2}-y^{2})}{2\sqrt{x^{2}+y^{2}}}\right) \cos\left(\frac{x^{2}-y^{2}}{x^{2}+y^{2}}    \right)=0.
    \]
\end{solution}


\begin{solution}

    Como   $h:\Rn{2}\to\R$ es diferenciable en todo su dominio por ser composición de funciones diferenciables,  la ecuaci\'on de su plano tangente en el punto $(1,0)$ est\'a dada por
    \begin{equation}
        z= h(1,0) + \grad h(1,0) (x-1,y),  \label{eq:zNabla}
    \end{equation}   luego ser\'a  suficente con encontar $ h(1,0)$ y $\grad h(1,0).$

    Por un lado,      $h(1,0)= f\circ g (1,0) =  f(0,0)$  y por otro lado,  como $f$ y $g$ son ambas diferenciables,  por la regla de la cadena,  tenemos que
    \begin{equation}
        \grad h(1,0)=\grad (f\circ g)(1,0)=\grad f(g(1,0)) \:\boldsymbol{D}_g(1,0) = \grad f (0,0) \:\boldsymbol{D}_g(1,0),  \label{eq:hNabla}
    \end{equation}    donde $\boldsymbol{D}_g$ es la matriz diferencial o  jacobiana de $g$.

    \noindent  Hallemos $\boldsymbol{D}_g$
    \begin{align*}
        \boldsymbol{D}_g(1,0) & =
        \left(\begin{array}{cc}
                      \displaystyle\partialx e^{xy}-1            & \displaystyle\partialy e^{xy}-1           \\[10pt]
                      \displaystyle\partialx  \sen (\pi x+\pi y) & \displaystyle\partialy \sen (\pi x+\pi y)
                  \end{array}\right)\left.\rule{0pt}{1.1cm}\right\rvert_{(1,0)}             \\[2pt]
                              & =\left(\begin{array}{cc}
                                               \displaystyle ye^{xy}                 & \displaystyle xe^{xy}              \\[5pt]
                                               \displaystyle   \pi\cos (\pi x+\pi y) & \displaystyle \pi\cos(\pi x+\pi y)
                                           \end{array}\right)\left.\rule{0pt}{0.7cm}\right\rvert_{(1,0)} \\[2pt]
                              & =\left(\begin{array}{cc}
                                               0    & 1    \\
                                               -\pi & -\pi
                                           \end{array}\right)
    \end{align*}
    Luego, reemplazando en  a   \eqref{eq:hNabla}
    \[
        \grad h(1,0) = \grad f(0,0)\left(\begin{array}{cc}
                0    & 1    \\
                -\pi & -\pi
            \end{array}\right) = \left(-\pi f_y(0,0),\;f_x(0,0)-\pi f_y(0,0)\right)
    \]

    Reemplazando en \eqref{eq:zNabla}
    \begin{align*}
        z & = f(0,0)+\left(-\pi f_y(0,0),\;f_x(0,0)-\pi f_y(0,0)\right) (x-1,y)  \\
        z & =f(0,0)   +\pi f_y(0,0)   -  \pi f_y(0,0)x+(f_x(0,0)-\pi f_y(0,0))y,
    \end{align*}
    y con la información de la consigna, se despejan
    \[\begin{cases}
            \;f_y(0,0)=-1 \\[5pt]
            \;f_x(0,0)-\pi f_y(0,0)=\pi+1
        \end{cases}
        \iff
        \begin{cases}
            \;f_y(0,0)=-1 \\[5pt]
            \;f_x(0,0)=1
        \end{cases}
    \]
    $\therefore\quad\grad f(0,0)=(1,-1)$.

    Como $f$ es diferenciable en todo su dominio, en particular lo es en $(0,0)$, vale que  $$f_{\mathbf{v}}(0,0)=\grad f(0,0)\cdot{\mathbf{v}}\;\;\;\;\;  \forall \; {\mathbf{v}} \in \mathbb{R}^{2}  \mbox{ con } \|{\mathbf{v}}\|=1. $$ Luego,  tomando  $ {\mathbf{v}} = (\frac{1}{\sqrt{2}},\frac{-1}{\sqrt{2}})$, obtenemos lo pedido, es decir,
    \[
        f_{{\mathbf{v}}} (0,0)=(1,-1)\cdot {\mathbf{v}}= \sqrt{2}.
    \]
\end{solution}

\begin{solution}
    Dado que $f$ es de clase  $C^2(\mathbb{R}^2)$,  se definine  el polinomio de Taylor de segundo orden centrado en $(0,0)$ de $f$, que noateremos por $P_2[f,(0,0)]$ como,
    \begin{equation}
        P_2[f,(0,0)](x,y)=f(0,0)+\grad f(0,0)\cdot(x,y)+\frac{1}{2}(x,y)\boldsymbol{H}_f(0,0)\begin{pmatrix}x\\y\end{pmatrix}, \label{eq:polTay2}
    \end{equation}
    donde $\boldsymbol{H}_f$ es la matriz hessiana de $f$.

    Hallemos los  términos del polinomio.

    \begin{itemize}
        \item[1.] $f(0,0)=1.$
        \item[2.] $ \grad f(0,0) = \left(e^{\sen(x)+y^2}\cos\:(x),\;e^{\sen(x)+y^2}2y\right)\Big\rvert_{(0,0)}=(1,0).$
        \item[3.] \begin{align*}
                  \boldsymbol{H}_f(0,0) & =\left(
                  \renewcommand{\arraystretch}{2} % Increase row spacing
                  \begin{matrix}
                      e^{\sen(x)+y^2}[\cos^2(x)-\sen(x)] & 2y\cos(x)e^{\sen(x)+y^2}             \\
                      2y\cos(x)e^{\sen(x)+y^2}           & 4y^2e^{\sen(x)+y^2}+2e^{\sen(x)+y^2}
                  \end{matrix}\right)\left.\rule{0pt}{1.1cm}\right\rvert_{(0,0)} \\
                  \boldsymbol{H}_f(0,0) & =\left(\begin{matrix}
                                                         1 & 0 \\0&2
                                                     \end{matrix}\right).
              \end{align*}
    \end{itemize}

    Ahora sustituimos en la expresión \eqref{eq:polTay2}.
    \begin{align*}
        P_2(x,y) & =1+(1,0)\cdot(x,y)+\frac{1}{2}(x,y)\left(\begin{matrix}1&0\\0&2\end{matrix}\right)\begin{pmatrix}x\\y\end{pmatrix} \\
        P_2(x,y) & =1+x+\frac{1}{2}(x^2+2y^2)                                                                                         \\
        P_2(x,y) & =\frac{1}{2}x^2+y^2+x+1
    \end{align*}

    Entonces queda evaluar en $P_2\left(-0\text{.}1,\:0\text{.}2\right) = \frac{189}{200} = 0\text{.}945$.
    $$\therefore\;f\left(-0\text{.}1,\:0\text{.}2\right)\approx0\text{.}945.$$
\end{solution}

\begin{solution}
    En este ejercicio debemos hallar y clasificar los extremos de la función $f$ sobre $\Omega$,  un conjunto acotado y abierto.
    Para ello,    primero buscamos los puntos críticos de $f$ en $\Omega$. Como $f$ es diferenciable en  $\Omega$, basta con hallar todos los $(x_0,y_0) \in  \Omega$ tal que $\grad f(x_0,y_0)=(0,0).$
    \[
        \grad f(x,y)=\left(\cos(x),\;-\sen(y)\right)
    \]
    \[
        \grad f(x_0,y_0) =0 \iff \begin{cases}
            \cos(x_0)=0 \\ \sen(y_0)=0
        \end{cases} \iff \begin{cases}
            x_0 = \frac{\pi}{2}+k\pi \\
            y_0 = k'\pi
        \end{cases};k,k'\in\mathbb{Z}
    \]
    Ahora sumamos la condición de que pertenezcan a $\Omega$.
    \[
        (x_0,y_0)\in\Omega\iff x_0=-\frac{\pi}{2}\;\lor\;x_0=\frac{\pi}{2}\quad\land\quad y_0=0
    \]
    Por lo tanto,  el conjunto de puntos críticos es
    \[
        P.C.=\{(-\pi/2,0),\;(\pi/2,0)\}.
    \]
    Ahora, para clasificarlos, debemos aplicar el criterio de la segunda derivada. Para ésto calculamos la matriz hessiana de $f$.
    \[
        \boldsymbol{H}_f(x,y)=\left(\begin{matrix}
                -\sen(x) & 0 \\ 0 & -\cos(y)
            \end{matrix}\right)\implies \text{det}(\boldsymbol{H}_f)(x,y)=\sen(x)\cos(y)
    \]

    \begin{itemize}
        \item[1.] $\text{det}(\boldsymbol{H}_f)(-\pi/2,0)=-1<0 \implies f\:\:\text{tiene un punto silla en}\:\: (-\pi/2,0).$
        \item[2.] $\text{det}(\boldsymbol{H}_f)(\pi/2,0)=1>0 \:\:\land\:\: f_{xx}(\pi/2,0)=1>0\implies f\:\:\text{tiene un mínimo} \\
                  \quad \text{local en}\:\:(\pi/2,0).$
    \end{itemize}
\end{solution}


        \newpage
        \section{Fecha 25 de abril de 2023}
            %------------Ejercicio 1---------------------------------------

\begin{question}
    Analizar la existencia de los siguientes límites

    \[
        (a) \quad \lim_{(x,y)\to(0,1)} \frac{x^2(y-1)\cos(\frac{1}{y-1})}{x^2+3(y-1)^2}
        \hfill
        \qquad\qquad(b) \quad \lim_{(x,y)\to(0,0)} \frac{\sen(x^3)y}{x^2-y+x^5}
    \]

\end{question}

%------------Ejercicio 2---------------------------------------

\begin{question}
    Sea \(f: \mathbb{R}^2 \to \mathbb{R}\) tal que
    \[
        f(x,y) =
        \begin{dcases}
            \frac{x^4}{(x^2-y)^2+x^4} & (x,y) \neq (0,0) \\
            0                         & (x,y) = (0,0)
        \end{dcases}
    \]
    \begin{enumerate}
        \item Analizar la continuidad de $f$ en el origen
        \item Analizar la diferenciabilidad de $f$ en el origen.
        \item Hallar, si existen, las derivadas paricales en el origen.
    \end{enumerate}
\end{question}

%------------Ejercicio 3---------------------------------------

\begin{question}
    Sea \(g(x,y) = yx^2+\sen(f(x,y))\) con $f$ un campo escalar \(C^1(\mathbb{R}^2)\) tal que \(f(0,0)=0\). Calcular
    \[
        \lim_{(x,y)\to(0,0)} \frac{g(x,y)-xf_x(0,0)-yf_y(0,0)+x^2+y^2}{\sqrt{x^2+y^2}}
    \]
\end{question}

%------------Ejercicio 4---------------------------------------

\begin{question}
    Analizar la existencia de máximos y mínimos, absolutos o relativos, en todo $\mathbb{R}^2$ de
    \[
        f(x,y) = e^{xy-1}-\frac{1}{2}x^2-\frac{1}{2}y^2.
    \]
\end{question}

\newpage
%------------Solucion 1---------------------------------------

\begin{solution}
    (a) Podemos observar que el límite es indeterminado, aún más, el argumento del coseno tiende a infinito. Para resolver, reescribimos el límite de la siguiente manera
    \[
        \lim_{(x,y)\to(0,1)} \frac{x^2(y-1)}{x^2+3(y-1)^2}\cos(\frac{1}{y-1}).
    \]
    Dado que el coseno es una función acotada,  bastaría con probar que $$ \lim_{(x,y)\to(0,1)} \frac{x^2(y-1)}{x^2+3(y-1)^2}=0.$$
    Para esto, usaremos las siguientes desigualdades,
    \begin{gather*}
        0 \leq \left|\frac{x^2(y-1)}{x^2+3(y-1)^2}\right| \leq \frac{\|(x,y-1)\|^2(y-1)}{x^2+3(y-1)^2}\leq\\[.2cm]
        \leq
        \frac{\|(x,y-1)\|^2(y-1)}{x^2+(y-1)^2} = \frac{\|(x,y-1)\|^2(y-1)}{\|(x,y-1)\|^2} = y-1.
    \end{gather*}
    Como  $$\lim_{(x,y)\to(0,1)} 0 = 0 \;\;\;\;\;\;\mbox{ y } \lim_{(x,y)\to(0,1)} (y-1) = 0, $$
    tenemos, usando el teorema de intercalaci\'on, que
    $$ \lim_{(x,y)\to(0,1)} \left|\frac{x^2(y-1)}{x^2+3(y-1)^2}\right| = 0.$$
    Luego  $$ \lim_{(x,y)\to(0,1)} \frac{x^2(y-1)}{x^2+3(y-1)^2}=0,$$ por lo tanto $$\lim_{(x,y)\to(0,1)} \frac{x^2(y-1)}{x^2+3(y-1)^2}\cos\left(\frac{1}{y-1}\right) = 0.$$


    (b) Tomemos  las curvas,  $\boldsymbol{\alpha}:\mathbb{R}\to\mathbb{R}^2:  \boldsymbol{\alpha}(t)=(t,t^5)$  \; y \;  $\boldsymbol{\beta}:\mathbb{R}\to\mathbb{R}^2: \boldsymbol{\beta}(t)=(t,t^2)$. Notar que $$\lim_{t\to0}\boldsymbol{\alpha}(t)=(0,0)  \; \mbox{ y } \;   \lim_{t\to0}\boldsymbol{\beta}(t)=(0,0).$$
    Llamando  $$f(x,y) = \frac{\sen(x^3)y}{x^2-y+x^5}$$ tenemos que
    \begin{equation}
        \lim_{t\to0}f\circ\boldsymbol{\alpha}(t) = \lim_{t\to0}\frac{\sen(t^3)t^5}{t^2-t^5+t^5} = \lim_{t\to0}\sen(t^3)t^3 = 0 \label{eq:curva1}
    \end{equation}
    \begin{equation}
        \lim_{t\to0}f\circ\boldsymbol{\beta}(t) = \lim_{t\to0}\frac{\sen(t^3)t^2}{t^2-t^2+t^5} = \lim_{t\to0}\frac{\sen(t^3)}{t^3} = 1. \label{eq:curva2}
    \end{equation}
    Como \; $\eqref{eq:curva1} \neq \eqref{eq:curva2}$
    concluimos  que $$ \nexists\lim_{(x,y)\to(0,0)}
        f(x,y).$$
\end{solution}

%------------Solucion 2---------------------------------------

\begin{solution}
    1. Para que la función sea continua en el origen  debe cumplir
    \[
        \lim_{(x,y)\to(0,0)} f(x,y) = f(0,0) =0.
    \]
    Analicemos
    \[
        \lim_{(x,y)\to(0,0)} \frac{x^4}{(x^2-y)^2+x^4}
    \]

    Tomemos  la curva,  $\boldsymbol{\alpha}:\mathbb{R}\to\mathbb{R}^2:  \boldsymbol{\alpha}(t)=(t,t^2),$ notemos que  $$\lim_{t\to0}\boldsymbol{\alpha}(t)=(0,0).$$   Podemos observar que $$ \lim_{t\to0} f\circ\boldsymbol{\alpha}(t) = 1$$  de aqu\'i concluimos que $f$ no es continua en el origen (aunque nada estamos diciendo de la existencia o no del l\'imite).

    2. $f$ no es diferenciable en el origen pues no es continua en dicho punto.

    3. Recordemos la definici\'on de derivada direccional de direcci\'on ${\mathbf{v}}$ evaluada en el origen de un campo escalar $f$.
    \[
        f_{{\mathbf{v}}}(0,0)=\lim_{h\to0}\frac{f((0,0)+h{\mathbf{v}})-f(0,0)}{h},
    \]
    con  ${\mathbf{v}}=(v_1,v_2)$ unitario.

    Entonces calculamos
    \begin{align*}
        f_{{\mathbf{v}}}(0,0) & =\lim_{h\to0}\frac{f(h{\mathbf{v}})}{h}=\lim_{h\to0}\frac{1}{h}\frac{(hv_1)^4}{((hv_1)^2-hv_2)^2+(hv_1)^4} \\[.2cm]
                              & =\lim_{h\to0}\frac{(hv_1)^4}{(hv_1)^4-2(hv_1)^2hv_2+(hv_2)^2+(hv_1)^4}                                     \\[.2cm]
                              & =\lim_{h\to0}\frac{h^4v_1^4}{h^4v_1^4-2h^3v_1^2v_2+h^2v_2^2+h^4v_1^4}                                      \\[.2cm]
                              & =\lim_{h\to0}\frac{h^2h^2v_1^4}{h^2(h^2v_1^4-2hv_1^2v_2+v_2^2+h^2v_1^4)}                                   \\[.2cm]
                              & =\lim_{h\to0}\frac{h^2v_1^4}{h^2v_1^4-2hv_1^2v_2+v_2^2+h^2v_1^4}=\frac{0}{v_2^2}=0,
    \end{align*}
    si $v_2^2 \neq 0 \iff v_1^2 \neq 1 \iff |v_1| \neq 1$.  Veamos el caso  $v_1 = 1$
    $$  \lim_{h\to0} \frac{f(h,0)-f(0,0)}{h} = \lim_{h\to0} \frac{1}{h}\frac{h^4}{(h^2)^2+h^4}=\lim_{h\to0}\frac{1}{h}\frac{1}{2}=\infty.$$ Es decir, no existe $f_x(0,0)$.  El caso  $v_1 = -1$ es an\'alogo.


    O sea, las derividas direccionales existen en todas direcciones, menos en la dirección del eje de abscisas, y son iguales a cero. Es decir,
    \[  \therefore\quad f_{{\mathbf{v}}}(0,0)=0\quad\forall{\mathbf{v}}\in\mathbb{R}^2: |v_1| \neq1,  \]
    \[ \quad  \quad   \nexists \:f_{{\mathbf{v}}}(0,0) \;\mbox{si }{\mathbf{v}}\in\mathbb{R}^2: |v_1| = 1. \]



\end{solution}

%------------Solucion 3---------------------------------------

\begin{solution}
    Para resolver este límite debemos intuir que en el numerador se encuentra la función $g$ menos su plano tangente en el $(0,0)$. Entonces buscamos el gradiente de $g$ en el origen.

    Dado que $f$ es  de  clase \(C^1(\mathbb{R}^2)\) luego $g$  resulta de la misma clase.  Usando la regla de la cadena  y el hecho de que $f(0,0)=0$ obtenemos
    \begin{align*}
        \nabla g(x,y)\Big\rvert_{(0,0)}= & \left( \;2xy+\cos\:(f(x,y))f_x(x,y),\;\; x^2+\cos\:(f(x,y))f_y(x,y)\; \right) \Big\rvert_{(0,0)} \\
        =                                & \left( f_x(0,0),\;\; f_y(0,0) \right).
    \end{align*}
    Como $g(0,0)=0$ podemos reescribir el límite como
    \begin{align*}
         & \lim_{(x,y)\to(0,0)} \left(\frac{g(x,y)-\nabla g(0,0)\cdot(x,y)-g(0,0)}{\sqrt{x^2+y^2}}+\frac{x^2+y^2}{\sqrt{x^2+y^2}}\right).
    \end{align*}
    El primer término tiende a cero dado que $g$ es  de  clase \(C^1(\mathbb{R}^2)\)  entonces es diferenciable en todo $\mathbb{R}^{2}$ y,  en particular, lo es en el origen.  Para el segundo término hacemos un cálculo auxiliar.
    \[
        \lim_{(x,y)\to(0,0)}\frac{x^2+y^2}{\sqrt{x^2+y^2}}=\lim_{(x,y)\to(0,0)}\sqrt{x^2+y^2}=0
    \]
    Estamos en condiciones de usar \'algebra de l\'imites,
    \[
        \therefore \lim_{(x,y)\to(0,0)} \left(\frac{g(x,y)-\nabla g(0,0)\cdot(x,y)-g(0,0)}{\sqrt{x^2+y^2}}+\frac{x^2+y^2}{\sqrt{x^2+y^2}}\right)=0.
    \]
\end{solution}

%------------Solucion 4---------------------------------------

\begin{solution}
    Dado que $f$ es diferenciable en todo $\Re^{2}$    para hallar los puntos cr\'iticos  basta con buscar cuando su  gradiente se anula.
    \[
        \nabla f(x,y)= \left( ye^{xy-1}-x,\; xe^{xy-1}-y \right)
    \]
    Igualando el gradiente a cero, nos queda el siguiente sistema de ecuaciones.
    \[
        \begin{cases}
            ye^{xy-1}=x \\
            xe^{xy-1}=y
        \end{cases}
    \]
    Si $x\neq0 \;\;\land\;\; y\neq0$, entonces
    \begin{equation}
        e^{xy-1}=\frac{x}{y}=\frac{y}{x}. \label{eq:gradEq0}
    \end{equation}
    De la última igualdad hallamos la siguiente relación  $$  x^2=y^2 \iff x=y \;\;\lor\;\; x=-y.$$
    Si  $x=y$ reemplazado en  \eqref{eq:gradEq0} queda
    $$  e^{x^2-1}=1  \iff  x^2-1=0 \iff x=1 \;\;\lor\;\; x=-1.$$
    Es f\'acil ver que el caso  $x=-y$  conlleva a un absurdo.   Por \'ultimo, observar que  el $(0,0)$  también es solución del sistema.  Por lo tanto, los puntos críticos son $(0,0),\;(1,1)$ y el $(-1,-1)$.

    Para clasificarlos,  como $f \in C(\Re^{2})$ utilizamos el criterio de la segunda derivada.
    \[
        \setlength{\abovedisplayskip}{0.5cm}
        \setlength{\belowdisplayskip}{0.5cm}
        \boldsymbol{H}_f(x,y) = \left(\begin{array}{cc}
                y^2e^{xy-1}-1  & e^{xy-1}(xy+1) \\[.4cm]
                e^{xy-1}(xy+1) & x^2e^{xy-1}-1
            \end{array}\right)
    \]
    Ahora evaluamos el determinante de la matriz hessiana para los puntos críticos
    \[
        \boldsymbol{H}_f(0,0) = \left(\begin{array}{cc}
                -1  & 1/e \\
                1/e & -1
            \end{array}\right)
        \implies \text{det} \left( \boldsymbol{H}_f(0,0) \right)  = 1 - \frac{1}{e^2} > 0
    \]
    y como $f_{xx}(0,0)=-1<0 \implies$ $f$ tiene un máximo local en 0.
    \[
        \boldsymbol{H}_f(1,1) = \left(\begin{array}{cc}
                0 & 2 \\
                2 & 0
            \end{array}\right)
        \implies \text{det} \left( \boldsymbol{H}_f(1,1) \right)  = -4 < 0
    \]
    $\implies \;f$ tiene un punto silla en (1,1).
    \[
        \boldsymbol{H}_f(-1,-1) = \left(\begin{array}{cc}
                0 & 2 \\
                2 & 0
            \end{array}\right)
        \implies \text{det} \left( \boldsymbol{H}_f(1,1) \right)  = -4 < 0
    \]
    $\implies \;f$ tiene un punto silla en $(-1,-1)$.

    Por último, para analizar si el máximo en $(0,0)$ es local o absoluto basta calcular, por ejemplo, $f(2,2)$ para ver que es mayor que $f(0,0)$.

    $\therefore\;f$ tiene dos puntos silla, uno en $(1,1)$ y otro en $(-1,-1)$, y un máximo local en $(0,0)$.

\end{solution}



        \newpage
        \section{Fecha Recuperatorio 2cuatri 2022}
            %------------Ejercicio 1---------------------------------------

\begin{question}
    Sea la función

    \[
        f(x,y) =
        \begin{cases}
            \displaystyle \frac{yx^3-xy^3}{x^2+y^2} & \text{si}\; (x,y) \neq (0,0) \\[10pt]
            \qquad 0                                & \text{si}\; (x,y)=(0,0)
        \end{cases}
    \]

    probar que no existe $\delta>0$ tal que $f$ sea clase $C^2$ en $B_{\delta}(0,0)$.
\end{question}

%------------Ejercicio 2---------------------------------------

\begin{question} Calcular
    \[
        \lim_{(x,y) \to (0,0)}
        \frac{e^{2x}e^{3y} - 1 - 2x -3y - 3x^2 - 6yx - \frac{11}{2}y^2}{x^2+y^2}
    \]
\end{question}

%------------Ejercicio 3---------------------------------------

\begin{question}
    Sea $S$ el conjunto de puntos en  $\Rn{3}$  que forman la esfera de centro $(3,4,5)$ tal que el $(0,0,0) \in S$.
    \begin{itemize}
        \item [a.] Hallar el plano tangente a $S$ en $(0,0,0)$.
        \item [b.] Hallar otro plano que sea tangente a $S$ y paralelo al del ítem a.
    \end{itemize}
\end{question}

%------------Ejercicio 4---------------------------------------

\begin{question}
    Hallar los puntos de $A:(x-1)^2 + (y-1)^2 = 4$ que realicen la distancia mínima y máxima al origen.
\end{question}

\newpage
%-----------Solucion 1--------------------------------------------

\begin{solution}
    Por el teorema de Schwartz sabemos que si $f \in\ C^2$ en  $B_{\delta}(0,0)$
    entonces $$  f_{xy} (x,y) = f_{yx} (x,y) \quad \forall (x,y) \in B_{\delta}(0,0).$$ En particular, deber\'ia pasar que $f_{xy} (0,0) = f_{yx}(0,0).$ Veamos que esto \'ultimo no sucede.

    \begin{equation}
        f_x(x,y) = \frac{(3yx^2-y^3)(x^2+y^2) - (yx^3-xy^3)2x}{(x^2+y^2)^2} \label{eq:partXej1}
    \end{equation}

    \begin{equation}
        f_y(x,y) = \frac{(x^3-3xy^2)(x^2+y^2) - (yx^3-xy^3)2y}{(x^2+y^2)^2}  \label{eq:partYej1}
    \end{equation}

    Para $(x,y) = (0,0)$, calculamos las derivadas por definición.
    \[
        f_x(0,0)  = \lim_{h\to0} \frac{f(h,0) - f(0,0)}{h} = 0
    \]
    \[
        f_y(0,0)  =  \lim_{h\to0} \frac{f(0,h) - f(0,0)}{h} = 0
    \]

    Ya tenemos definidas las derivadas parciales de $f$ en todo $\Rn{2}.$
    \[
        f_x(x,y) =
        \begin{cases}
            \eqref{eq:partXej1} & \text{si}\ (x,y) \neq (0,0) \\
            0                   & \text{c.c.}
        \end{cases}
    \]

    \[
        f_y(x,y) =
        \begin{cases}
            \eqref{eq:partYej1} & \text{si}\ (x,y) \neq (0,0) \\
            0                   & \text{c.c.}
        \end{cases}
    \]

    Ahora podemos calcular las derivadas parciales cruzadas de segundo orden por definición.

    \begin{equation}
        f_{xy}(0,0) = \lim_{h\to0} \frac{f_x(0,h) - f_x(0,0)}{h} \label{eq:limDef1}
    \end{equation}

    \begin{equation}
        f_{yx}(0,0) = \lim_{h\to0} \frac{f_y(h,0) - f_y(0,0)}{h}  \label{eq:limDef2}
    \end{equation}

    Por \eqref{eq:partXej1} y \eqref{eq:partYej1}, se puede observar que
    \[
        f_x(0,h) = -h \text{ y } f_y(h,0) = h.
    \]

    Entonces por \eqref{eq:limDef1} Y \eqref{eq:limDef2}
    \[
        f_{xy}(0,0) = -1 \neq f_{yx}(0,0) = 1.
    \]

    $\therefore\;$ no existe $\delta>0$ tal que $f$ sea clase $C^2$ en $B_{\delta}(0,0)$.
\end{solution}

%-----------Solucion 2--------------------------------------------

\begin{solution}
    Para resolverlo debemos intuir que está conformado por el límite de buena aproximación de un polinomio de Taylor de segundo orden centrado en el origen de una función. Buscamos una expresión para dicho polinomio y para la función.

    Llamemos $f(x,y) = e^{2x}e^{3y}$, notemos que $f$ es de clase $C^3(\Rn{2}).$  Entonces, podemos definir su polinomio de Taylor de segundo orden centrado en el origen, $ P_2(x,y)$, y adem\'as vale el teorema de resto de Taylor
    \[
        P_2(x,y) = f(0,0) + \nabla f(0,0)\cdot(x,y) + \frac{1}{2}(x,y)\boldsymbol{H}_f(0,0)(x,y)^\mathrm{T},
    \]
    siendo $\boldsymbol{H}_f(0,0)$ la matriz hessiana de $f$ evaluada en el origen.

    Tenemos que
    \begin{gather*}
        f(0,0) = 1,\\[.2cm]
        \nabla f(x,y) = \left(2e^{2x}e^{3y}, 3e^{2x}e^{3y}\right) \implies \nabla f(0,0) = (2,3),\\[.25cm]
        \boldsymbol{H}_f(x,y) = \left(\begin{array}{cc}
                4e^{2x}e^{3y} & 6e^{2x}e^{3y} \\[10pt]
                6e^{2x}e^{3y} & 9e^{2x}e^{3y}
            \end{array}\right) \implies \boldsymbol{H}_f(0,0) = \left(\begin{array}{cc}
                4 & 6 \\
                6 & 9
            \end{array}\right).
    \end{gather*}
    Entonces queda
    \[
        \begin{aligned}
            P_2(x,y)      & = 1+(2,3)\cdot(x,y)+\frac{1}{2}(x,y)
            \left(\begin{array}{cc}  
                    4 & 6 \\
                    6 & 9
                \end{array}\right)
            \left(\begin{array}{cc}
                          x \\
                          y
                      \end{array}\right)                                                     \\[4pt]
            \iff P_2(x,y) & = 1 + 2x + 3y + 2x^2 + 6xy + \frac{9}{2}y^2.
        \end{aligned}
    \]
    Expresamos el error como
    \[
        R_2(x,y) = f(x,y) - P_2(x,y).
    \]
    Por tanto, podemos reescribir el límite original como
    \[\\[2pt]
        \lim_{(x,y) \to (0,0)} \frac{R_2(x,y) - x^2 - y^2}{\|(x,y)\|^2} =
        \lim_{(x,y) \to (0,0)} \left( \frac{R_2(x,y)}{\|(x,y)\|^2} - \frac{x^2+y^2}{\|(x,y)\|^2} \right) = -1,
    \]
    pues el primer término tiende a 0 por el teorema de Taylor  y el segundo  tiende claramente a $1.$
\end{solution}

%-----------Solucion 3--------------------------------------------

\begin{solution}
    a. Dado que el origen pertenece a $S$ y conocemos su centro, podemos hallar la ecuación de la esfera.
    \[
        S:(x-3)^2 + (y-4)^2 + (z-5)^2 = 50
    \]
    \begin{center}
        \begin{tikzpicture}
            \begin{axis}[
                    view={70}{15},
                    axis equal,
                    width=14cm,
                    height=12cm,
                    axis lines = center,
                    xlabel = {$x$},
                    ylabel = {$y$},
                    zlabel = {$z$},
                    zmin=-1,
                    xmin=-1,
                    ymin=-1,
                    zmax=14,
                    xmax=14,
                    ymax=12,
                    colormap/viridis
                    % ticks=none,
                ]
                \addplot3[
                    surf,
                    opacity=0.5,
                    samples=21,
                    domain=0:360,
                    y domain=0:180,
                    z buffer=sort
                ]
                ({3 + sqrt(50) * sin(y) * cos(x)},
                {4 + sqrt(50) * sin(y) * sin(x)},
                {5 + sqrt(50) * cos(y)});

            \end{axis}
        \end{tikzpicture}
    \end{center}

    Llamando $$f:\mathbb{R}^3 \rightarrow \mathbb{R}:  f(x,y,z) = (x-3)^2 + (y-4)^2 + (z-5)^2 - 50$$
    tenemos que $$S=C(f,0). $$
    Buscamos el plano $\Pi$, tangente a $C(f,0)$ y que pase por $(0,0,0)$. Sabemos que la ecuación de  $\Pi$  es  de la forma
    \[
        \Pi:\nabla f(0,0,0) \cdot (x-0,y-0,z-0) = 0.
    \]
    Es f\'acil ver que $\nabla f(0,0,0) = (-6,-8,-10).$  Luego, una posible ecuaci\'on es
    \[\\[1pt]
        \Pi: 3x + 4y + 5y = 0.
    \]

    b. Dado que el único plano paralelo a otro tangente a un punto en una esfera es el que se encuentra en el polo opuesto de ésta, buscamos el plano tangente $\Pi'$ a ese otro punto $\mathbf{w}$. Llamando $\mathbf{v} = (-3,-4,-5)$ al vector que sale del centro de la esfera y termina en el origen, queda que
    \[
        \mathbf{w} = (3,4,5) - \mathbf{v} = (6,8,10).
    \]
    Como $\Pi$ y $\Pi'$ son paralelos, su vector normal $\textbf{n}$ es el mismo.
    \[
        \begin{aligned}
            \implies & \Pi ': \textbf{n} \cdot ((x,y,z) - \mathbf{w}) = 0 \\
            \iff     & \Pi ': 3(x-6) + 4(y-8) + 5(z-10) = 0               \\
            \iff     & \Pi ':3x + 4y + 5y = 100
        \end{aligned}
    \]

    \begin{center}
        \begin{tikzpicture}
            \begin{axis}[
                    view={60}{7},
                    axis equal,
                    width=14cm,
                    height=14cm,
                    axis lines = center,
                    xlabel = {$x$},
                    ylabel = {$y$},
                    zlabel = {$z$},
                    zmin=-20,
                    xmin=-10,
                    ymin=-10,
                    zmax=30,
                    xmax=40,
                    ymax=30,
                    colormap/viridis
                ]

                % Add the second plane
                \addplot3[
                    surf,
                    fill = blue,
                    opacity=0.6,
                    domain=-10:15,
                    y domain=-10:15,
                ]
                {- (3*x + 4*y)/5}; % Equation of the second plane: 3x + 4y + 5z = 0

                \addplot3[
                    surf,
                    opacity=0.5,
                    samples=21,
                    domain=0:360,
                    y domain=0:180,
                    z buffer=sort
                ]
                ({3 + sqrt(50) * sin(y) * cos(x)},
                {4 + sqrt(50) * sin(y) * sin(x)},
                {5 + sqrt(50) * cos(y)});

                % Add the first plane
                \addplot3[
                    surf,
                    fill = red,
                    opacity=0.6,
                    domain=-10:15,
                    y domain=-10:15,
                ]
                {20-(3*x + 4*y)/5}; % Equation of the first plane: 3x + 4y + 5z = 100

            \end{axis}
        \end{tikzpicture}
    \end{center}
\end{solution}


\vspace{0.3 cm}

%-----------Solucion 4--------------------------------------------

\begin{solution}
    Sean
    \[
        f: \mathbb{R}^2 \rightarrow \mathbb{R}, \; f(x,y) = \text{dist}^2\Big((x,y);(0,0)\Big)  = x^2 + y^2
    \]
    y
    \[
        g: \mathbb{R}^2 \rightarrow \mathbb{R}, \; g(x,y) = (x-1)^2 + (y-1)^2 - 4,
    \]
    luego $A=C(g,0) $.

    Como $f,g \in C^1( \mathbb{R}^1),$  el teorema de los multiplicadores de Lagrange nos dice que si $f \big\rvert _A$ tiene un extremo local en un punto \textbf{x} entonces necesariamente los vectores $\nabla f(\textbf{x})$ y $\nabla g(\textbf{x})$ son paralelos si $\nabla g(\textbf{x}) \neq 0$.   Además como $f$ es continua sobre el conjunto $A$ que es cerrado y acotado, por el teorema de Weierstrass, sabemos que $f$ alcanza m\'aximo y m\'inimo en $A$. Luego planteamos el siguiente sistema de ecuaciones
    \[
        \nabla f(x,y) = \lambda \nabla g(x,y), \; \lambda \in \mathbb{R}, \; (x,y) \in A \\
        \iff \begin{cases}
            2x = 2\lambda(x-1) \\
            2y = 2\lambda(y-1) \\
            (x-1)^2 + (y-1)^2 = 4
        \end{cases}
    \]
    Resolviendo el sistema queda $x=y$  \;  y \[ \begin{cases}
            x=   x_1=1+\sqrt{2} \\
            x=  x_2=1-\sqrt{2}.
        \end{cases}\]
    $\implies (x_1,x_1)$  y $(x_2,x_2)$  son ambos extremos absulutos de la función $f$  en $A$. Como $f(x_1,x_1) > f(x_2,x_2) \implies (x_1,x_1)$ es máximo absoluto y $(x_2,x_2)$ es mínimo absoluto.

    Luego  la distancia mínima del circulo al origen es $\sqrt{f(x_2,x_2)} = 2-\sqrt{2} $ y la distancia máxima $\sqrt{f(x_1,x_1)} = 2+\sqrt{2}$.
\end{solution}




        \newpage
    \chapter{Segundos parciales resueltos}
        \section{Fecha 25 de junio de 2023}
            %------------Ejercicio 1---------------------------------------

\begin{question}
    Sea  $S$  la superficie dada por $x^{2}+ y^{2} = z$ con $z \geq 1$  y $x^{2}+ y^{2} \leq 4$  y sea  $\mathbf{F}:\mathbb{R}^{3}\rightarrow\mathbb{R}^{3}$  un campo de clase $C^{1}$ con tercer componente  nula y $\grad \cdot \mathbf{F}$ constante 3. Calcular el flujo de  $\mathbf{F}$  a trav\'es de $S$ indicando la orientaci\'on elegida.
\end{question}

%------------Ejercicio 2---------------------------------------

\begin{question}
    Sea $\mathbf{F}(x,y,z) = (y^{2}-2xz,\;2xy+z^{3},\;3yz^{2}-x^{2}) $ y sea  $C$ una curva simple parametrizada por $\boldsymbol{\sigma}:[0,1]\rightarrow \mathbb{R}^{3}$
    tal que $\boldsymbol{\sigma}(0)=(0,0,0)$ y $\boldsymbol{\sigma}(1)=(1,2,0)$. Calcular $\int_{C} \mathbf{F}\cdot d\mathbf{s}$ indicando la orientaci\'on elegida.
\end{question}

%------------Ejercicio 3---------------------------------------

\begin{question}
    Calcular la masa del cuerpo limitado por las ecuaciones $y-x=1$ y $x^{2}+ z^{2} = 1$ en el primer octante con funci\'on de densidad $\rho$ constante.
\end{question}

%------------Ejercicio 4---------------------------------------

\begin{question}
    Sean  $k\in\mathbb{R}$,  un campo vectorial $\mathbf{F}:\mathbb{R}^{2}\rightarrow\mathbb{R}^{2}$   de clase $C^{1}$ tal que $\grad\times \mathbf{F}=k$ y $D=\{ (x,y) \in \mathbb{R}^{2} :  |x| \leq 1 \:  ; \:   x \leq y\leq 1  \}.$ Hallar $k$ tal que $\int_{\partial D^{+}}\mathbf{F}\cdot d\mathbf{s} = 8$.
\end{question}

%------------Solucion 1---------------------------------------
\newpage
\begin{solution}
    Es \'util primero entender con qu\'e superficies y regiones se trabajar\'a.
    \begin{center}
        \begin{tikzpicture}
            \begin{axis}[
                    view={60}{20},
                    xlabel=$x$,
                    ylabel=$y$,
                    zlabel=$z$,
                    xmin=-2.5,
                    ymin=-2.5,
                    zmin=0,
                    xmax=2.5,
                    ymax=2.5,
                    zmax=5,
                    samples=50,
                    width=10cm,
                    height=12cm,
                    colormap/viridis,
                ]
                \addplot3 [surf, opacity=0.8, draw=none, restrict z to domain=1:4,
                    data cs=polar, domain=0:360, y domain=0:4] (x, y, y^2);
            \end{axis}
        \end{tikzpicture}
    \end{center}

    La idea es aplicar el teorema de Gauss pero  $S$ no es una superficie cerrada.   Para ello pensemos en una superficie $S'$  cerrada tal  que $S \subseteq S'.$  Definamos   $S'=S\;\cup\;S_1\;\cup\;S_2$,  donde $S_1$ y $S_2$, son las ``tapas'' del paraboloide ``cortado''.  Algebraicamente $S_1=\{(x,y,z)\in \Rn{3} : x^2+y^2\le 1\;\land z=1\}$ y $S_2=\{(x,y,z)\in \Rn{3} : x^2+y^2\le4\;\land z=4\}$.

    Llamando  $\Omega$  al cuerpo encerrado por $S'$ y orientando a $S'$ de manera exterior estamos en condiciones de usar el teorema de Gauss.
    \begin{gather*}
        \oiint _{S'} \mathbf{F}\cdot d\mathbf{A} =
        \iint _{S} \mathbf{F}\cdot d\mathbf{A} +
        \iint _{S_1} \mathbf{F}\cdot d\mathbf{A} +
        \iint _{S_2} \mathbf{F}\cdot d\mathbf{A}
        =\iiint _\Omega \grad \cdot \mathbf{F}\;dV\\[.2cm]
        \iff \iint _{S} \mathbf{F}\cdot d\mathbf{A} =
        \iiint _\Omega \grad \cdot \mathbf{F}\;dV -
        \iint _{S_1} \mathbf{F}\cdot d\mathbf{A} -
        \iint _{S_2} \mathbf{F}\cdot d\mathbf{A}\\[.2cm]
        \iff \iint _{S} \mathbf{F}\cdot d\mathbf{A} =
        I_1-I_2-I_3
    \end{gather*}

    Resolvemos el primer t\'ermino.
    \begin{align*}
        I_1=\iiint _\Omega \grad \cdot \mathbf{F}\;dV = \iiint _\Omega3\:dV = 3\:\text{Vol}(\Omega)
    \end{align*}
    Pasando a  $\Omega$ en coordenadas cil\'indricas
    \[\begin{dcases}
            x=\rho \cos \phi \\
            y=\rho \sen \phi \\
            z=z
        \end{dcases}\] donde, $$0\leq\rho\leq \sqrt{z}\:\land\:0\leq\phi\leq 2\pi \land 1\leq z \leq 4.$$

    Entonces nos queda
    \[
        \text{Vol}(\Omega) =  \int_1^4  \int_0^{2\pi} \int_0^{\sqrt{z}}  \rho\:d\rho\:d\phi \:dz=2\pi  \int_1^4 \Big( \frac{\rho^2}{2}\Big|_0^{\sqrt{z}} \Big) \:dz =  \pi   \int_1^4   z  \:dz= \frac{15 \pi}{2}.
    \] Luego, $I_1 =  \frac{45}{2}\pi. $

    Para clacular $I_2$  debemos parametrizar $S_1$.  Sea  $D_1=\{(u,v)\in\Rn{2}:u^2+v^2\leq1\}$ y  sea  $$\boldsymbol{\Sigma}:D_1\subset\Rn{2}\to\Rn{3}  \mbox{ tal que }   \boldsymbol{\Sigma}(u,v)=(u,v,1).$$
    Entonces podemos reescribir
    \[
        I_2=\iint _{D_1} (\mathbf{F}\circ\boldsymbol{\Sigma})\cdot
        (\boldsymbol{\Sigma}_u\times\boldsymbol{\Sigma}_v)\:d\mathbf{A}.
    \]
    Primero calculamos las derivadas parciales de la parametrizaci\'on.
    \begin{align*}
        \boldsymbol{\Sigma}_u & =(1,0,0) \\
        \boldsymbol{\Sigma}_v & =(0,1,0)
    \end{align*}
    Entonces
    \[
        \boldsymbol{\Sigma}_u\times\boldsymbol{\Sigma}_v=(0,0,1)=\boldsymbol{\eta}.
    \]
    Podemos observar que $\boldsymbol{\eta}$ no preserva la orientaci\'on  para $S_1$ heredada  por la orientaci\'on exterior de $S'$ elegida anteriormente para poder aplicar el teorema de Gauss. En otras palabras,  $\boldsymbol{\eta}$ ``apunta''  hacia el interior de $\Omega$.  Por lo que,
    \[
        I_2=-\iint _{D_1} (\mathbf{F}\circ\boldsymbol{\Sigma})\cdot \boldsymbol{\eta}\:d\mathbf{A}.
    \]
    Calculamos el integrando aparte. Acord\'emonos que la tercer coordenada de $\mathbf{F}$ es nula. Podemos llamar
    \[
        \mathbf{F}\circ\boldsymbol{\Sigma}=(P,Q,0),
    \]
    entonces nos queda que
    \[
        (\mathbf{F}\circ\boldsymbol{\Sigma})\cdot \boldsymbol{\eta}=(P,Q,0)\cdot (0,0,1)=0.
    \]
    Por lo tanto $I_2=0$.

    Procediendo de la misma manera nos daremos cuenta que tambi\'en $I_3=0$.

    $$\therefore\;\iint _{S} \mathbf{F}\cdot d\mathbf{A}=I_1=\frac{45}{2}\pi$$
\end{solution}

%------------Solucion 2---------------------------------------

\begin{solution}
Busquemos, si existe,  una funci\'on potencial para $\mathbf{F},$ es decir, buscamos $f$ tal que $\grad f = F.$ Para ello,  planteamos el sistema de tres ecuaciones diferenciales dado por:
\[\begin{dcases}
        y^2-2xz=f_x \\
        2xy+z^3=f_y \\
        3yz^2-x^2=f_z
    \end{dcases}\]
Entonces
\begin{equation}
    \int f_x(x,y,z)\:dx=y^2x-zx^2+g(y,z)=f(x,y,z), \label{eq:cons1}
\end{equation}
donde $g$ es la constante con respecto a $x$ producto de la integraci\'on. Derivando \eqref{eq:cons1} con respecto a $y$ obtenemos
\begin{gather*}
    f_y(x,y,z)= 2yx+g_y(y,z)\\[.2cm]
    2yx+g_y(y,z)  = 2xy+z^3 \implies g_y(y,z)=z^3\\[.2cm]
    g(y,z)=\int  g_y(y,z) \:dy =  z^3y+h(z).
\end{gather*}
Reemplazando en  la ecuaci\'on \eqref{eq:cons1},
\begin{equation}
y^2x-zx^2+z^3+h(z) =  f(x,y,z).   \label{eq:cons2}
\end{equation}
Derivando \eqref{eq:cons2} con respecto a $z$ obtenemos
\begin{gather*}
    f_z(x,y,z)=-x^2+3yz^2+h_z(z)\\
    -x^2+3yz^2+h_z(z)=3yz^2-x^2   \implies h_z(z) =0 \\[.2cm]
    h(z) = C,\quad\text{con}\;\;C\in\R.
\end{gather*}
Por lo tanto llegamos a una familia de soluciones.
\[
    f(x,y,z)=y^2x-zx^2+z^3+C  \quad\text{con}\;\;C\in\R
\]
Demostramos que $\mathbf{F}$ es un campo conservativo pues $\mathbf{F}=\grad f.$ Luego, podemos calcular
\[
    \int _{\boldsymbol{\sigma}} \mathbf{F}\cdot d\mathbf{s}=
    \int _{\boldsymbol{\sigma}} \grad f\cdot d\mathbf{s} = f(\boldsymbol{\sigma}(1))-f(\boldsymbol{\sigma}(0))=f(1,2,0)-f(0,0,0)=4.
\]
\end{solution}

%------------Solucion 3---------------------------------------

\begin{solution}
    La masa de un cuerpo de volumen $\Omega$ es
    \[
        M=\iiint_\Omega \rho\:dV.
    \]
    En este caso
    \[
        \Omega=\{(x,y,z)\in\Rn{3}:x^2+z^2\leq1\;\land\;y\leq1+x\; \land\;x,y,z\geq0\}.
    \]
    Conviene trabajar en coordenadas cil\'indricas. Sea
    \[\begin{dcases}
            x=r\sen\phi \\
            y=y         \\
            z=r\cos\phi.
        \end{dcases}\]
    Entonces en el nuevo sistema de coordenadas
    \[
        \Omega^*=\{(r,\phi,y)\in\Rn{3}:    0\leq  r\leq1\; \land\; 0 \leq y\leq1+r\sen\phi\;\land\;  0\leq\phi\leq\frac{\pi}{2}\}.
    \]

    Ahora podemos reescribir, dado que $\rho$ es constante,
    \begin{align*}
        \frac{M}{\rho} = & \iiint_{\Omega^*} r\:dy\:d\phi\:dr=\int_0^1 \int_0^{\frac{\pi}{2}} \int_0^{1+r\sen\phi} r\:dy\:d\phi\:dr \\[.2cm]
        =                & \int_0^1 \int_0^{\frac{\pi}{2}}r(1+r\sen\phi)\:d\phi\:dr=\int_0^1 \int_0^{\frac{\pi}{2}} ( r+r^2\sen\phi) \:d\phi\:dr \\[.2cm]
        =                & \int_0^1\int_0^{\frac{\pi}{2}} r\:d\phi\:dr+\int_0^1\int_0^{\frac{\pi}{2}}r^2\sen\phi\:d\phi\:dr                      \\[.2cm]
        =                & \int_0^1r\:dr\int_0^{\frac{\pi}{2}}d\phi+\int_0^1r^2\:dr\int_0^{\frac{\pi}{2}}\sen\phi\:d\phi                         \\[.2cm]
        =                & \frac{\pi}{4}+\frac{1}{3}.
    \end{align*}
    \[
        \therefore\:M=\rho\left(\frac{\pi}{4}+\frac{1}{3}\right)
    \]
\end{solution}

%------------Solucion 4---------------------------------------

\begin{solution}
    La regi\'on $D$ se grafica de la siguiente manera.

    \begin{center}
        \begin{tikzpicture}
            \begin{axis}[
                    axis lines=center,
                    axis equal,
                    xlabel=$x$,
                    ylabel=$y$,
                    xmin=-1,
                    xmax=1,
                    ymin=-1,
                    ymax=1.2,
                    xtick distance=1,
                    ytick distance=1,
                ]
                \addplot[name path=A, draw=none]{x} node[pos=0.55, right]{$y=x$};
                \addplot[name path=B, draw=none]{1};
                \addplot[fill=violet!90, opacity=0.7]fill between[of=A and B, soft clip={domain=-1:1}];
            \end{axis}
        \end{tikzpicture}
    \end{center}


    Podemos observar que $D$ es simplemente conexo y que $\partial D$ es una curva simple cerrada, por lo tanto vale el teorema de Green.
    \[
        \oint_{\partial D^+} \mathbf{F}\cdot d\mathbf{s}=\iint_D \grad\times\mathbf{F}\cdot d\mathbf{A}=k\iint_DdA=8
    \]
    Nos quedar\'ia calcular el  $\textcolor{red}{A(D)}$,  \'area de $D$, que, por ser un tri\'angulo, se puede calcular simplemente.
    \[
        A(D) = \iint_D dA=\frac{4}{2}=2
    \]
    Entonces nos queda
    \[
        2k=8\iff k=4.
    \]
\end{solution}

        \newpage
        \section{Fecha recuperatorio 29 de junio de 2023}
            %------------------Ejercicio 1--------------------------------

\begin{question}
  Sea  $S$  la superficie dada por $x^{2}+ y^{2} = z$ con $z \geq 2$  y $x^{2}+ y^{2} \leq 9.$  Calcular el \'area de $S.$
\end{question}

%------------------Ejercicio 2--------------------------------

\begin{question}
  Sea $C$ la curva simple definida por la intersecci\'on de las superficies $x^{2}+z^{2}=16$,  $y+z=4$ con $y\leq 4$ y sea $g$ una funci\'on escalar $C^{1}(\mathbb{R})$. Calcular la integral sobre $C$ del campo $\mathbf{F}(x,y,z)=(x,\;g(y),\;g(z)).$ Indicar la orientaci\'on elegida.
\end{question}

%------------------Ejercicio 3--------------------------------

\begin{question}
  Sean $\mathbf{F}(x,y,z)=(-xy+yz,\;-y^{2}+xz,\; xz)$, $a \in \mathbb{R}$ positivo  y $S$ la superficie de seis caras que encierra el cuerpo dado por $0\leq x \leq  a$, $0 \leq  y \leq  a $ y  $0 \leq  z \leq a$ orientada con normales exteriores.   Hallar $a$ de manera que el flujo de $\mathbf{F}$ a trav\'es  de $S$ sin su cara superior sea $-24$.
\end{question}

%------------------Ejercicio 4--------------------------------

\begin{question}
  Calcule la masa del cuerpo definido por $z\leq \sqrt{x^{2} + y^{2} }$,  $x^{2} + y^{2} + z^{2} \leq 32$,  en el primer octante, si su densidad es proporcional a la distancia de un  punto al plano $z=0.$
\end{question}

\newpage

%------------------Solucion 1--------------------------------

\begin{solution}
  $S$ es una secci\'on de paraboloide, como se puede observar en el siguiente gr\'afico.

  \begin{center}
    \begin{tikzpicture}
      \begin{axis}[
          view={60}{15},
          xlabel=$x$,
          ylabel=$y$,
          zlabel=$z$,
          xmin=-2.5,
          ymin=-2.5,
          zmin=0,
          xmax=2.5,
          ymax=2.5,
          zmax=10,
          samples=30,
          width=10cm,
          height=14cm,
          colormap/viridis,
        ]
        \addplot3 [surf, opacity=0.8, draw=none, restrict z to domain=2:9,
          data cs=polar, domain=0:360, y domain=0:4] (x, y, y^2);
      \end{axis}
    \end{tikzpicture}
  \end{center}

  Escribimos el conjunto $S$ como
  \[
    S=\{(x,y,z)\in \Re^3: x^2+y^2=z;\;2\leq z\leq 9 \}.
  \]
  Queremos calcular \[ \textcolor{red}{A}(S) = \iint_S dA.\]
  Para ello necesitaremos una parametrizaci\'on de $S$. Sea
  $$D=\{(\rho, \phi) \in\Re^2:    \sqrt{2}\leq \rho \leq 3;\;0\leq  \phi \leq 2\pi \}$$  y  sea  $$\boldsymbol{\Sigma}:D \subset\Re^2\to\Re^3  \mbox{ tal que }   \boldsymbol{\Sigma}(\rho, \phi)=(\rho \cos\phi, \rho \sen\phi , \rho^{2}).$$

  Primero calculamos las derivadas parciales de la parametrizaci\'on.
  \begin{align*}
    \boldsymbol{\Sigma}_{\rho} & =(\cos \phi,\;\sen \phi,\; 2\rho)         \\
    \boldsymbol{\Sigma}_\phi   & =(  -\rho \sen \phi,\;\rho \cos \phi,\;0)
  \end{align*}
  Entonces
  $$
    \boldsymbol{\Sigma}_{\rho} \times\boldsymbol{\Sigma}_\phi =
    (-2\rho^2 \cos\phi  , \;-2\rho^2 \sen \phi, \;\rho),
  $$
  $$\|  \boldsymbol{\Sigma}_{\rho} \times\boldsymbol{\Sigma}_\phi\|
    = \rho\sqrt{4\rho^2+1}.$$
  Quedando
  \begin{equation}
    \iint_S dA
    = \int_0^{2\pi} \int_{\sqrt{2}}^3 \|\boldsymbol{\Sigma}_{\rho}
    \times\boldsymbol{\Sigma}_\phi \| \; d\rho d\phi
    = \int_0^{2\pi} \int_{\sqrt{2}}^3 \rho\sqrt{4\rho^2+1}\;d\rho d\phi,
    \label{eq:integral1}
  \end{equation}
  \begin{gather*}
  u=4\rho^2+1 \rightarrow du = 8\rho\;d\rho, 
  \\[.2cm]
  = \frac{1}{8}\int_0^{2\pi} \int_9^{37} \sqrt{u}\;du d\phi
  = \frac{2\pi}{8}\frac{u^{\frac{3}{2}}}{\frac{3}{2}}\Bigg|_9^37
  = \frac{\pi}{4\frac{3}{2}} (9^{\frac{3}{2}} - 2^{\frac{3}{2}}) 
  \\[.2cm]
  = \frac{\pi}{6} (\sqrt{9^3} - \sqrt{8}).
  \end{gather*}
\end{solution}

%------------------Solucion 2--------------------------------

\begin{solution}

  \textcolor{red}{(insertar imagen)}

  Notemos que $C$ es una curva no cerrada. Para cerrarla, llamemos $C=C_1$, 
  $C_2=\{(x,y,z)\in\Re^3:z=0\;\land\;y=4\;\land\;|x|\leq4\}$ y $C_0=C_1+C_2$. Y sea $\text{int}(C_0)=D$. Ahora $C_0$ es una curva simple cerrada. Entonces podemos aplicar el teorema de Stokes, eligiendo la orientaci\'on de $C_0$ tal que se cumpla la regla de la mano derecha.
  \[
      \oint_{C_0}\mathbf{F}\cdot d\mathbf{s}= \int_{C_1}\mathbf{F}\cdot d\mathbf{s} + \int_{C_2}\mathbf{F}\cdot d\mathbf{s} = \iint_D \nabla\times\mathbf{F}\cdot d\mathbf{A},
  \]
  ya que $\mathbf{F}$ es $C^1(\Re)$, pues sus componentes son $C^1(\Re)$, y $D\in\text{dom}(\mathbf{F})$.
  Notemos que $\nabla\times\mathbf{F}\equiv0\implies$
  \[
     \int_{C_1}\mathbf{F}\cdot d\mathbf{s} =- \int_{C_2}\mathbf{F}\cdot d\mathbf{s}.
  \]
  Ahora busquemos una trayectoria de $C_2$ que respete su orientaci\'on. Sea $\boldsymbol{\sigma}:[-4,4]\rightarrow\Re^3$
  tal que $\boldsymbol{\sigma}(t)=(t,4,0)$. Entonces $\boldsymbol{\sigma}'(t)=(1,0,0)$. Luego la integral de curva de $C_2$ es
  \begin{gather*}
    \int_{C_2}\mathbf{F}\cdot d\mathbf{s}
    =\int_{-4}^4 (\mathbf{F}\circ\boldsymbol{\sigma})\cdot\boldsymbol{\sigma}'\:dt
    =\int_{-4}^4 t\:dt = 0.
  \end{gather*}
  $\therefore$ la integral sobre la curva original $C=C_1$ es tambi\'en 0.
\end{solution}

%------------------Solucion 3--------------------------------

\begin{solution}
  Llamemos $\Omega$ al volumen que encierra $S$ y sea $S'$ la superficie de la cara superior de $S$, por lo que $S'$ tiene orientaci\'on igual que la tapa de $S$. En los siguientes gr\'aficos se puede ver la representaci\'on de $S$ y $S'$ respectivamente.

  \begin{tikzpicture}
    \begin{axis}[
        axis equal,
        axis lines = center,
        height = 11cm, width = 11cm,
        view={120}{20},
        xlabel={$x$},
        ylabel={$y$},
        zlabel={$z$},
        zmin = -1, zmax = 3,
        xmin = -1, xmax = 3,
        ymin = -1, ymax = 3,
        xtick = {0, 2},
        xticklabels = {$0$, $a$},
        ytick = {0, 2},
        yticklabels = {$0$, $a$},
        ztick = {0, 2},
        zticklabels = {$0$, $a$},
        xlabel style={at={(axis cs:3,0,0)}, anchor=north west},
        xticklabel style={anchor=north west},
        ylabel style={at={(axis cs:0,3.1,0.05)}},
        yticklabel style={at={(axis cs:0,2,0)}},
      ]

      \addplot3[fill=magenta!80, opacity=0.8, draw=none] coordinates {(0,0,0) (0,2,0) (2,2,0) (2,0,0) (0,0,0)};
      \addplot3[fill=magenta!70, opacity=0.8, draw=none] coordinates {(0,0,0) (0,0,2) (2,0,2) (2,0,0) (0,0,0)};
      \addplot3[fill=magenta!60, opacity=0.8, draw=none] coordinates {(0,0,0) (0,2,0) (0,2,2) (0,0,2) (0,0,0)};
      \addplot3[fill=magenta!65, opacity=0.8, draw=none] coordinates {(2,0,0) (2,2,0) (2,2,2) (2,0,2) (2,0,0)};
      \addplot3[fill=magenta!50, opacity=0.8, draw=none] coordinates {(0,2,0) (0,2,2) (2,2,2) (2,2,0) (0,2,0)};
      \addplot3[fill=cyan!50, opacity=0.8, draw=none] coordinates {(0,0,2) (0,2,2) (2,2,2) (2,0,2) (0,0,2)};
    \end{axis}
  \end{tikzpicture}
  \hfill
  \begin{tikzpicture}
    \begin{axis}[
        axis equal,
        axis lines = center,
        height = 11cm, width = 11cm,
        view={120}{20},
        xlabel={$x$},
        ylabel={$y$},
        zlabel={$z$},
        zmin = -1, zmax = 3,
        xmin = -1, xmax = 3,
        ymin = -1, ymax = 3,
        xtick = {0, 2},
        xticklabels = {$0$, $a$},
        ytick = {0, 2},
        yticklabels = {$0$, $a$},
        ztick = {0, 2},
        zticklabels = {$0$, $a$},
        xlabel style={at={(axis cs:3,0,0)}, anchor=north west},
        xticklabel style={anchor=north west},
        ylabel style={at={(axis cs:0,3.1,0.05)}},
        yticklabel style={at={(axis cs:0,2,0)}},
      ]
      \addplot3[fill=cyan!50, opacity=0.8, draw=none] coordinates {(0,0,2) (0,2,2) (2,2,2) (2,0,2) (0,0,2)};
    \end{axis}
  \end{tikzpicture}


  Por el teorema de la divergencia tenemos que
  \[
    \iint_S \mathbf{F}\cdot d\mathbf{A}=\iiint_\Omega \nabla \cdot \mathbf{F}\;dV.
  \]
  Entonces la integral de flujo que buscamos es
  \[
    \iint_S \mathbf{F}\cdot d\mathbf{A} - \iint_{S'} \mathbf{F}\cdot d\mathbf{A}=
    \iiint_\Omega \nabla \cdot \mathbf{F}\;dV - \iint_{S'} \mathbf{F}\cdot d\mathbf{A}=-24.
  \]
  Primero calculamos el t\'ermino con la divergencia de $\mathbf{F}$.
  \[
    \nabla\cdot\mathbf{F}=-y-2y+x=x-3y
  \]
  \begin{gather*}
    \implies\iiint_\Omega\nabla\cdot\mathbf{F}\;dV=\int_0^a \int_0^a \int_0^a x-3y\;dxdydz\\[.2cm]
    \iff a\int_0^a \left(\frac{x^2}{2}-3yx\right)\Bigg|_0^a\;dy=a\int_0^a \frac{a^2}{2}-3ya\;dy\\[.2cm]
    \iff a\left( \frac{a^2}{2}y-3a\frac{y^2}{2} \right)=\frac{a^4}{2}-3\frac{a^4}{2}=-a^4
  \end{gather*}
  Segundo, el flujo sobre la tapa. Sea $\boldsymbol{\Sigma} : D\subset\Re^2\to S'\subset\Re^3$, una paremetrizaci\'on de $S'$ tal que $\boldsymbol{\Sigma}(u,v)=(u,v,a)$. Donde $D = \{(u,v)\in\Re^2:0\leq u\leq a;\;0\leq v \leq a\}$. Entonces
  \[
    \iint_{S'} \mathbf{F}\cdot d\mathbf{A}=\iint_D \mathbf{F}\circ\boldsymbol{\Sigma} \cdot (\boldsymbol{\Sigma}_u\times\boldsymbol{\Sigma}_v)\;dA.
  \]
  $\boldsymbol{\Sigma}_u=(1,0,0)$ y $\boldsymbol{\Sigma}_v=(0,1,0)$ $\implies \boldsymbol{\Sigma}_u\times\boldsymbol{\Sigma}_v=(0,0,1)=\boldsymbol{\eta}$. Por lo que $\boldsymbol{\Sigma}$ respeta la orientaci\'on de $S'$. Y $\mathbf{F}\circ\boldsymbol{\Sigma}=(-uv+va,\;-v^2+ua,\;ua) \implies \mathbf{F}\circ\boldsymbol{\Sigma}\cdot\boldsymbol{\eta}=ua$. Por lo tanto
  \[
    \iint_{S'} \mathbf{F}\cdot d\mathbf{A}=\int_0^a\int_0^a au\;dudv=a^2\frac{a^2}{2}=\frac{a^4}{2}.
  \]
  \[
    \therefore -a^4-\frac{a^4}{2}=-24\iff \frac{3}{2}a^4=24\iff a^4=16 \iff a=2.
  \]

\end{solution}

%------------------Solucion 4--------------------------------

\begin{solution}
  La distancia $d$ de un punto $(x_0,y_0,z_0)$ en el primer octante a el plano $z=0$ est\'a dada por $$d\Big(  (x_0,y_0,z_0); (x_0,y_0, 0) \Big)= \sqrt{(x_0-x_0)^2+(y_0-y_0)^2+z_0^2}=z_0.$$ Luego  la funci\'on de densidad del s\'olido es $\delta(z)=kz$,  con $k\in\Re$. Adem\'as  sea $$\Omega=\{(x,y,z)\in\Re^3:z\leq\sqrt{x^2+y^2}\:\land\:x^2+y^2+z^2\leq32\:\land\: x,y,z\geq0\}.$$

  \textcolor{red}{ (insertar imagen)}

  Trabajaremos  en coordenadas esf\'ericas.  En este sentido,  recordemos lo siguiente.
  \[
    \begin{dcases}
      x=\rho\sen\phi\cos\theta \\
      y=\rho\sen\phi\sen\theta \\
      z=\rho\cos\phi
    \end{dcases}
  \]
  Por las condiciones de $\Omega$ tenemos que
  $$z\leq\sqrt{x^2+y^2}\iff\rho\cos\phi\leq\sqrt{\rho^2\sen^2\phi}=\rho |\sen\phi|.$$
  Adem\'as, como estamos en el primer octante,  nos queda que $\frac{\pi}{4} \leq\phi\leq\frac{\pi}{2}$ y  $0\leq\theta\leq\frac{\pi}{2}.$

  Por otro lado,  $$ \rho^2\leq32  \; \land \;  \rho \geq 0 \iff     0\leq  \rho\leq\sqrt{32}= 4 \sqrt{2}.$$
  Escribimos nuestro nuevo volumen $\Omega^*$  y usamos el teorema de cambio de variables.
  \[
    \Omega^*=\{(\rho,\phi,\theta)\in\Re^3:0\leq\rho\leq4\sqrt{2}\:\land\:0\leq\phi\leq\frac{\pi}{4}\:\land\:0\leq\theta\leq\frac{\pi}{2}\}
  \]


  \begin{align*}
    M & =\iiint_{\Omega^*}k\delta(z)\rho^2\sen\phi\:dV=\iiint_{\Omega^*}k\rho\cos\phi\rho^2\sen\phi\:d\rho d\phi d\theta \\
      & =k\int_0^{\frac{\pi}{2}} \int_0^{\frac{\pi}{4}} \int_0^{4\sqrt{2}}\rho^3\cos\phi\sen\phi \:d\rho d\phi d\theta   \\
      & =k2\pi\int_0^{\frac{\pi}{4}}\cos\phi\sen\phi\:d\phi\int_0^{4\sqrt{2}}\rho^3\:d\rho                               \\
      & =k2\pi\frac{1}{4}256=128k\pi
  \end{align*}

\end{solution}

        \newpage
        \section{Fecha recuperatorio extra}
            %------------------Ejercicio 1--------------------------------

\begin{question}
    Sean  $V=\{ (x,y,z) \in \mathbb{R}^{3}:  x\leq 0, \;  y
        \leq 0, \;  0 \leq  z \leq 1-x^{2}-y^{2}  \}$ y
    $S=\{ (x,y,z) \in \mathbb{R}^{3}:  x\geq 0, \; y \geq 0,
        z\geq 0,\;  z = 1-x^{2}-y^{2}  \}$.
    Calcluar el volumen de $V$ y el \'area de $S$.
\end{question}

%------------------Ejercicio 2--------------------------------

\begin{question}
    Sean  $S$ una superficie  esf\'erica de radio $R$ con
    centro en el primer octante  y  $\mathbf{F}$ un campo de
    clase $C^{1}$  con $\nabla\cdot\mathbf{F}(x,y,z) = x+y+z$.
    Probar que  el flujo saliente de $\mathbf{F}$ a trav\'es de
    $S$  es no negativo.
\end{question}

%------------------Ejercicio 3--------------------------------

\begin{question}
    Sean  $C$ el borde del tri\'angulo de v\'ertices $(-1,0)$,
    $(1,0)$ y $(0,1)$ y el campo $\mathbf{F}(x,y) =
        \Big(e^{\cos(x)+x^{2}}+xy^{2}+2y, \ln(1+e^{y^{2}}) +yx^{2}+3x \Big).$
    Calcular,  indicando la orientaci\'on elegida:
    $$ \int_{C} \mathbf{F}\cdot ds$$
\end{question}

%------------------Ejercicio 4--------------------------------

\begin{question}
    Probar que el campo $\mathbf{F}(x,y) = \Big(y e^{xy}+y\cos(xy)
        ,xe^{xy}+\cos(xy) x \Big)$ es conservativo.  Dada $C$ la curva
    parametrizada por $\boldsymbol{\alpha}(t) = (t^{2},t)$ con $t
        \in [0,1]$ calcular el trabajo realizado por $\mathbf{F}$ sobre $C.$
\end{question}

\newpage

%------------------Solucion 1--------------------------------

\begin{solution}
    Nos piden calcular $\iiint_V dV$ y $\iint_S dA$. Para la
    integral de volumen conviene trabajar en coordenadas
    cil\'indricas y para la de superficie habr\'a que parametrizar.

    \begin{center}
        \begin{tikzpicture}
          \begin{axis}[
                axis lines=center,
                axis equal,
                view={110}{20},
                xlabel=$x$,
                ylabel=$y$,
                zlabel=$z$,
                xtick={1},
                ytick={-1,1},
                ztick={1,1.5},
                xmin=-1,
                ymin=-0.5,
                zmin=0,
                xmax=1.7,
                ymax=1.1,
                zmax=1,
                samples=30,
                width=14cm,
                height=12cm,
                colormap/viridis,
                xlabel style={at={(0.31,0.2)}},
            ]
            \addplot3 [surf, opacity=0.8, draw=none, restrict z to domain=0:1,
            data cs=polar, domain=0:90, y domain=0:1] (x, y, 1-y^2);
          \end{axis}
        \end{tikzpicture}
      \end{center}

    Aplicando la transformaci\'on $V \rightarrow V^*$ queda
    \begin{align*}
        V^*      & =\{(\rho, \phi, z)\in\Re^3: \rho\cos\phi \leq 0,\;
        \rho\sen\phi \leq 0,\; 0 \leq z \leq 1-\rho^2\}               \\
        \iff V^* & =\{(\rho, \phi, z)\in\Re^3: \pi \leq \phi \leq
        \frac{3}{2}\pi,\; 0 \leq z \leq 1,\; 0 \leq \rho \leq
        \sqrt{1-z} \}.
    \end{align*}
    Y por el teorema de cambio de variable tenemos que
    \begin{align*}
        \iiint_V dV & = \iiint_{V^*} \rho\:dV = \int_\pi^{\frac{3}{2}\pi}
        \int_0^1\int_0^{\sqrt{1-z}}\rho\:d\rho dz d\phi                   \\
                    & =\frac{\pi}{2}\int_0^1\frac{1}{2}(1-z)\:dz =
        \frac{\pi}{8}.
    \end{align*}

    Para parametrizar $S$, llamamos
    \[
        D = \{(\rho, \phi)\in\Re^2 : \pi\leq\phi\leq\frac{3}{2}\pi,\;
        0\leq\rho\leq 1\}.
    \]
    y
    \[
        \boldsymbol{\Sigma}:D\subset\Re^2\rightarrow\Re^3\text{ tal que }
        \boldsymbol{\Sigma}(\rho,\phi)=(\rho\cos\phi,\;\rho\sen\phi,\;\rho^2).
    \]
    Calculamos las derivadas paraciales de $\boldsymbol{\Sigma}$.
    \begin{align*}
        \boldsymbol{\Sigma}_{\rho}&=(\cos \phi,\;\sen \phi,\; 2\rho)\\
        \boldsymbol{\Sigma}_\phi&=(  -\rho \sen \phi,\;\rho \cos \phi,\;0)
        \end{align*}
    Luego
    $$
        \boldsymbol{\Sigma}_{\rho} \times\boldsymbol{\Sigma}_\phi =
        (-2\rho^2 \cos\phi  , \;-2\rho^2 \sen \phi, \;\rho),
    $$ 
    $$\|\boldsymbol{\Sigma}_{\rho} \times\boldsymbol{\Sigma}_\phi\|
        = \rho\sqrt{4\rho^2+1}.
    $$ 
    Entonces escribimos, por definici\'on, el \'area de $S$ como
    \[
        \iint_S dA = \iint_D \| \boldsymbol{\Sigma}_{\rho}
        \times\boldsymbol{\Sigma}_\phi\|\:d\rho d\phi = \int_\pi^{\frac{3}{2}\pi}\int_0^1\rho\sqrt{4\rho^2+1}\:d\rho d\phi.
    \]
    Nos queda la misma integral que \eqref{eq:integral1}, s\'olo que con distintos l\'imites de integraci\'on. Por lo que, procediendo de la misma manera, sustituyendo $u = 4\rho^2+1$, obtenemos
    \begin{gather*}
        \iint_S dA =
        \frac{1}{8}\int_\pi^{\frac{3}{2}\pi}\int_1^5\sqrt{u}\:du =
        \frac{\pi}{16}\frac{u^{\frac{3}{2}}}{\frac{3}{2}}\Bigg\lvert_1^5 =
        \frac{\pi}{24}(\sqrt{125}-1).
    \end{gather*}
\end{solution}

%------------------Solucion 2--------------------------------

\begin{solution}
    Tenemos que $S=\{(x,y,z)\in\Re^3:
        (x-x_0)^2+(y-y_0)^2+(z-z_0)^2=R^2\}$, con $R>0$ y
    $(x_0,y_0,z_0)$ en el primer octante.
    Por el teorema de la divergencia
    \begin{equation}
        \iint_S \mathbf{F}\cdot\:d\mathbf{A} = \iiint_\Omega
        \nabla\cdot\mathbf{F}\:dV
        = \iiint_\Omega (x+y+z)\:dxdydz, \label{eq: divEj2}
    \end{equation}
    tomando la orientaci\'on de $S$ como exterior.

    Aplicando una tranformaci\'on a coordenadas esf\'ericas, tal que
    \begin{align*}
        x & =\rho\sen\phi\cos\theta+x_0 \\
        y & =\rho\sen\phi\sen\theta+y_0 \\
        z & =\rho\cos\phi+z_0,
    \end{align*}
    nos queda el conjunto $S^*$
    \[
        S^*=\{(\rho,\phi,\theta)\in\Re^3:0\leq\rho\leq R,\;
        0\leq\theta\leq2\pi,\;0\leq\phi\leq\pi\}.
    \]
    De la ecuaci\'on \eqref{eq: divEj2} y usando el teorema
    de cambio de variable,
    \begin{gather*}
        \iiint_\Omega (x+y+z) \:dV = \iiint_{\Omega^*}
        (\rho\sen\phi\cos\theta + \rho\sen\phi\sen\theta
        + \rho\cos\phi + x_0 + y_0 + z_0)
        \:\rho^2\sen\phi\:d\rho d\theta d\phi=\\
        = \int_0^\pi\int_0^{2\pi}\int_0^R (\rho\sen\phi\cos\theta
        + \rho\sen\phi\sen\theta + \rho\cos\phi
        + x_0 + y_0 + z_0)\rho^2\sen\phi\:d\rho d\theta d\phi.
    \end{gather*}
    Al distribuir el $\rho^2\sen\phi$ entre los primeros dos
    t\'erminos, queda para estos \(\rho^3\sen\phi^2\cos\theta\) y \\
    \(\rho^3\sen\phi^2\sen\theta\); que al integrarlos entre 0
    y $2\pi$, con respecto a $\theta$, se anulan. Entonces queda,
    distribuyendo la suma del integrando,
    \[
        \int_0^\pi\int_0^{2\pi}\int_0^R \rho^3\cos\phi\sen\phi\:
        d\rho d\theta d\phi + \int_0^\pi\int_0^{2\pi}\int_0^R
        (x_0 + y_0 + z_0)\rho^2\sen\phi\:d\rho d\theta d\phi.
    \]
    A su vez, la primer integral es 0 porque al integrar una
    funci\'on peri\'odica impar, $\cos\phi\sen\phi$, en un
    semi per\'iodo \'esta se anula. Por \'ultimo queda
    \[
        (x_0 + y_0 + z_0)\int_0^\pi\int_0^{2\pi}\int_0^R
        \rho^2\sen\phi\:d\rho d\theta d\phi = (x_0 + y_0 + z_0)
        \frac{4}{3}\pi R^3 > 0,
    \]
    pues $(x_0, y_0, z_0)$ pertenece al primer octante.
    Por lo que queda demostrado que el flujo es positivo.

    \textcolor{red}{por qu\'e ser\'ia -no negativo-? por ej. 
    cuando R = 0 o cuando el centro de la esfera es el origen es = 0}
\end{solution}

%------------------Solucion 3--------------------------------

\begin{solution}
    Nos piden integrar sobre la siguiente curva $C$.

    \begin{center}
        \begin{tikzpicture}
            \begin{axis}[
                    axis lines=center,
                    axis equal,
                    xlabel=$x$, ylabel=$y$,
                    xmin=-1.5, xmax=1.5,
                    ymin=0, ymax=1,
                    xtick distance=1, ytick distance=1,
                ]
                \draw[cyan, thick] (-1,0) -- (1,0) -- (0,1) -- cycle;
            \end{axis}
        \end{tikzpicture}
    \end{center}

    Tomamos la orientaci\'on de $C$ como antihoraria.
    % depende de como este definido el interior de una curva
    Dado que el tri\'angulo $T$  es simplemente conexo,   $T
        \in \text{dom}(\mathbf{F})$ y $\mathbf{F}\in C^1$, entonces
    por el teorema de Green
    \[
        \oint_C \mathbf{F}\cdot d\mathbf{s} = \iint_T
        \nabla\times\mathbf{F}\:dA.
    \]
    Como $\nabla\times\mathbf{F} = 1$, queda que \[\oint_C
        \mathbf{F}\cdot d\mathbf{s} = \iint_T dA = \textcolor{red}{Area}(T)
        = \frac{2\cdot1}{2} = 1.\]
\end{solution}

%------------------Solucion 4--------------------------------

\begin{solution}
    Ya que nos piden demostrar que $\mathbf{F}$ es conservativo
    y calcular una integral de l\'inea sobre una curva no
    cerrada nos conviene buscar, si es que existe, la funci\'on
    potencial de $\mathbf{F}$. Nos encontramos con la ecuaci\'on
    $\nabla f = \mathbf{F}$, la cual describe el siguiente
    sistema de ecuaciones diferenciales.
    \[
        \begin{dcases}
            f_x(x,y) = ye^{xy} + y \cos{(xy)} \\[.2cm]
            f_y(x,y) = xe^{xy} + x \cos{(xy)}
        \end{dcases}
    \]
    Mirando cada t\'ermino detenidamente se puede llegar a
    la conclusi\'on de que $$f(x,y) = e^{xy} + \sen{(xy)} + C,
        \mbox{ con } C\in\Re$$ es soluci\'on del sistema. Lo que
    significa que $\mathbf{F}$ es un campo conservativo.

    Luego para calcular $\int_C \mathbf{F}\cdot d\mathbf{s}$
    utilizamos el siguiente teorema.
    \[
        \int_C \nabla f\cdot d\mathbf{s} =
        f(\boldsymbol{\alpha}(1)) - f(\boldsymbol{\alpha}(0)) =
        f((1,1)) - f((0,0)) = e + \sen1 - 1.
    \]
\end{solution}



    
\end{document}