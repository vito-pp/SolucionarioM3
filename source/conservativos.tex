\begin{definition}
    \textcolor{red}{Definicion conjunto conexo...}
\end{definition}

\begin{theorem} \label{thm:t1}
    Sean $A\subseteq\Re^n$ abierto y conexo y $f:A\subseteq\Re^n\to\Re$ de clase $C^1$. Sea $\boldsymbol{\sigma}:[a,b]\to A\subseteq\Re^n$ una trayectoria $C^1$ a trozos. Entonces
    \[
       \int_{\boldsymbol{\sigma}}\nabla f\cdot d\mathbf{s}=f(\boldsymbol{\sigma}(b))-f(\boldsymbol{\sigma}(a)).
    \]    
\end{theorem}

\begin{corollary}
    En las condiciones del \autoref{thm:t1}, $\int_{\boldsymbol{\sigma}}\nabla f\cdot d\mathbf{s}$ no depende de $\boldsymbol{\sigma}$, s\'olo depende de los puntos inicla y final de la trayectoria. Si $\boldsymbol{\sigma}_1$, $\boldsymbol{\sigma}_2$ son trayectorias $C^1$ a trozos tales que $\boldsymbol{\sigma}_1:[a,b]\to A\subseteq\Re^n$ y $\boldsymbol{\sigma}_2:[a,b]\to A\subseteq\Re^n$, entonces
    $$\int_{\boldsymbol{\sigma}_1}\nabla f\cdot d\mathbf{s}=\int_{\boldsymbol{\sigma}_2}\nabla f\cdot d\mathbf{s}.$$
\end{corollary}

\begin{corollary}
    En las condiciones del \autoref{thm:t1}, si $\mathbf{F}:A\subseteq\Re^n\to\Re^n$ es un campo vectorial para el cual $\exists f:A\subseteq\Re^n\to\Re$ clase $C^1$ tal que 
    $$\nabla f(\mathbf{x})=\mathbf{F}(\mathbf{x})\quad\forall\:\mathbf{x}\in A,$$
    entonces para todo par de puntos $\mathbf{p}_0,\:\mathbf{p}_1\in A$
    $$\int_{C_1}\mathbf{F}\cdot d\mathbf{s}=\int_{C_2}\mathbf{F}\cdot d\mathbf{s},$$
    para cualquier par de curvas $C_1,\;C_2\subset A$ que vayan de $\mathbf{p}_0$ a $\mathbf{p}_1$ 
\end{corollary}

\begin{corollary}
    En las condiciones del \autoref{thm:t1}, si $\mathbf{F}=\nabla f$ Entonces
    $$\oint_C\mathbf{F}\cdot d\mathbf{s}=0,$$
    para toda curva $C$ simple cerrada contenida en $A$.
\end{corollary}

\begin{theorem}
    Sean $A\subseteq\Re^n$ abierto y conexo, $\mathbf{F}:A\subseteq\Re^n\to\Re^n$ continuo tal que $\int_C \mathbf{F}\cdot d\mathbf{s}$ es independiente del camino en $A$. Entonces $f:A\subseteq\Re^n\to\Re$ tal que
    $$f(\mathbf{x})=\int_{\mathbf{x}_0}^{\mathbf{x}}\mathbf{F}\cdot d\mathbf{s},$$
    con $\mathbf{x}_0\in A$, cumple que: 
    \begin{enumerate}
        \item $f$ es $C^1$.
        \item $\nabla f(\mathbf{x})=\mathbf{F}(\mathbf{x})\quad\forall\:\mathbf{x}\in A.$
    \end{enumerate}
\end{theorem}

\begin{theorem}
    Sean $A\subseteq\Re^n$ abierto y conexo, $\mathbf{F}:A\subseteq\Re^n\to\Re$ continuo. Las siguientes condiciones sobre $\mathbf{F}$ son equivalentes:
    \begin{enumerate}
        \item $\exists f:A\subseteq\Re^n\to\Re$ de clase $C^1$ tal que $\mathbf{F}=\nabla f.$ 
        \item La integral de $\mathbf{F}$ es independiente del camino en $A$.
        \item $\oint_C\mathbf{F}\cdot d\mathbf{s}=0$ para toda curva cerrada simple $C$ en $A$.
    \end{enumerate}
\end{theorem}