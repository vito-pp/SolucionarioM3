\begin{definition}
    Sea un conjunto $A\subseteq\Re^n$. $A$ se llama \textbf{conexo} si dados dos puntos cualquiera de $A$ se los puede unir por una curva continua que este incluida en $A$.

    \textcolor{red}{(no la encuentro en el Tromba. La copie como en mis apuntes)}\final
\end{definition}

\begin{theorem} \label{thm:t1}
    Sean $A\subseteq\Re^n$ abierto y conexo y $f:A\subseteq\Re^n\to\Re$ de clase $C^1$. Sea $\boldsymbol{\sigma}:[a,b]\to A\subseteq\Re^n$ una trayectoria $C^1$ a trozos. Entonces
    \[
       \int_{\boldsymbol{\sigma}}\nabla f\cdot d\mathbf{s}=f(\boldsymbol{\sigma}(b))-f(\boldsymbol{\sigma}(a)).\finalmath
    \]    
\end{theorem}

\begin{corollary}
    Sean $A\subseteq\Re^n$ abierto y conexo y $f:A\subseteq\Re^n\to\Re$ de clase $C^1$, entonces $\int_{\boldsymbol{\sigma}}\nabla f\cdot d\mathbf{s}$  s\'olo depende de los puntos inicial y final de la trayectoria. Por lo tanto, sean las trayectorias $\boldsymbol{\sigma}_1:[a_1,b_1]\to A_1\subseteq\Re^n$ y $\boldsymbol{\sigma}_2:[a_2,b_2]\to A_2\subseteq\Re^n$, tales que $\boldsymbol{\sigma}_1(a_1)=\boldsymbol{\sigma}_2(a_2)\;\land\;\boldsymbol{\sigma}_1(b_1)=\boldsymbol{\sigma}_2(b_2)$, entonces
    \[
    \int_{\boldsymbol{\sigma}_1}\nabla f\cdot d\mathbf{s}=\int_{\boldsymbol{\sigma}_2}\nabla f\cdot d\mathbf{s}.\finalmath
    \]
\end{corollary}

\begin{corollary}
    Sean $A\subseteq\Re^n$ abierto y conexo y $f:A\subseteq\Re^n\to\Re$ de clase $C^1$. Si $\mathbf{F}:A\subseteq\Re^n\to\Re^n$ es un campo vectorial para el cual $\exists f:A\subseteq\Re^n\to\Re$ clase $C^1$ tal que 
    $$\nabla f(\mathbf{x})=\mathbf{F}(\mathbf{x})\quad\forall\:\mathbf{x}\in A,$$
    entonces para todo par de curvas $C_1$ y $C_2$ en $A$, con los mismos extremos y misma orientaci\'on,
    \[
    \int_{C_1}\mathbf{F}\cdot d\mathbf{s}=\int_{C_2}\mathbf{F}\cdot d\mathbf{s}. \finalmath
    \]
\end{corollary}

Estos \'ultimos dos corolarios nos demuestran que la integral de curva 

\begin{corollary}
    Sean $A\subseteq\Re^n$ abierto y conexo y $f:A\subseteq\Re^n\to\Re$ de clase $C^1$. Si $\mathbf{F}=\nabla f$, entonces
    $$\oint_C\mathbf{F}\cdot d\mathbf{s}=0,$$
    para toda curva $C$ simple cerrada contenida en $A$.\final
\end{corollary}

\begin{theorem}
    Sean $A\subseteq\Re^n$ abierto y conexo, $\mathbf{F}:A\subseteq\Re^n\to\Re^n$ continuo tal que $\int_C \mathbf{F}\cdot d\mathbf{s}$ es independiente del camino en $A$. Entonces $f:A\subseteq\Re^n\to\Re$ tal que
    $$f(\mathbf{x})=\int_{\mathbf{x}_0}^{\mathbf{x}}\mathbf{F}\cdot d\mathbf{s},$$
    con $\mathbf{x}_0\in A$, cumple que: 
    \begin{enumerate}
        \item $f$ es $C^1$.
        \item $\nabla f(\mathbf{x})=\mathbf{F}(\mathbf{x})\quad\forall\:\mathbf{x}\in A.$\final
    \end{enumerate}
\end{theorem}

\begin{theorem}
    Sean $A\subseteq\Re^n$ abierto y conexo, $\mathbf{F}:A\subseteq\Re^n\to\Re$ continuo. Las siguientes condiciones sobre $\mathbf{F}$ son equivalentes:
    \begin{enumerate}
        \item $\exists f:A\subseteq\Re^n\to\Re$ de clase $C^1$ tal que $\mathbf{F}(\mathbf{x})=\nabla f(\mathbf{x}),\;\forall\mathbf{x}\in A.$ 
        \item La integral de $\mathbf{F}$ es independiente del camino en $A$.
        \item $\oint_C\mathbf{F}\cdot d\mathbf{s}=0$ para toda curva cerrada simple $C$ en $A$.\final
    \end{enumerate}
\end{theorem}

\textcolor{red}{\textit{esta seccion la copie textual del video Campos Conservativos de youtube. quizas para que quede definida la parte de ``La integral de $\mathbf{F}$ es independiente del camino en $A$'' convenga copiar los teoremas y corolarios del tromba?}}