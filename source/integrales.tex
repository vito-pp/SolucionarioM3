\textcolor{red}{No hace falta la parte de recuerdo de mate 1 con el teorema de valor medio?}

\textcolor{red}{Falta definicion integral de linea sobre trayectorias}

\begin{definition}
    Sea $\Gamma$ una curva simple en $\Re^n$ y sea $\boldsymbol{\sigma}:I\subseteq\Re\to\Re^n$ una parametrizaci\'on regular de $\Gamma$. Entonces
    \[
        \text{long}\:\Gamma=\text{L}(\Gamma)=\sum_{i=1}^{n}\int_{I_i}\|\boldsymbol{\sigma}'(t)\|dt,    
    \]
    donde $\boldsymbol{\sigma}\lvert_{I_i}$ es de clase $C^1$.
\end{definition}

\begin{definition}
    Sea $\Gamma$ una cirva simple en $\Re^n$ y sea $\boldsymbol{\sigma}:[a,b\subseteq\Re\to\Re^n]$ una parametrizaci\'on regular de $\Gamma$. Sea $f:A\subseteq\Re^n\to\Re$ continua sobre $\Gamma$. Se define la integral de l\'inea del campo $f$ sobre la curva $\Gamma$, y se nota por $\int_{\Gamma}f\:ds$ a
    \[
        \int_{\Gamma}f\:ds=\sum_{i=1}^{n}\int_{I_i}f\circ\boldsymbol{\sigma}(t)\|\boldsymbol{\sigma}'(t)\|\:dt,  
    \]
    donde $\boldsymbol{\sigma}\lvert_{I_i}$ es de clase $C^1$.
\end{definition}

\textbf{Aplicaci\'on.}
Cuando $f(x,y)>0$ esta integral tiene una interpretaci\'on geom\'etrica como el ``\'area de una valla". Podemos construir una ``valla" cuya base sea la imagen de $\boldsymbol{\sigma}$ y altura $f(x,y)$ en $(x,y)$. Si $\boldsymbol{\sigma}$ recorre s\'olo una vez la imagen de $\boldsymbol{\sigma}$, la integral $\int_{\boldsymbol{\sigma}}f(x,y)\:ds$ representa el \'area de una lado de la valla.

\textcolor{red}{ilustracion de la aplicacion}

\begin{definition}
    Integral de un campo vectorial a lo largo de una trayectoria.
    Sean $\mathbf{F}:A\subseteq\Re^n\to\Re^n$ un campo vectorial y $\boldsymbol{\sigma}:[a,b]\to A\subseteq\Re^n$ una trayectorial $C^1$ a trozos tales que $\mathbf{F}\circ\boldsymbol{\sigma}:[a,b]\to\Re^n$ es una funci\'on continua. Se define la \textbf{integral de} $\mathbf{F}$ \textbf{a lo largo de} $\boldsymbol{\sigma}$ como
    \[
        \int_{\boldsymbol{\sigma}}\mathbf{F}\cdot d\mathbf{s}=\int_a^b\mathbf{F}\circ\boldsymbol{\sigma}(t)\cdot\boldsymbol{\sigma}'(t)\:dt.  
    \]
\end{definition}

\textbf{Notaci\'on.} $\int_{\boldsymbol{\sigma}}\mathbf{F}\cdot d\mathbf{s}=\int_{\boldsymbol{\sigma}}\mathbf{F}\cdot\hat{\mathbf{t}}\:ds$, donde $\hat{\mathbf{t}}$ es versor tangencial al campo $\mathbf{F}$ en todo punto.

\begin{definition}
    Integral de un campo vecrorial a lo largo de una curva simple.
    Sea $\mathbf{F}:A\subseteq\Re^n\to\Re^n$ un campo vectorial continuo sobre una curva simple oriantada $\Gamma\subset A$. Sea $\boldsymbol{\sigma}:[a,b]\to A\subseteq\Re^n$ una parametrizaci\'on de $\Gamma$ que preserva su oriantaci\'on. Se define la \textbf{integral de} $\mathbf{F}$ \textbf{sobre la curva simple} $\boldsymbol{\Gamma}$ como
    \[
        \int_{\Gamma}\mathbf{F}\cdot d\mathbf{s}=\int_{\boldsymbol{\sigma}}\mathbf{F}\cdot d\mathbf{s}  
    \]
\end{definition}

\textbf{Notaci\'on.} Si $\Gamma$ es cerrada, se suele notar a la integral como
\[
    \oint_{\Gamma}\mathbf{F}\cdot d\mathbf{s},  
\]
y se la llama integral de circulaci\'on de $\mathbf{F}$ sobre $\Gamma$.
