\textcolor{red}{definir rectangulo B en R2. }
\textcolor{red}{definir particion de un rectangulo B. quien es $\Delta A_{ij}$  quien es $B_{ij}$. Explicar la notacion que aparece en la primer definicion }



\begin{definition}
Dada una funci\'on $f:B\to\R$, donde $B$ es un rect\'angulo en $\Rn{2}$, se define la suma de Reimann de $f$ sobre $B$ a 
\[
    S_n=\sum_{i=0}^{n-1}\sum_{j=0}^{n-1} f(c_{ij})\Delta A_{ij},
\]  
donde $c_{ij}\in B_{ij}$, el $ij$-\'esimo rect\'angulo en la partici\'on de $B$, y $\Delta A$ es el \'area de $B_{ij}$.
\end{definition}



\begin{definition} 
    Sea $f:B=[a,b]\times[c,d]\to\R$ una funci\'on. Diremos que $f$ es integrable sobre $B$ si  el l\'imite $\lim_{n\to\infty}S_n$  existe y es finito, en tal caso el valor de dicho  l\'imite  se llama \textbf{integral doble} de $f$ sobre $B$ y se nota 
    \[
          \iint_B f\:dA=\iint_B f(x,y)\:dxdy.
    \]
\end{definition}

\begin{obs}
\textcolor{red}{agregar condiciones sobre f para la  existencia del limite}
\end{obs}


Para extender esta noci\'on de integral a  conjuntos acotados m\'as generales que no sean  rect\'angulos, definimos lo siguiente. 

\begin{definition}\label{int_D} \textcolor{red}{Hacer un dibujo}
Dada $f:D\to\R$, se define la funci\'on $f^*$ tal que \textcolor{red}{quien es D?}
\[
    f^*(x,y)=
    \begin{dcases*}
        f(x,y) & si $(x,y)\in D$ \\[.2cm]
        0        & si $(x,y)\notin D.$
    \end{dcases*}
\]
Entonces si $B$ es un rect\'angulo que contiene a $D$ y $\partial D$ est\'a formada por las gr\'aficas de un n\'umero finito de funciones continuas,  $f^*$ ser\'a integrable.  Se define
\[
    \iint_D f\:dA=\iint_B f^*\:dA.  
\]
\end{definition}

\textcolor{red}{Mejor la explicacion anterior}

\begin{obs} 
La  definici\'on  (\ref{int_D})  es independiente  de la selecci\'on de $B$.
\end{obs}

\begin{propertie}
    Sean $f,\;g$ dos funciones integrables en una regi\'on $D\subset\Rn{2}$, entonces:
    \begin{enumerate}
        \item[i.] $\alpha f+\beta g$ es integrable en $D$, $\forall\;\alpha,\;\beta\in\R$ y, adem\'as,
        \[
            \iint_D \left(\alpha f+\beta g\right)dA=\alpha\iint_D f\:dA+\beta\iint_D g\:dA.
        \]
        \item[ii.] El producto $fg$ es integrable en $D$.
        \item[iii.] Si $|g(x,y)|\geq k>0\;\forall(x,y)\in D$, el cociente $f/g$ es integrable en $D$.
        \item[iv.] Si $f\geq 0$ en $D$, $\iint_D f\:dA\geq0$.
        \item[v.]Si $f\leq g$ en $D$, $\iint_D f\:dA\leq\iint_D g\:dA.$
        \item[vi.]Si $|f|$ es integrable en $D$, entonces 
        \[
            \left|\iint_D f\:dA\right|\leq\iint_D|f|\:dA.  
        \]    
        \item[vii.] Si $D=D_1\cup D_2$ donde $\;D_1\cap D_2$ es un conjunto de medida nula, entonces   $f$ es integrable sobre $D_1$ y sobre $D_2$ y, adem\'as,  
        \[
            \iint_D f\:dA=\iint_{D_1} f\:dA+\iint_{D_2}f\:dA.    
        \]
    \end{enumerate}
\end{propertie}

%-----------------------------------------------------------
\begin{definition}  Un conjunto  $D\subseteq\Rn{2}$ se llama elemental si puede ser descripto de alguna de las siguientes maneras. 
   
    \begin{enumerate}
    \item[i.]
    Existen funciones continuas $\phi_1,\;\phi_2:[a,b]\to\R$ tales que $D$ es el conjunto de puntos $(x,y)$ que satisfacen
    \[
        x\in[a,b], \qquad \phi_1(x)\leq y\leq\phi_2(x),  
    \]% no se pq no funciona \hfill para no usar \qquad
    donde $\phi_1(x) < \phi_2(x)\;\forall\;x\in[a,b].$  En tal  el conjunto $D$ se llama  \textit{elemental de tipo 1.} 
    \item[ii.]
    Si existen funciones continuas $\psi_1,\;\psi_2:[c,d]\to\R$ tales que $D$ es el conjunto de puntos $(x,y)$ que satisfacen
    \[
        y\in[c,d], \qquad \psi_1(y)\leq x\leq\psi_2(y),  
    \]
    donde $\psi_1(y) < \psi_2(y)\;\forall\;y\in[c,d].$ En tal  el conjunto $D$ se llama  \textit{elemental de tipo 2.} 
    \item[iii.]
    $D$ se llama conjunto elemental de \textit{tipo 3} si es \textit{tipo 1} y \textit{tipo 2} a la vez.
    
    
    
    
    \textcolor{red}{Dibujar los casos.}
    
    
    \end{enumerate}
\end{definition}

\begin{propertie} \label{col:fubini}
   
    \begin{enumerate}
        \item  Sea $f:D\to\R$ continua sobre  $D\subseteq\Rn{2}$  un conjunto de tipo 1, entonces
        \[
            \iint_D f=\int_a^b\left(\int_{\phi_1(x)}^{\phi_2(x)}f(x,y)\:dy\right)dx.  
        \]
        \item  Sea $f:D\to\R$ continua sobre  $D\subseteq\Rn{2}$  un conjunto de tipo 2, entonces
        % y las $\psi_i$ son continuas. 
        Entonces
        \[
            \iint_D f=\int_a^b\left(\int_{\psi_1(y)}^{\psi_2(y)}f(x,y)\:dx\right)dy.  
        \]
        \item  Sea $f:D\to\R$ continua sobre  $D\subseteq\Rn{2}$  un conjunto de tipo 3, entonces
        \[
            \iint_D f=\int_a^b\left(\int_{y=\phi_1(x)}^{y=\phi_2(x)}f(x,y)\:dy\right)dx=\int_a^b\left(\int_{x=\psi_1(y)}^{x=\psi_2(y)}f(x,y)\:dx\right)dy. 
        \]
    \end{enumerate}
\end{propertie}
\begin{example}
    Calcular $\iint_D f$, donde $f(x,y)=x^2y$, donde $D$ es la regi\'on del plano encerrado entre $y=x^3$, $y=x^2$, con $x\in[0,1]$. 

    \begin{center}
    \begin{tikzpicture}
        \begin{axis}[
            axis lines = middle,
            xmin = 0, xmax = 1.5,
            ymin = 0, ymax = 1.5,
            xlabel = {$x$},
            ylabel = {$y$},
            domain = 0:1.25,
            samples = 100,
            xtick={0.5,1},
            ytick={0.5,1},
            clip=false,
            ]
        
            % Define the curves
            \addplot[name path=A, blue, thick, domain=0:1.22] {x^2} node[pos=0.75, right] {$y=x^2$};
            \addplot[name path=B, red, thick, domain=0:1.15] {x^3} node[pos=0.3, below right] {$y=x^3$};
        
            % Fill the area between the curves
            \addplot[orange!50, fill opacity=0.5] fill between[of=A and B, soft clip={domain=0:1}];
        
            % Label the enclosed area as D
            \node at (axis cs:0.6,0.29) {$D$};
        
        \end{axis}
    \end{tikzpicture}
    \end{center}

    En este caso 
    \[
        D=\{(x,y)\in\Rn{2}:0\leq x\leq1;\;x^3\leq y\leq x^2\}.  
    \]
    Luego $D$ es de tipo 1. Notar que la \'unica manera de demostrar que un conjunto es elemental es dando su descripci\'on impl\'icita. Por el \autoref{col:fubini}
    \begin{gather*}
        \iint_D x^2y\:dA=\int_0^1\left(\int_{x^3}^{x^2}\:dy\right)dx=\int_0^1\left(x^2\frac{1}{2}y^2\Big\lvert_{x^3}^{x^2}\right)dx\\
        =\int_0^1\frac{1}{2}x^2\left((x^2)^2-(x^3)^2\right)=\int_0^1\frac{1}{2}x^6-\frac{1}{2}x^8\;dx\\=\frac{1}{14}x^7-\frac{1}{18}x^9\Big\lvert_0^1=\frac{1}{14}-\frac{1}{18}.
    \end{gather*}
\end{example}
\begin{definition} %cambie la palabra region --> conjunto
    Se define el \'area de un conjunto $D\subset\Rn{2}$, y se nota $\text{A}(D)$, a  
    \[
        \text{A}(D)=\iint_D1\:dA.
    \]
\end{definition}
\begin{theorem}
    \textbf{Teorema del valor medio para integrales dobles}. Sea $f:D\subseteq\Rn{2}\to\R$ continua y $D$ es un conjunto elemental. Entonces existe $(x_0,y_0)$ en $D$ tal que
    \[
        \int_D f(x,y)\:dA=f(x_0,y_0)\text{A}(D).
    \]
\end{theorem}


\textcolor{red}{Falta definir  $\boldsymbol{D}_\mathbf{F}$}


\begin{definition}
    Sea $\mathbf{F}:A\subseteq\Rn{n}\to\Rn{n}$ un campo $\mathcal{C}^1(A)$ y sea $\boldsymbol{D}_\mathbf{F}$ su matriz diferencial definida en $A$. Se define el \textbf{jacobiano}, o \textbf{determinante jacobiano}, y se lo nota $J_\mathbf{F}$ a 
    \[
        J_\mathbf{F}=\left|\text{det}(\boldsymbol{D}_\mathbf{F})\right|.
    \]
\end{definition}


\textcolor{red}{dos ejemplos  sencillos en n=2 y n=3}


\begin{theorem} % lo escribi como lo tenia en mis apuntes
    \textbf{Teorema del cambio de variables para integrales dobles}. Sea $f:D\subseteq\Rn{2}\to\Rn{2}$ un campo escalar integrable. Sea $\mathbf{T}:D^*\subseteq\Rn{2}\to D$ continua de clase $\mathcal{C}^1$ en $\interior{(D^*)}$, inyectiva en $\interior{(D^*)}$ y tal que $\mathbf{T}(D^*)=D$. Entonces
    \[
        \iint_D f\:dA=\iint_{D^*}(f\circ\mathbf{T})J_\mathbf{T}\:dA.
    \]
\end{theorem}