\begin{definition}\textbf{Conjuntos elementales en }$\Re^2$.\textcolor{red}{region o conjunto?}
    Sea $D\subseteq\Re^2$.
    \begin{enumerate}
    \item[i.]
    Decimos que la regi\'on $D$ es de \textit{tipo 1} si existen funciones continuas $\phi_1,\;\phi_2:[a,b]\to\Re$ tales que $D$ es el conjunto de puntos $(x,y)$ que satisfacen
    \[
        x\in[a,b], \qquad \phi_1(x)\leq y\leq\phi_2(x),  
    \]% no se pq no funciona \hfill para no usar \qquad
    donde $\phi_1(x)\leq\phi_2(x)\;\forall\;x\in[a,b].$
    \item[ii.]
    Decimos que la regi\'on $D$ es de \textit{tipo 2} si existen funciones continuas $\psi_1,\;\psi_2:[c,d]\to\Re$ tales que $D$ es el conjunto de puntos $(x,y)$ que satisfacen
    \[
        y\in[c,d], \qquad \psi_1(y)\leq x\leq\psi_2(y),  
    \]
    donde $\psi_1(y)\leq\psi_2(y)\;\forall\;y\in[c,d].$
    \item[iii.]
    $D$ se llama conjunto de \textit{tipo 3} si es \textit{tipo 1} y \textit{tipo 2} a la vez.
    \end{enumerate}
    A \'estos conjuntos (\textit{tipo 1, 2, 3}) se los llama elementales.
\end{definition}

\begin{corollary} \label{col:fubini}
    \textcolor{red}{corolario de que teorema es?}
    \begin{enumerate}
        \item Sea $D\subseteq\Re^2$ un conjunto de tipo 1. Supongamos que $f:D\to\Re$ es continua y las $\phi_i$ son continuas. Entonces
        \[
            \iint_D f=\int_a^b\left(\int_{y=\phi_1(x)}^{y=\phi_2(x)}f(x,y)\:dy\right)dx.  
        \]
        \item Sea $D\subseteq\Re^2$ un conjunto de tipo 2. Supongamos que $f:D\to\Re$ es continua y las $\psi_i$ son continuas. Entonces
        \[
            \iint_D f=\int_a^b\left(\int_{x=\psi_1(y)}^{x=\psi_2(y)}f(x,y)\:dx\right)dy.  
        \]
        \item Si $D$ es de tipo 3 y $f,\;\phi_i,\psi_i\;$ son continuas, entonces 
        \[
            \iint_D f=\int_a^b\left(\int_{y=\phi_1(x)}^{y=\phi_2(x)}f(x,y)\:dy\right)dx=\int_a^b\left(\int_{x=\psi_1(y)}^{x=\psi_2(y)}f(x,y)\:dx\right)dy. 
        \]
    \end{enumerate}
\end{corollary}
\begin{example}
    Se quiere calcular la integral $\iint_D f$, donde $f(x,y)=x^2y$, mientras que $D$ es la regi\'on del plano encerrado entre $y=x^3$, $y=x^2$, con $x\in[0,1]$. \textcolor{red}{agregar figura}

    En este caso 
    \[
        D=\{(x,y)\in\Re^2:0\leq x\leq1;\;x^3\leq y\leq x^2\}.  
    \]
    Luego $D$ es de tipo 1. Notar que la \'unica manera de demostrar que un conjunto es elemental es dando su descrici\'on impl\'icita. Por el \autoref{col:fubini}
    \begin{gather*}
        \iint_D x^2y\:dA=\int_0^1\left(\int_{x^3}^{x^2}\:dy\right)dx=\int_0^1\left(x^2\frac{1}{2}y^2\Big\lvert_{x^3}^{x^2}\right)dx\\
        =\int_0^1\frac{1}{2}x^2\left((x^2)^2-(x^3)^2\right)=\int_0^1\frac{1}{2}x^6-\frac{1}{2}x^8\;dx\\=\frac{1}{14}x^7-\frac{1}{18}x^9\Big\lvert_0^1=\frac{1}{14}-\frac{1}{18}.
    \end{gather*}
\end{example}
\begin{definition} %cambie la palabra region --> conjunto
    Se define el \'area de un conjunto $D\subset\Re^n$ como la integral, si existe, de la funci\'on 1. Es decir,
    \[
        \text{A}(D)=\iint_D1\:dA.  
    \]
\end{definition}
\begin{theorem}
    \textbf{Teorema del valor medio para integrales dobles}. Suponer que $f:D\to\Re$ es continua y $D$ es un conjunto elemental. Entonces para alg\'un punto $(x_0,y_0)$ en $D$, tenemos
    \[
        \int_D f(x,y)\:dA=f(x_0,y_0)\text{A}(D).
    \]
\end{theorem}
\begin{theorem} % lo escribi como lo tenia en mis apuntes
    \textbf{Teorema del cambio de variables para integrales dobles}. Sea $f:D\subseteq\Re^2\to\Re^2$ un campo escalar integrable. Sea $T:D^*\subseteq\Re^2\to D$ continua de clase $C^1$ en int($D^*$), inyectiva en int($D^*$) y $T(D^*)=D$. Entonces
    \[
        \iint_D f\:dA=\iint_{D^*}f\circ T\:|\text{det}(\boldsymbol{D}_T)|
    \]
    % supuse que ya esta definida la matriz diferencial D_T
    Se suele notar a 
    \[
        |\text{det}(\boldsymbol{D}_T)|=J(T)
    \]
    y se lo llama \textbf{jacobiano} de la matriz diferencial $\boldsymbol{D}_T$.
\end{theorem}