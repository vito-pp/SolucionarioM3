\begin{definition}
Dada una funci\'on continua $f:B\to\Re$, donde $B$ es un rect\'angulo en $\Re^2$, se define la suma de Reimann de $f$ sobre $B$, partiendo los dos lados de $B$ en $n$ partes iguales y formando la suma, a
\[
    S_n=\sum_{i=0}^{n-1}\sum_{j=0}^{n-1} f(c_{ij})\Delta A,
\]  
donde $c_{ij}\in B_{ij}$, el $ij$-\'esimo rect\'angulo en la partici\'on de $B$, y $\Delta A$ es el \'area de $B_{ij}$.\final
\end{definition}

\begin{definition} 
    Sean $B=[a,b]\times[c,d]$ y $f:B\to\Re$ una funci\'on acotada. El l\'imite $\lim_{n\to\infty}S_n$, se llama \textbf{integral doble} de $f$ sobre $B$, y se nota $\iint_U f\:dA$, a
    \[
          \iint_U f\:dA=\iint_U f(x,y)\:dxdy.\finalmath
    \]
\end{definition}

\textcolor{red}{\textit{Aclaramos que el} $\lim_{n\to\infty}B_{ij}=dA$\textit{? Y que $dA=dxdy$}?}

Para extender esta noci\'on de integral a un conjunto acotado m\'as general, esto es, conjuntos que puedan ser encerrados por un rect\'angulo, definimos lo siguiente. 

\begin{definition}
Dada $f:B\to\Re$, se define la funci\'on $f^*$ tal que
\[
    f^*(x,y)=
    \begin{dcases*}
        f(x,y) & si $(x,y)\in D$ \\[.2cm]
        0        & si $(x,y)\notin D.$
    \end{dcases*}
\]
Entonces si $B$ es una caja que contiene a $D$ y $\partial D$ est\'a formada por las gr\'aficas de un n\'umero finito de funciones continuas, $f^*$ ser\'a integrable. Luego, se define
\[
    \iint_D f\:dA=\iint_B f^*\:dA.  \finalmath
\]
\end{definition}

\textbf{Observaci\'on.} Notar que esta integral es independiente de la selecci\'on de $B$.

\begin{propertie}
    Sean $f,\;g$ dos funciones integrables en una regi\'on $D\subset\Re^2$, entonces:
    \begin{enumerate}
        \item[i.] $\alpha f+\beta g$ es integrable en $D$, $\forall\;\alpha,\;\beta\in\Re$ y adem\'as
        \[
            \iint_D \left(\alpha f+\beta g\right)dA=\alpha\iint_D f\:dA+\beta\iint_D g\:dA.
        \]
        \item[ii.] El producto $fg$ es integrable en $D$.
        \item[iii.] Si $|g(x,y)|\geq k>0\;\forall(x,y)\in D$, el cociente $f/g$ es integrable en $D$.
        \item[iv.] Si $f\geq$ en $D$, $\iint_D f\:dA\geq0$.
        \item[v.]Si $f\leq g$ en $D$, $\iint_D f\:dA\leq\iint_D g\:dA.$
        \item[vi.]Si $|f|$ es integrable en $D$, entonces 
        \[
            \left|\iint_D f\:dA\right|\leq\iint_D|f|\:dA.  
        \]    
        \item[vii.] Si $D=D_1\cup D_2,\;D_1\cap D_2\neq\varnothing,$ es una partici\'on de $D$, $f$ es integrable en $D$ $\iff$ $f$ es integrable en $D_1$ y $D_2$. En este caso 
        \[
            \iint_D f\:dA=\iint_{D_1} f\:dA+\iint_{D_2}f\:dA. \finalmath   
        \]
    \end{enumerate}
\end{propertie}

%-----------------------------------------------------------
\begin{definition}\textbf{Conjuntos elementales en }$\Re^2$.
    Sea $D\subseteq\Re^2$.
    \begin{enumerate}
    \item[i.]
    Decimos que el conjunto elemental $D$ es de \textit{tipo 1} si existen funciones continuas $\phi_1,\;\phi_2:[a,b]\to\Re$ tales que $D$ es el conjunto de puntos $(x,y)$ que satisfacen
    \[
        x\in[a,b], \qquad \phi_1(x)\leq y\leq\phi_2(x),  
    \]% no se pq no funciona \hfill para no usar \qquad
    donde $\phi_1(x)\leq\phi_2(x)\;\forall\;x\in[a,b].$
    \item[ii.]
    Decimos que el conjunto elemental $D$ es de \textit{tipo 2} si existen funciones continuas $\psi_1,\;\psi_2:[c,d]\to\Re$ tales que $D$ es el conjunto de puntos $(x,y)$ que satisfacen
    \[
        y\in[c,d], \qquad \psi_1(y)\leq x\leq\psi_2(y),  
    \]
    donde $\psi_1(y)\leq\psi_2(y)\;\forall\;y\in[c,d].$
    \item[iii.]
    $D$ se llama conjunto elemental de \textit{tipo 3} si es \textit{tipo 1} y \textit{tipo 2} a la vez.\final
    \end{enumerate}
\end{definition}

\begin{corollary} \label{col:fubini}
    \begin{enumerate}
        \item Sea $D\subseteq\Re^2$ un conjunto de tipo 1. Y sea $f:D\to\Re$ continua. Entonces
        \[
            \iint_D f=\int_a^b\left(\int_{\phi_1(x)}^{\phi_2(x)}f(x,y)\:dy\right)dx.  
        \]
        \item Sea $D\subseteq\Re^2$ un conjunto de tipo 2. Sea $f:D\to\Re$ continua.
        % y las $\psi_i$ son continuas. 
        Entonces
        \[
            \iint_D f=\int_a^b\left(\int_{\psi_1(y)}^{\psi_2(y)}f(x,y)\:dx\right)dy.  
        \]
        \item Si $D$ es de tipo 3 y $f$
        % ,\;\phi_i,\psi_i\;$ son 
        continua, entonces 
        \[
            \iint_D f=\int_a^b\left(\int_{y=\phi_1(x)}^{y=\phi_2(x)}f(x,y)\:dy\right)dx=\int_a^b\left(\int_{x=\psi_1(y)}^{x=\psi_2(y)}f(x,y)\:dx\right)dy.\finalmath 
        \]
    \end{enumerate}
\end{corollary}
\begin{example}
    Calcular $\iint_D f$, donde $f(x,y)=x^2y$, donde $D$ es la regi\'on del plano encerrado entre $y=x^3$, $y=x^2$, con $x\in[0,1]$. 

    \begin{center}
    \begin{tikzpicture}
        \begin{axis}[
            axis lines = middle,
            xmin = 0, xmax = 1.5,
            ymin = 0, ymax = 1.5,
            xlabel = {$x$},
            ylabel = {$y$},
            domain = 0:1.25,
            samples = 100,
            xtick={0.5,1},
            ytick={0.5,1},
            clip=false,
            ]
        
            % Define the curves
            \addplot[name path=A, blue, thick, domain=0:1.22] {x^2} node[pos=0.75, right] {$y=x^2$};
            \addplot[name path=B, red, thick, domain=0:1.15] {x^3} node[pos=0.3, below right] {$y=x^3$};
        
            % Fill the area between the curves
            \addplot[orange!50, fill opacity=0.5] fill between[of=A and B, soft clip={domain=0:1}];
        
            % Label the enclosed area as D
            \node at (axis cs:0.6,0.29) {$D$};
        
        \end{axis}
    \end{tikzpicture}
    \end{center}

    En este caso 
    \[
        D=\{(x,y)\in\Re^2:0\leq x\leq1;\;x^3\leq y\leq x^2\}.  
    \]
    Luego $D$ es de tipo 1. Notar que la \'unica manera de demostrar que un conjunto es elemental es dando su descripci\'on impl\'icita. Por el \autoref{col:fubini}
    \begin{gather*}
        \iint_D x^2y\:dA=\int_0^1\left(\int_{x^3}^{x^2}\:dy\right)dx=\int_0^1\left(x^2\frac{1}{2}y^2\Big\lvert_{x^3}^{x^2}\right)dx\\
        =\int_0^1\frac{1}{2}x^2\left((x^2)^2-(x^3)^2\right)=\int_0^1\frac{1}{2}x^6-\frac{1}{2}x^8\;dx\\=\frac{1}{14}x^7-\frac{1}{18}x^9\Big\lvert_0^1=\frac{1}{14}-\frac{1}{18}.\finalmath
    \end{gather*}
\end{example}
\begin{definition} %cambie la palabra region --> conjunto
    Se define el \'area de un conjunto $D\subset\Re^n$, y se nota $\text{A}(D)$ como la integral, si existe, de la funci\'on 1. Es decir,
    \[
        \text{A}(D)=\iint_D1\:dA.\finalmath
    \]
\end{definition}
\begin{theorem}
    \textbf{Teorema del valor medio para integrales dobles}. Sea $f:D\subseteq\Re^2\to\Re$ continua y $D$ es un conjunto elemental. Entonces existe $(x_0,y_0)$ en $D$ tal que
    \[
        \int_D f(x,y)\:dA=f(x_0,y_0)\text{A}(D).\finalmath
    \]
\end{theorem}

\begin{definition}
    Sea $\mathbf{F}:A\subseteq\Re^n\to\Re^n$ un campo $C^1(A)$ y sea $\boldsymbol{D}_\mathbf{F}$ su matriz diferencial definida en $A$. Se define el \textbf{jacobiano}, o \textbf{determinante jacobiano}, y se lo nota $J_\mathbf{F}$ a 
    \[
        J_\mathbf{F}=\left|\text{det}(\boldsymbol{D}_\mathbf{F})\right|.\finalmath
    \]
    \textcolor{red}{\textit{quizas esta definicion convnga ponerla en la primera parte teorica.}}
\end{definition}

\begin{theorem} % lo escribi como lo tenia en mis apuntes
    \textbf{Teorema del cambio de variables para integrales dobles}. Sea $f:D\subseteq\Re^2\to\Re^2$ un campo escalar integrable. Sea $\mathbf{T}:D^*\subseteq\Re^2\to D$ continua de clase $C^1$ en $\interior{D^*}$, inyectiva en $\interior{D^*}$ y $\mathbf{T}(D^*)=D$. Entonces
    \[
        \iint_D f\:dA=\iint_{D^*}(f\circ\mathbf{T})J_\mathbf{T}\:dV.\finalmath
    \]
\end{theorem}