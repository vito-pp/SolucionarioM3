\label{sec:diferenciabilidad}

\begin{definition}[Diferenciabilidad de campos escalares] \label{def:dif_escalar}
\mbox{}
Sean $f: A \subset \Rn{n} \to \R$ y $x_o \in \interior{(A)}$. Decimos que $f$ es \emph{diferenciable en $x_o$} si $\exists m \in \Rn{n}$ tal que
\[
  \lim_{x \to x_o} \frac{f(x) - f(x_o) - m \cdot (x - x_o)}{\norm{x - x_o}} = 0.
\]
Si $A$ es un conjunto abierto, decimos que $f$ es \emph{diferenciable} si es diferenciable en cada punto $x_o \in A$.
\end{definition}

\begin{definition}[Diferenciabilidad, caso general] \label{def:dif_general}
\mbox{}

Sean $f: A \subset \Rn{n} \to \Rn{m}$ y $x_o \in \interior{(A)}$. Decimos que $f$ es \emph{diferenciable en $x_o$} si $\exists M \in \Rnp{m}{n}$ tal que
\[
  \lim_{x \to x_o} \frac{\norm{f(x) - f(x_o) - M \cdot (x - x_o)}}{\norm{x - x_o}} = 0.
\]
Si $A$ es un conjunto abierto, decimos que $f$ es \emph{diferenciable} si es diferenciable en cada punto $x_o \in A$.
\end{definition}

\begin{theorem}[Diferenciabilidad y continuidad] \label{teo:difCont}
\mbox{}

Sean $f: A \subset \Rn{n} \to \R$ y $x_o \in \interior{(A)}$. Si $f$ es diferenciable en $x_o$, entonces $f$ es continua en $x_o$.
 \begin{proof}
 \mbox{}
 
 Como $f$ es diferenciable en $x_o, \exists m \in \Rn{n}$ tal que
 \[
  \lim_{x \to x_o} \frac{f(x) - f(x_o) - m \cdot (x - x_o)}{\norm{x - x_o}} = 0.
 \]
Por lo tanto, $\exists \delta_o > 0$ tal que 
\[
 x \in B_{\delta_o}^* \cap A \then 
 \abs{ \frac{f(x) - f(x_o) - m \cdot (x - x_o)}{\norm{x - x_o}} } < 1.
\]
Entonces, para este $\delta_o$, se cumple que
\[
 \abs{ f(x) - f(x_o) - m \cdot (x - x_o) } < \norm{x - x_o}, \, \forall x \in B_{\delta_o}^* \cap A.
\]
% Como ambos miembros de esta desigualdad (estricta) son iguales cuando $x = x_o$, vale que
% \[
%  \abs{ f(x) - f(x_o) - m \cdot (x - x_o) } \le \norm{x - x_o}, \, \forall x \in B_{\delta_o} \cap A.
% \]
Por lo tanto,
\begin{align*}
\abs{ f(x) - f(x_o) } &=   \abs{ f(x) - f(x_o) - m \cdot (x - x_o) + m \cdot (x - x_o) } \le \\
                      &\le \abs{ f(x) - f(x_o) - m \cdot (x - x_o) } + \abs{ m \cdot (x - x_o) } < \\
                      &< \norm{x - x_o} + \abs{ m \cdot (x - x_o) }, \, 
                      \forall x \in B^*_{\delta_o} \cap A.
\end{align*}
Por la desigualdad de Cauchy-Schwarz, $\abs{ m \cdot (x - x_o) } \le \norm{m} \norm{x - x_o}$, y entonces:
\begin{align*}
\abs{ f(x) - f(x_o) } &< \norm{x - x_o} + \abs{ m \cdot (x - x_o) } \le \\
                      &\le \norm{x - x_o} + \norm{m} \norm{x - x_o} = \\
                      &=  \left( 1 + \norm{m} \right) \norm{x - x_o}, \, 
                      \forall x \in B^*_{\delta_o} \cap A.
\end{align*}
Esta condici\'on garantiza la continuidad de $f$ en $x_o$. En efecto, dado $\epsilon > 0$, sea $\delta = \min \{\delta_o, \frac{\epsilon}{1 + \norm{m}} \}$. Para este $\delta$ se cumple que:
\begin{align*}
 \abs{ f(x) - f(x_o) } &< \left( 1 + \norm{m} \right) \norm{x - x_o} < \\
                       &<   \left( 1 + \norm{m} \right) \delta \le \\
                       &\le \left( 1 + \norm{m} \right) \frac{\epsilon}{1 + \norm{m}} = \epsilon, 
                       \, \forall x \in B^*_{\delta} \cap A,
\end{align*}
lo que, por definici\'on de l\'imite \footnote{Como $x_o \in \interior{(A)}$, $x_o$ es un punto de acumulaci\'on de $A$. }, significa que 
\[
 \lim_{x \to x_o}f(x) = f(x_o).
\]
Es decir, $f$ es continua en $x_o$.
\end{proof}
\begin{obs} La implicaci\'on rec\'iproca de la proposici\'on \eqref{teo:difCont} es falsa. 
 \begin{proof} La funci\'on
  \begin{align*}
      g & :\R \to \R \\
        & g(t) = \abs{t}
 \end{align*}
es continua en $t_o = 0$, pero no es diferenciable en $t_o = 0$. Se deja como ejercicio para el lector verificar esta afirmaci\'on.
 \end{proof}
\end{obs}
\end{theorem}

\begin{theorem}[Diferenciabilidad y existencia de derivadas parciales] \label{teo:difDer}
\mbox{}

\iffalse
Sean $f: A \subset \Rn{2} \to \R$ y $(x_o, y_o) \in \interior{(A)}$, tales que $\exists \alpha, \beta \in \R \, /$
\[
 \lim_{(x,y) \to (x_o,y_o)} \frac{f(x,y) - f(x_o,y_o) - \alpha(x - x_o) - \beta(y - y_o)}
                                 {\norm{(x - x_o, y - y_o)}} = 0
\]
(es decir, $f$ es diferenciable en $(x_o,y_o)$). Entonces existen las derivadas parciales de $f$ en $(x_o,y_o)$ y, adem\'as,
\[
 \dif{f}{x} (x_o,y_o) = \alpha, \quad \dif{f}{y}(x_o,y_o) = \beta.
\]

 \begin{proof}
 \mbox{}
 
Sea $(x,y) = (x_o,y_o) + h e_1$, siendo $h \in \R$ y $e_1$ el primer vector can\'onico ($e_1 = (1,0)$). Es decir, definimos una funci\'on $g:(-\delta,\delta) \to \Rn{2} / g(h) = (x_o,y_o) + h e_1$, con $\delta > 0$, lo suficientemente peque\~{n}o como para que $\im(g) \subset A$ \footnote{Cu\'al de las hip\'oteis asegura la existencia de un tal $\delta$?}, y llamamos $(x,y) = g(h)$. Notemos que $g(h) \to (x_o,y_o)$ cuando $h \to 0$. 

Por la diferenciabilidad, tenemos que 
\[
 \lim_{(x,y) \to (x_o,y_o)} \frac{f(x,y) - f(x_o,y_o) - \alpha(x - x_o) - \beta(y - y_o)}
                                 {\norm{(x - x_o, y - y_o)}} = 0
\]
y entonces, si componemos con la funci\'on $g$ (reemplazamos $(x,y) = (x_o,y_o) + h e_1$), la propiedad del l\'imite de la composici\'on \eqref{prop:lim_comp} implica que
\begin{align*}
  0 & = \lim_{h \to 0} \frac{f(x_o + h,y_o) - f(x_o,y_o) - \alpha(x_o + h - x_o) - \beta(y_o - y_o)}
                                 {\norm{(x_o + h - x_o, y_o - y_o)}} = \\
    & = \lim_{h \to 0} \frac{f(x_o + h,y_o) - f(x_o,y_o) - \alpha h}{\norm{h e_1}} = \\
    & = \lim_{h \to 0} \frac{f(x_o + h,y_o) - f(x_o,y_o) - \alpha h}{\abs{h}} .
\end{align*}                                
Ahora bien, este l\'imite es $0$ si y solo si
\[
 \lim_{h \to 0} \abs{ \frac{f(x_o + h,y_o) - f(x_o,y_o) - \alpha h}{\abs{h}} } = 0,
\]
por la propiedad \eqref{prop:lim_1}. Entonces,
\begin{align*}
  0 & =  \lim_{h \to 0} \abs{ \frac{f(x_o + h,y_o) - f(x_o,y_o) - \alpha h}{\abs{h}} }
      = \lim_{h \to 0} \frac{ \abs{ f(x_o + h,y_o) - f(x_o,y_o) - \alpha h }}{\abs{ \, \abs{h} \, }} = \\
    & = \lim_{h \to 0} \frac{ \abs{ f(x_o + h,y_o) - f(x_o,y_o) - \alpha h }}{\abs{h}}
      = \lim_{h \to 0} \abs{ \frac{ f(x_o + h,y_o) - f(x_o,y_o) - \alpha h }{h}},
\end{align*} 
lo cual, otra vez por la propiedad \eqref{prop:lim_1}, implica que 
\[
 0 = \lim_{h \to 0} \frac{ f(x_o + h,y_o) - f(x_o,y_o) - \alpha h }{h} =
     \lim_{h \to 0} \left( \frac{ f(x_o + h,y_o) - f(x_o,y_o) }{h} - \alpha \right) .
\]
\fi

\iffalse
Entonces, por la propiedad \eqref{prop:lim_2}, tenemos que
\[
 \lim_{h \to 0} \frac{ f(x_o + h,y_o) - f(x_o,y_o) }{h} = \alpha,
\]
\fi
Sean $f: A \subset \Rn{n} \to \R$ y $x_o \in \interior{(A)}$, tales que $\exists m = (m_1, m_2, \cdots, m_n) \in \Rn{n} \, /$
\[
 \lim_{x \to x_o} \frac{f(x) - f(x_o) - m \cdot (x - x_o)}
                                 {\norm{x - x_o}} = 0
\]
(es decir, $f$ es diferenciable en $x_o$). Entonces existen las derivadas parciales de $f$ en $x_o$ y, adem\'as,
\[
 \dif{f}{x_j} (x_o) = m_j \quad \forall j = 1, 2, \cdots, n.
\]

 \begin{proof}
 \mbox{}
 
Sea $x = x_o + h e_j$, siendo $h \in \R$ y $e_j$ el $j$-\'esimo vector can\'onico (el vector de $\Rn{n}$ que tiene un $1$ en la posici\'on $j$ y ceros en todas las dem\'as). Es decir, definimos una funci\'on $g:(-\delta,\delta) \to \Rn{n} / g(h) = x_o + h e_j$, con $\delta > 0$, lo suficientemente peque\~{n}o como para que $\im(g) \subset A$ \footnote{¿Cul\'al de las hip\'otesis asegura la existencia de un tal $\delta$?}, y llamamos $x = g(h)$. Notemos que $g(h) \to x_o$ cuando $h \to 0$. 

Por la diferenciabilidad, tenemos que 
\[
 \lim_{x \to x_o} \frac{f(x) - f(x_o) - m \cdot (x - x_o)}
                                 {\norm{x - x_o}} = 0
\]
y entonces, si componemos con la funci\'on $g$ (reemplazamos $x = x_o + h e_j$), la propiedad del l\'imite de la composici\'on \eqref{prop:lim_comp} implica que
\begin{align*}
  0 & = \lim_{h \to 0} \frac{f(x_o + h e_j) - f(x_o) - m \cdot ((x_o + he_j) - x_o)}
                                 {\norm{(x_o + he_j) - x_o}} = \\
    & = \lim_{h \to 0} \frac{f(x_o + h e_j) - f(x_o) - m \cdot (he_j)}{\norm{h e_j}} = \lim_{h \to 0} \frac{f(x_o + h e_j) - f(x_o) - h(m \cdot e_j)}{\norm{h e_j}} =\\
    & = \lim_{h \to 0} \frac{f(x_o + h e_j) - f(x_o) - h m_j}{\abs{h}} .
\end{align*}                                
Ahora bien, este l\'imite es $0$ si y solo si
\[
 \lim_{h \to 0} \abs{ \frac{f(x_o + h e_j) - f(x_o) - h m_j}{\abs{h}} } = 0,
\]
por la propiedad \eqref{prop:lim_1}. Entonces,
\begin{align*}
  0 & = \lim_{h \to 0} \abs{ \frac{f(x_o + h e_j) - f(x_o) - h m_j}{\abs{h}} }
      = \lim_{h \to 0} \frac{\abs{f(x_o + h e_j) - f(x_o) - h m_j}}{\abs{\abs{h}}} = \\
    & = \lim_{h \to 0} \frac{\abs{f(x_o + h e_j) - f(x_o) - h m_j}}{\abs{h}}
      = \lim_{h \to 0} \abs{ \frac{f(x_o + h e_j) - f(x_o) - h m_j}{h} },
\end{align*} 
lo cual, otra vez por la propiedad \eqref{prop:lim_1}, implica que 
\[
 0 = \lim_{h \to 0} \frac{f(x_o + h e_j) - f(x_o) - h m_j}{h} =
     \lim_{h \to 0} \left( \frac{f(x_o + h e_j) - f(x_o)}{h} - m_j \right) .
\]

Es f\'acil probar por definici\'on que esta \'ultima condici\'on implica que
\[
 \lim_{h \to 0} \frac{ f(x_o + h e_j) - f(x_o) }{h} = m_j.
\]
En efecto, dado $\epsilon > 0$ existe $\delta > 0$ tal que 
\[
 \abs{ \left( \frac{ f(x_o + h e_j) - f(x_o) }{h} - m_j \right) - 0 } < \epsilon, \, \forall h : 0 < \abs{h} < \delta,
\]
porque el l\'imite es 0. Como claramente 
\[
 \abs{ \left( \frac{ f(x_o + h e_j) - f(x_o) }{h} - m_j \right) - 0 } = 
 \abs{\frac{ f(x_o + h e_j) - f(x_o) }{h} - m_j},
\]
se tiene (por definici\'on de l\'imite) que 
\[
 \lim_{h \to 0} \frac{ f(x_o + h e_j) - f(x_o) }{h} = m_j.
\]
Esto \'ultimo, por definici\'on de derivada parcial, significa que
\[
  \dif{f}{x_j} (x_o) = m_j,
\]
como quer\'iamos probar. 

 \end{proof}

\begin{obs}
 Es un error decir, en el \'ultimo paso de la demostraci\'on, que por la linealidad del l\'imite
 \[
  \lim_{h \to 0} \left( \frac{f(x_o + h e_j) - f(x_o)}{h} - m_j \right) = 
  \lim_{h \to 0} \left( \frac{f(x_o + h e_j) - f(x_o)}{h} \right) - \lim_{h \to 0} m_j = 0,
 \]
ya que la existencia del l\'imite 
\[
\lim_{h \to 0} \left( \frac{f(x_o + h e_j) - f(x_o)}{h} \right)
\]
es parte de lo que debemos probar.
\end{obs}
\begin{obs} La implicaci\'on rec\'iproca de la proposici\'on \eqref{teo:difDer} es falsa.
\begin{proof}
\mbox{}
  
La funci\'on
 \[
       f :\Rn{2} \to \R
      \]
      \[
        f(x,y) = 
        \begin{cases}
         0 & \text{ si } x y = 0  \\
         1 & \text{ en otro caso }
        \end{cases}
\]
tiene derivadas parciales en el origen, pero no es diferenciable en el origen.\\
Se deja como ejercicio para el lector la demostraci\'on de esta afirmaci\'on.
\end{proof}
\end{obs}
\end{theorem}

Una consecuencia directa del teorema \eqref{teo:difDer} es el siguiente corolario:
\begin{corollary}
 Sean $f: A \subset \Rn{n} \to \R$ y $x_o \in \interior{(A)}$. Son equivalentes:
 \begin{enumerate} %[I.]
  \item $f$ es diferenciable en $x_o$;
  \item existen todas las derivadas parciales de $f$ en $x_o$ y 
  \[
   \lim_{x \to x_o} \frac{f(x) - f(x_o) - \grad f (x_o) \cdot (x - x_o)}{\norm{x - x_o}} = 0.
  \]
 \end{enumerate}
\end{corollary}

\begin{theorem}[Diferenciabilidad y existencia de derivadas direccionales]\label{teo:difDir}
% La idea es incluir tres demostraciones, una usando la misma idea que para la demo de las derivadas parciales, otra usando que el delta f se puede escribir de una determinada manera y construyendo el cociente incremental, una tercera utilizando la regla de la cadena
\mbox{}

Sea $f: A \subset \Rn{n} \to \R$ diferenciable en $x_o \in \interior{(A)}$. Entonces, $\forall \check{u} \in \Rn{n} \text{tal que } \norm{\check{u}} = 1$ existe en $x_o$ la derivada de $f$ en la direcci\'on de $\check{u}$ y, adem\'as,
\[
 \frac{\partial f}{\partial \check{u}} (x_o) = \grad f (x_o) \cdot \check{u}.
\]

 \begin{proof}
 \mbox{}
 
 Sea $x = x_o + h \check{u}$, siendo $h \in \R$. Es decir, definimos una funci\'on 
 \begin{align*}
      g & :(-\delta,\delta) \to \Rn{n} \\
        & g(h) = x_o + h \check{u}
 \end{align*}

con $\delta > 0$, lo suficientemente peque\~{n}o como para que $\im(g) \subset A$, y llamamos $x = g(h)$. Notemos que $g(h) \to x_o$ cuando $h \to 0$. 

Por la diferenciabilidad, tenemos que 
\[
 \lim_{x \to x_o} \frac{f(x) - f(x_o) - \grad f (x_o) \cdot (x - x_o)}
                                 {\norm{x - x_o}} = 0
\]
y entonces, si componemos con la funci\'on $g$ (reemplazamos $x = x_o + h \check{u}$), la propiedad del l\'imite de la composici\'on \eqref{prop:lim_comp} implica que
\begin{align*}
  0 & = \lim_{h \to 0} \frac{f(x_o + h \check{u}) - f(x_o) - \grad f (x_o) \cdot ((x_o + h _{\delta}(x_o)\check{u}) - x_o) }
                    {\norm{(x_o + h \check{u}) - x_o}} = \\
    & = \lim_{h \to 0} \frac{f(x_o + h \check{u}) - f(x_o) - \grad f (x_o) \cdot (h \check{u}) }
                    {\norm{h \check{u}}} = \\
    & = \lim_{h \to 0} \frac{f(x_o + h \check{u}) - f(x_o) - h \grad f (x_o) \cdot \check{u} }
                    {\abs{h} \norm{\check{u}}} = \\
    & = \lim_{h \to 0} \frac{f(x_o + h \check{u}) - f(x_o) - h \grad f (x_o) \cdot \check{u} }
                    {\abs{h}}.  
\end{align*}                                
Como en la demostraci\'on de la propiedad \eqref{teo:difDer}, este l\'imite es $0$ si y solo si
\[
 0 = \lim_{h \to 0} \frac{f(x_o + h \check{u}) - f(x_o) - h \grad f (x_o) \cdot \check{u}}{h} = 
     \lim_{h \to 0} \left( \frac{f(x_o + h \check{u}) - f(x_o)}{h} - \grad f (x_o) \cdot \check{u} \right),
\]
que, como en la demostraci\'on de \eqref{teo:difDer}, implica que 
\[
 \lim_{h \to 0} \frac{f(x_o + h \check{u}) - f(x_o)}{h} = \grad f (x_o) \cdot \check{u}
\]
y por lo tanto, por definici\'on,
\[
 \frac{\partial f}{\partial \check{u}} (x_o) = \grad f (x_o) \cdot \check{u}.
\]
 \end{proof}
\end{theorem}

\begin{theorem}[Valores extremos de la derivada direccional] \label{teo:max_deriv}
\mbox{}

 Sean $f: A \subset \Rn{n} \to \R$ diferenciable en un punto $x_o \in \interior{(A)}$. Entonces 
 \[
  \max_{\norm{\check{u}} = 1} \frac{\partial f}{\partial \check{u}} (x_o) = \norm{\grad f (x_o)}, \quad 
  \min_{\norm{\check{u}} = 1} \frac{\partial f}{\partial \check{u}} (x_o) = -\norm{\grad f (x_o)}
 \]
y, si $\grad f(x_o) \ne \mathbf{0}$,
\[
 \frac{\partial f}{\partial \check{u}} (x_o) = \norm{\grad f (x_o)} \iff \check{u} = \frac{\grad f(x_o)}{\norm{\grad f(x_o)}}
\]
y
\[
 \frac{\partial f}{\partial \check{u}} (x_o) = -\norm{\grad f (x_o)} \iff \check{u} = -\frac{\grad f(x_o)}{\norm{\grad f(x_o)}}.
\]
\begin{proof}
\mbox{}

 Como $f$ es diferenciable en $x_o$, por el teorema \eqref{teo:difDir}, $\forall \check{u} \in \Rn{n} \text{tal que } \norm{\check{u}} = 1$ existe en $x_o$ la derivada de $f$ en la direcci\'on de $\check{u}$ y, adem\'as,
\[
 \frac{\partial f}{\partial \check{u}} (x_o) = \grad f (x_o) \cdot \check{u}.
\]
Por la desigualdad de Cauchy-Schwarz, 
\[
 \abs{\frac{\partial f}{\partial \check{u}} (x_o)} = \abs{\grad f (x_o) \cdot \check{u}} \le \norm{\grad f (x_o)}\norm{\check{u}} = \norm{\grad f (x_o)},
\]
de modo que tenemos las cotas:
\[
 -\norm{\grad f (x_o)} \le \frac{\partial f}{\partial \check{u} (x_o)} \le \norm{\grad f (x_o)}.
\]
Adem\'as, la desigualdad de Cauchy-Schwarz nos dice que vale
\[
 \abs{\frac{\partial f}{\partial \check{u}} (x_o)} = \norm{\grad f (x_o)}
\]
si y solo si el conjunto $\{\grad f(x_o), \check{u} \}$ es linealmente dependiente, condici\'on que, si $\grad f(x_o) \ne \mathbf{0}$, es equivalente a decir que $\check{u}$ es un vector unitario en la direcci\'on de $\grad f(x_o)$. Existen exactamente dos vectores $\check{u}$ con esta caracter\'istica:
\[
 \check{u}_1 = \frac{\grad f(x_o)}{\norm{\grad f(x_o)}} \quad \text{ y } \quad \check{u}_2 = -\frac{\grad f(x_o)}{\norm{\grad f(x_o)}}.
\]
Por c\'alculo directo:
\begin{align*}
 \frac{\partial f}{\partial \check{u}_1} (x_o) &= \grad f (x_o) \cdot \frac{\grad f(x_o)}{\norm{\grad f(x_o)}} = \frac{ \grad f (x_o) \cdot \grad f(x_o) }{\norm{\grad f(x_o)}} = \frac{\norm{\grad f (x_o)}^2 }{\norm{\grad f(x_o)}} = \norm{\grad f (x_o)} \\
%  
 \frac{\partial f}{\partial \check{u}_2} (x_o) &= \grad f (x_o) \cdot \left( - \frac{\grad f(x_o)}{\norm{\grad f(x_o)}} \right) = -\frac{ \grad f (x_o) \cdot \grad f(x_o) }{\norm{\grad f(x_o)}} = -\frac{\norm{\grad f (x_o)}^2 }{\norm{\grad f(x_o)}} = -\norm{\grad f (x_o)},
\end{align*}
de modo que efectivamente
\[
 \max_{\norm{\check{u}}=1} \frac{\partial f}{\partial \check{u}} (x_o) = \norm{\grad f (x_o)}
\]
y el m\'aximo se realiza en $\check{u}_1 = \frac{\grad f(x_o)}{\norm{\grad f(x_o)}}$,
y
\[
 \min_{\norm{\check{u}}=1} \frac{\partial f}{\partial \check{u}} (x_o) = -\norm{\grad f (x_o)}
\]
y el m\'onimo se realiza en $\check{u}_2 = -\frac{\grad f(x_o)}{\norm{\grad f(x_o)}}$. \\

Finalmente, observemos que, si $\grad f(x_o) = \mathbf{0}$, 
\[
 \frac{\partial f}{\partial \check{u}} (x_o) = \grad f (x_o) \cdot \check{u} = 0 = \norm{\grad f(x_o)} \quad \forall \check{u} \in \Rn{n},
\]
y por lo tanto:
\[
 \max_{\norm{\check{u}}=1} \frac{\partial f}{\partial \check{u}} (x_o) = \norm{\grad f(x_o)} = 0
\]
y
\[
 \min_{\norm{\check{u}}=1} \frac{\partial f}{\partial \check{u}} (x_o) = -\norm{\grad f(x_o)} = 0.
\]

\end{proof}

\end{theorem}

\begin{theorem}[Regla de la cadena] \label{teo:cadena}
\mbox{}

  Sean $f, g$ funciones 
    \begin{align*}
    g &:A \subset \Rn{n} \to B \subset \Rn{m} \\
    f &:B \subset \Rn{m} \to \Rn{p},
    \end{align*}
  y un punto $x_o \in \interior{A} \text{ tal que } g(x_o) \in \interior{B}$. Supongamos que $g$ es diferenciable en $x_o$ y que $f$ es diferenciable en $yo = g(x_o)$. Entonces la composic\'on $f \circ g$ es diferenciable en $x_o$ y, adem\'as,
  \[
   \mathbf{D}(f \circ g)_{(x_o)} = \mathbf{D}f_{(y_o)} 
   \mathbf{D}g_{(x_o)}.
  \]
\end{theorem}

\begin{theorem}[Teorema de la Funci\'on Inversa] \label{teo:inversa}
\mbox{}

 Sean $A \subset \Rn{n}$ un conjunto abierto y $F: A \subset \Rn{n} \to \Rn{n}$ una funci\'on de clase $\mathcal{C}^1$. Sea $x_o \in A$ y supongamos que $\det(\mathbf{D}F_{(x_o)}) \ne 0$. Entonces existen un entorno abierto $U \subset A$ de $x_o$ y un entorno abierto $V \subset \Rn{n}$ de $F(x_o)$ tal que $F:U \to V$ (la restricci\'on de $F$ a $U$) tiene inversa $F^{-1}:V \to U$. Esta funci\'on $F^{-1}$ es de clase $\mathcal{C}^1$ y, adem\'as,
 \[
  \mathbf{D}F^{-1}_{(F(x_o))} = \left[ \mathbf{D}F_{(x_o)} \right]^{-1}.
 \]
 Si $F$ es de clase $\mathcal{C}^p, p \ge 1,$ entonces $F^{-1}$ tambi\'on.
\end{theorem}

\begin{theorem}[Caso particular del Teorema de la Funci\'on Impl\'icita] \label{teo:implicit_part}
\mbox{}

Sean $A \subset \Rn{3}$ un conjunto abierto y $F: A \to \R$ una funci\'on de clase $\mathcal{C}^1$. Sea $(x_o,y_o,z_o) \in A$ tal que $F(x_o,y_o,z_o) = 0$. Supongamos que 
 \[
   F_z (x_o,y_o,z_o) \ne 0.
 \]
 Entonces existen un entorno abierto $U \subset \Rn{2}$ de $(x_o,y_o)$
 %, un entorno abierto $V \subset \Rn{m}$ de $y_o$ 
 y una \'onica funci\'on $g: U \to \R$ tales que $g(x_o,y_o) = z_o$ y
 \[
  F(x,y,g(x,y)) = 0 \, \forall (x,y) \in U.
 \]
 M\'as a\'on, $g$ es de clase $\mathcal{C}^1$ y, adem\'as, $\forall (x,y) \in U$ vale que
 \begin{align*}
  g_{x}(x,y) &= -\frac{F_x (x,y,g(x,y))}{F_z (x,y,g(x,y))}  \\
  g_{y}(x,y) &= -\frac{F_y (x,y,g(x,y))}{F_z (x,y,g(x,y))} \\
 \end{align*}
 Si $F$ es de clase $\mathcal{C}^p, p \ge 1,$ entonces $g$ tambi\'on.
 
\end{theorem}


\begin{theorem}[Teorema de la Funci\'on Impl\'icita] \label{teo:implicit}
\mbox{}

 Sean $A \subset \Rn{n} \times \Rn{m}$ un conjunto abierto y $F: A \to \Rn{m}$ una funci\'on de clase $\mathcal{C}^1$. Sea $(x_o,z_o) \in A$ tal que $F(x_o,z_o) = \mathbf{0}$. Formemos el determinante
 \[
  \Delta = \det \begin{bmatrix*}[r]
                \dif{f_1}{z_1} & \dif{f_1}{z_2} & \cdots & \dif{f_1}{z_m} \\
                               &                &        &                \\
                \dif{f_2}{z_1} & \dif{f_2}{z_2} & \cdots & \dif{f_2}{z_m} \\
                               &                &        &                \\
                \vdots         &                & \ddots & \vdots         \\
                               &                &        &                \\
                \dif{f_m}{z_1} & \dif{f_m}{z_2} & \cdots & \dif{f_m}{z_m}
                \end{bmatrix*},
 \]
 donde $F = (f_1, f_2, \cdots , f_m)$ y todas las derivadas est\'on evaluadas en $(x_o,z_o)$. Si $\Delta \ne 0$, entonces existen un entorno abierto $U \subset \Rn{n}$ de $x_o$
 %, un entorno abierto $V \subset \Rn{m}$ de $z_o$ 
 y una \'onica funci\'on $g: U \to \Rn{m}$ tales que $g(x_o) = z_o$ y
 \[
  F(x,g(x)) = \mathbf{0} \, \forall x \in U.
 \]
 M\'as a\'on, $g$ es de clase $\mathcal{C}^1$ y, adem\'as, si $g = (g_1, g_2, \cdots , g_m)$,
 \[
  \begin{bmatrix*}[r]
                \dif{g_1}{x_1} & \dif{g_1}{x_2} & \cdots & \dif{g_1}{x_n} \\
                               &                &        &                \\
                \dif{g_2}{x_1} & \dif{g_2}{x_2} & \cdots & \dif{g_2}{x_n} \\
                               &                &        &                \\
                \vdots         &                & \ddots & \vdots         \\
                               &                &        &                \\
                \dif{g_m}{x_1} & \dif{g_m}{x_2} & \cdots & \dif{g_m}{x_n}
                \end{bmatrix*} =
 -\begin{bmatrix*}[r]
                \dif{f_1}{z_1} & \dif{f_1}{z_2} & \cdots & \dif{f_1}{z_m} \\
                               &                &        &                \\
                \dif{f_2}{z_1} & \dif{f_2}{z_2} & \cdots & \dif{f_2}{z_m} \\
                               &                &        &                \\
                \vdots         &                & \ddots & \vdots         \\
                               &                &        &                \\
                \dif{f_m}{z_1} & \dif{f_m}{z_2} & \cdots & \dif{f_m}{z_m}
                \end{bmatrix*}^{-1}
  \begin{bmatrix*}[r]
                \dif{f_1}{x_1} & \dif{f_1}{x_2} & \cdots & \dif{f_1}{x_n} \\
                               &                &        &                \\
                \dif{f_2}{x_1} & \dif{f_2}{x_2} & \cdots & \dif{f_2}{x_n} \\
                               &                &        &                \\
                \vdots         &                & \ddots & \vdots         \\
                               &                &        &                \\
                \dif{f_m}{x_1} & \dif{f_m}{x_2} & \cdots & \dif{f_m}{x_n}
                \end{bmatrix*}.            
 \]
Si $F$ es de clase $\mathcal{C}^p, p \ge 1,$ entonces $g$ tambi\'on.
 
\end{theorem}
