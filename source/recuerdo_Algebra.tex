\begin{definition} Se define el  espacio $\mathbb{R}^n$  como el conjunto formado por todas las $n$-uplas de n\'umeros reales ordenados.   Es decir:
    \begin{equation*}
        \mathbb{R}^n = \left\{ \boldsymbol{v} / \boldsymbol{v}=(x_1,x_2,x_3,...,x_n), \; x_i \in \mathbb{R} \;\forall \;   1\leq  i \leq n  \right\}
    \end{equation*}
\end{definition}






\begin{definition}  Sean  $\mathbb{V}$ y $\mathbb{W}$ sub-espacios vectoriales de $\mathbb{R}^n$.  Se  define a  una transformación lineal  como a  una función 
    \begin{equation*}
       \boldsymbol{T}: \mathbb{V} \rightarrow  \mathbb{W}
    \end{equation*}
    que cumple con la condición de linealidad:
    \begin{equation}\label{tl}
        \boldsymbol{T}(\alpha \cdot \boldsymbol{v_1} + \beta \cdot \boldsymbol{v_2})=\alpha \cdot \boldsymbol{T}(\boldsymbol{v_2}) + \beta \cdot \boldsymbol{T}(\boldsymbol{v_2}),
    \end{equation}
    
    donde $\boldsymbol{v_1},\boldsymbol{v_2} \in \mathbb{V}$ y $\alpha,\beta \in \mathbb{R}$. 
\end{definition}



\begin{definition} Sean    $\boldsymbol{v}\in\mathbb{R}^n$   no nulo y  $\boldsymbol{x_0}\in\mathbb{R}^n,$  el conjunto $L \subset \mathbb{R}^n $  que cumple  
 \begin{equation*}
      L = \{   \boldsymbol{x} \in\mathbb{R}^n  \:| \:    \boldsymbol{x}=  \lambda \cdot \boldsymbol{v} + \boldsymbol{x_0},  \:  \mbox{  con }  \lambda\in\mathbb{R} \}.
    \end{equation*}  es una recta en $ \mathbb{R}^n $  y a la ecuaci\'on $$ \boldsymbol{x}=  \lambda \cdot \boldsymbol{v} + \boldsymbol{x_0}$$  se  llama  \textbf{expresi\'on vectorial} de la misma.  Adem\'as, llamaremos al vector $\boldsymbol{x_0}$   como \textbf{vector posici\'on} ( o punto de paso )  de la recta   y al vector  $\boldsymbol{v}$ como \textbf{vector director} de la recta.
 \end{definition}


\textcolor{red}{DIBUJO}


\begin{definition} Sean    $\boldsymbol{n}\in\mathbb{R}^3$   no nulo  y  $\boldsymbol{x_0}\in\mathbb{R}^3,$  el conjunto $\pi \subset \mathbb{R}^3 $  que cumple  
    \begin{equation*}
      \pi  = \{   \boldsymbol{x} \in\mathbb{R}^3  \:| \:    \boldsymbol{n}\cdot(\boldsymbol{x} - \boldsymbol{x_0}) =0   \}.
    \end{equation*} es un plano en $\mathbb{R}^3$  y a la ecuaci\'on   $$ \boldsymbol{n}\cdot(\boldsymbol{x} - \boldsymbol{x_0}) =0$$  se llama  \textbf{expresi\'on canónica} del mismo.  Adem\'as, llamaremos al vector $\boldsymbol{n}$   como \textbf{vector normal} del plano 
    y al vector  $\boldsymbol{x_0}$  \textbf{ punto de paso }  del  plano. 
 \end{definition}
   
   \textcolor{red}{DIBUJO}
   
  \textcolor{red}{Agregar forma vectorial del plano}

  
  

\begin{definition} Sean  $(x_0,y_0) \in \mathbb{R}^2$ y $r \in \mathbb{R}_{ > 0}. $

  El conjunto $C \subset \mathbb{R}^2$ que cumple
    \begin{equation*}\label{circ}
    C = \{  (x,y)  \in \mathbb{R}^2 \:|\:   (x-x_0)^2+(y-y_0)^2=r^2 \}
    \end{equation*}  es una circunferencia en el plano. Adem\'as, al vector $(x_0,y_0)$  se lo llama centro y al n\'umero $r$  se lo llama radio de la misma.
   
  El conjunto $D \subset \mathbb{R}^2$ que cumple
    \begin{equation*}
    D = \{  (x,y)  \in \mathbb{R}^2 \:|\:   (x-x_0)^2+(y-y_0)^2 \leq r^2 \}
    \end{equation*}  es un c\'irculo en el plano. Adem\'as, al vector $(x_0,y_0)$  se lo llama centro y al n\'umero $r$  se lo llama radio del mismo.  
   \end{definition}
   \textcolor{red}{DIBUJO}
   
   \textcolor{red}{Copiar el m ismo  formato para la cascara de la esfera y la esfera }
   
   
\begin{definition}\textbf{Ecuación de la esfera.}
    La cáscara de una esfera (es decir, sólo su superficie exterior, hueca) 
    en $\mathbb{R}^3$ se expresa con la ecuación
    \begin{equation*}
        (x-x_0)^2+(y-y_0)^2+(z-z_0)^2=r^2
    \end{equation*}
    Se tiene $(x_0,y_0,z_0)$ como el centro de la cáscara y $r$ como el radio de la misma.
    Para obtener la esfera completa, es decir, cáscara y relleno interior, se usa la inecuación:
    \begin{equation*}
        (x-x_0)^2+(y-y_0)^2+(z-z_0)^2\leq r^2
    \end{equation*}
    Esto es análogo al caso de la circunferencia y el círculo.
\end{definition}

\textcolor{red}{DIBUJO}
   