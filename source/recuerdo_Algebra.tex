\begin{definition}
    Un espacio $\mathbb{R}^n$ es un espacio vectorial real al que pertenecen
    los vectores de $n$ componentes. Es decir:
    \begin{equation*}
        \mathbb{R}^n = \left\{ \boldsymbol{v} / \boldsymbol{v}=(x_1,x_2,x_3,...,x_n), \; x_i \in \mathbb{R} \;\forall i \in \mathbb{N}_{\leq n} \right\}
    \end{equation*}
\end{definition}
\begin{definition}
    Una \textbf{Transformación Lineal} es una función $\boldsymbol{T}:A\subseteq \mathbb{V} \rightarrow A\subseteq \mathbb{W}$, con 
    $\mathbb{V}$ y $\mathbb{W}$ espacios vectoriales, que cumple con la condición de linealidad:
    \begin{equation*}
        \boldsymbol{T}(\alpha \cdot \boldsymbol{v_1} + \beta \cdot \boldsymbol{v_2})=\alpha \cdot \boldsymbol{T}(\boldsymbol{v_2}) + \beta \cdot \boldsymbol{T}(\boldsymbol{v_2})
    \end{equation*}
    Donde $\boldsymbol{v_1},\boldsymbol{v_2} \in \mathbb{V}$ y $\alpha,\beta \in \mathbb{K}$. En el caso de espacios vectoriales reales,
    $\mathbb{K} = \mathbb{R}$.
\end{definition}
A continuación, se detallan las ecuaciones de curvas y superficies comunes.
\begin{definition}
    (Ecuación de la recta en el espacio) La ecuación vectorial de una recta en $\mathbb{R}^3$ es:
    \begin{equation*}
        (x,y,z) = \lambda \cdot \boldsymbol{v} + \boldsymbol{x_0}
    \end{equation*}
    con $\lambda\in\mathbb{R}$ un parámetro, $\boldsymbol{v}\in\mathbb{R}^3$ el vector director de la recta,
    y $\boldsymbol{x_0}\in\mathbb{R}^3$ un punto de paso. El punto de paso es un punto cual se quiera
    sobre la recta y el vector director se puede expresar como $\boldsymbol{v}=\boldsymbol{x_1}-\boldsymbol{x_2}$, 
    con $\boldsymbol{x_1}$ y $\boldsymbol{x_2}$ dos puntos distintos cualquiera sobre la recta.
\end{definition}
\begin{definition}
    (Ecuación del plano) La ecuación canónica del plano (en $\mathbb{R}^3$) es $ax+by+cz=d$, que también puede expresarse,
    vectorialmente, como $(a,b,c)\cdot(x-x_0,y-y_0,z-z_0)=0$. Esto último equivale a:
    \begin{equation*}
        \boldsymbol{n}\cdot(\boldsymbol{x} - \boldsymbol{x_0})=0
    \end{equation*}
    con $\boldsymbol{n}\in\mathbb{R}^3$ un vector normal al plano, y $\boldsymbol{x_0}\in\mathbb{R}^3$
    un punto de perteneciente al plano. El vector normal, además, puede calcularse como el producto vectorial
    de dos vectores pertenecientes al plano.
\end{definition}
\begin{definition}
    (Ecuación de la circunferencia) La ecuación de una circunferencia en el plano ($\mathbb{R}^2$)
    se expresa como 
    \begin{equation*}
        (x-x_0)^2+(y-y_0)^2=r^2
    \end{equation*} 
    En este caso, $(x_0,y_0)$ es el centro
    de la circunferencia, y $r$ el radio de la misma.
\end{definition}
\begin{definition}
    (Ecuación del círculo) Para representar un circulo en el plano ($\mathbb{R}^2$), se lo hace como el
    área dentro de una circunferencia. Por lo tanto, una región circular en el plano se expresa
    como
    \begin{equation*}
        (x-x_0)^2+(y-y_0)^2\leq r^2
    \end{equation*} 
    incluyendo su borde, la circunferencia exterior.
\end{definition}
\begin{definition}
    (Ecuación de la esfera) La cáscara de una esfera (es decir, sólo su superficie exterior, hueca) 
    en $\mathbb{R}^3$ se expresa con la ecuación
    \begin{equation*}
        (x-x_0)^2+(y-y_0)^2+(z-z_0)^2=r^2
    \end{equation*}
    Se tiene $(x_0,y_0,z_0)$ como el centro de la cáscara y $r$ como el radio de la misma.
    Para obtener la esfera completa, es decir, cáscara y relleno interior, se usa la inecuación:
    \begin{equation*}
        (x-x_0)^2+(y-y_0)^2+(z-z_0)^2\leq r^2
    \end{equation*}
    Esto es análogo al caso de la circunferencia y el círculo.
\end{definition}