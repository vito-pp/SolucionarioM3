% segui la definicion del tromba
Dada una funci\'on continua $f:B\to\Re$, donde $B$ es un paralelep\'ipedo rectangular (una caja) en $\Re^3$, se puede definir la integral de $f$ sobre $B$ como un l\'imite de sumas, o sea una integral de Reimann. Partiendo los tres lados de $B$ en $n$ partes iguales y formando la suma
\[
    S_n=\sum_{i=0}^{n-1}\sum_{j=0}^{n-1}\sum_{k=0}^{n-1}f(c_{ijk})\Delta V,
\]  
donde $c_{ijk}\in B_{ijk}$, el $ijk$-\'esimo paralelep\'ipedo rectangular en la partici\'on de $B$, y $\Delta V$ es el volumen de $B_{ijk}$
\textcolor{red}{agregar dibujo caja en r3}

\begin{definition} 
    Sean $B=[a,b]\times[c,d]\times[p,q]$ y $f:B\to\Re$ una funci\'on acotada. Si existe el $\lim_{n\to\infty}S_n$, para cualquier $c_{ijk}$, llamamos \textbf{integral triple} de $f$ sobre $B$ a
    \[
          \iiint_B f\:dV=\iiint_B f(x,y,z)\:dxdydz
    \]
\end{definition}

Para tener la extender esta noci\'on de integral a un conjunto acotado m\'as general $W\subset\Re^3$, esto es conjuntos que puedan ser encerrados por una caja. Dada $f:U\to\Re$, extender $f$ a una funci\'on $f^*$ tal que
\[
    f^*(x,y,z)=
    \begin{dcases*}
        f(x,y,z) & si $(x,y,z)\in U$ \\[.2cm]
        0        & si $(x,y,z)\notin U$
    \end{dcases*}.
\]
Entonces si $B$ es una caja que contiene a $U$ y $\partial U$ est\'a formada por las gr\'aficas de un n\'umero finito de funciones continuas, % no se que significa esto ultimo
$f^*$ ser\'a integrable y definimos
\[
    \iiint_U f\:dV=\iiint_B f^*\:dV.  
\]
Esta integral es independiente de la selecci\'on de $B$.

\begin{propertie}
    Sean $f,\;g$ dos funciones acotadas e integrables en una regi\'on $U\subset\Re^3$, entonces:
    \begin{enumerate}
        \item[i.] $\alpha f+\beta g$ es integrable en $U$, $\forall\;\alpha,\;\beta\in\Re$ y adem\'as
        \[
            \iiint_U \alpha f+\beta g\:dV=\alpha\iiint_U f\:dV+\beta\iiint_U g\:dV.
        \]
        \item[ii.] El producto $fg$ es integrable en $U$.
        \item[iii.] Si $|g(x,y,z)|\geq k>0\;\forall(x,y,z)\in U$, el conciente $f/g$ es integrable en $U$.
        \item[iv.] Si $f\geq$ en $U$, $\iiint_U f\:dV\geq0$.
        \item[v.]Si $f\leq g$ en $U$, $\iiint_U f\:dV\leq\iiint_U g\:dV.$
        \item[vi.]Si $|f|$ es integrable en $U$, entonces 
        \[
            \left|\iiint_U f\:dv\right|\leq\iiint_U|f|\:dV.  
        \]    
        \item[vii.] Si $U=D_1\cup D_2$ es una partici\'on de $U$, $f$ es integrable en $U$ $\iff$ $f$ es integrable en $D_1$ y $D_2$. En este caso 
        \[
            \iiint_U f\:dV=\iiint_{D_1} f\:dV+\iiint_{d_2}f\:dV.    
        \]
    \end{enumerate}
\end{propertie}

\begin{theorem}
    \textbf{Teorema del valor medio}. Si $f$ es continua en $U$, $\exists(x_0,y_0,z_0)\in U$ tal que 
    \[
        \iiint_U f\:dV=f(x_0,y_0,z_0)\text{Vol}\:(U),
    \]
    donde Vol$\:(U)$ es el volumen de $U$. 
    % no esta definido el volumen de un conjunto (\iiint_U 1 = vol(U)) 
\end{theorem}

\begin{definition}
    \textbf{Regi\'on elemental en }$\Re^3$. Sea $U\subseteq\Re^3$. Se llama 
\end{definition}
