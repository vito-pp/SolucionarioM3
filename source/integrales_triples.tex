\begin{definition}
Dada una funci\'on continua $f:U\to\Re$, donde $U$ es un paralelep\'ipedo rectangular (una caja) en $\Re^3$, se define la suma de Reimann de $f$ sobre $U$, partiendo los tres lados de $U$ en $n$ partes iguales, a
\[
    S_n=\sum_{i=0}^{n-1}\sum_{j=0}^{n-1}\sum_{k=0}^{n-1}f(c_{ijk})\Delta V,
\]  
donde $c_{ijk}\in U_{ijk}$, el $ijk$-\'esimo paralelep\'ipedo rectangular en la partici\'on de $U$, y $\Delta V$ es el volumen de $U_{ijk}$.\final
\end{definition}

\begin{definition} 
    Sean $U=[a,b]\times[c,d]\times[p,q]$ y $f:U\to\Re$ una funci\'on acotada.El l\'imite $\lim_{n\to\infty}S_n$ se llama \textbf{integral triple} de $f$ sobre $U$, y se nota $\iiint_U f\:dV$ a
    \[
          \iiint_U f\:dV=\iiint_U f(x,y,z)\:dxdydz.\finalmath
    \]
\end{definition}

\textcolor{red}{\textit{Aclaramos que el} $\lim_{n\to\infty}B_{ijk}=dV$\textit{? Y que $dV=dxdydz$?}}

Para extender esta noci\'on de integral a un conjunto acotado m\'as general, esto es, conjuntos que puedan ser encerrados por una caja, definimos lo siguiente. 

\begin{definition}
Dada $f:U\to\Re$, se define la funci\'on $f^*$ tal que
\[
    f^*(x,y,z)=
    \begin{dcases*}
        f(x,y,z) & si $(x,y,z)\in W$ \\[.2cm]
        0        & si $(x,y,z)\notin W.$
    \end{dcases*}
\]
Entonces si $U$ es una caja que contiene a $W$ y $\partial W$ est\'a formada por las gr\'aficas de un n\'umero finito de funciones continuas, $f^*$ ser\'a integrable. Luego, se define
\[
    \iiint_W f\:dV=\iiint_U f^*\:dV.  \finalmath
\]
\end{definition}

\textbf{Observaci\'on.} Notar que esta integral es independiente de la selecci\'on de $U$.

\begin{propertie}
    Sean $f,\;g$ dos funciones integrables en una regi\'on $W\subset\Re^3$, entonces:
    \begin{enumerate}
        \item[i.] $\alpha f+\beta g$ es integrable en $W$, $\forall\;\alpha,\;\beta\in\Re$ y adem\'as
        \[
            \iiint_W \left(\alpha f+\beta g\right)dV=\alpha\iiint_W f\:dV+\beta\iiint_W g\:dV.
        \]
        \item[ii.] El producto $fg$ es integrable en $W$.
        \item[iii.] Si $|g(x,y,z)|\geq k>0\;\forall(x,y,z)\in W$, el cociente $f/g$ es integrable en $W$.
        \item[iv.] Si $f\geq$ en $W$, $\iiint_W f\:dV\geq0$.
        \item[v.]Si $f\leq g$ en $W$, $\iiint_W f\:dV\leq\iiint_W g\:dV.$
        \item[vi.]Si $|f|$ es integrable en $W$, entonces 
        \[
            \left|\iiint_W f\:dV\right|\leq\iiint_W|f|\:dV.  
        \]    
        \item[vii.] Si $W=W_1\cup W_2,\;W_1\cap W_2\neq\varnothing,$ es una partici\'on de $W$, $f$ es integrable en $W$ $\iff$ $f$ es integrable en $W_1$ y $W_2$. En este caso 
        \[
            \iiint_W f\:dV=\iiint_{W_1} f\:dV+\iiint_{W_2}f\:dV. \finalmath   
        \]
    \end{enumerate}
\end{propertie}

\begin{definition}
    Se define el volumen de un conjunto $W\subset\Re^n$, y se nota $\text{Vol}(W)$, como la integral, si existe, de la funci\'on 1. Es decir,
    \[
        \text{Vol}(W)=\iiint_W1\:dV.\finalmath
    \]
\end{definition}

\begin{theorem}
    \textbf{Teorema del valor medio}. Sea $f:W\subseteq\Re^3\to\Re^3$ continua y $W$ es un conjunto elemental. Entonces existe $(x_0,y_0,z_0)$ en $W$ tal que 
    \[
        \iiint_W f\:dV=f(x_0,y_0,z_0)\text{Vol}\:(W).\finalmath
    \]
\end{theorem}

\begin{definition}
    \textbf{Conjunto elemental en }$\Re^3$.
    Sea $W\subseteq\Re^3$. Sean $\phi_1,\;\phi_2:[a,b]\to\Re$ funciones y sean $\gamma_1,\;\gamma_2:D\to\Re$ funciones continuas. Un conjunto $W\subseteq\Re^3$ se llama elemental, si puede ser escrito como el conjunto de puntos $(x,y,z)$ que satifacen
    \begin{equation*}
        x\in[a,b], \qquad \phi_1(x)\leq y\leq\phi_2(x), \qquad \gamma_1(x,y)\leq z\leq\gamma_2(x,y). 
    \end{equation*}
    O bien, intercambiando el orden de las variables.\final
\end{definition}

\begin{theorem}
    \textbf{Teorema del cambio de variables para integrales triples}. Sea $f:W\subseteq\Re^3\to\Re^3$ un campo escalar integrable. Sea $\mathbf{T}:W^*\subseteq\Re^3\to W\subseteq\Re^3$ continua de clase $C^1$ en $\interior{W^*}$, inyectiva en $\interior{W^*}$ y $\mathbf{T}(W^*)=W$. Entonces
    \[
        \iiint_W f\:dV=\iiint_{W^*}(f\circ\mathbf{T})J_\mathbf{T}\:dV.\finalmath
    \]
\end{theorem}
