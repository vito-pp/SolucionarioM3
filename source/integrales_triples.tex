\textcolor{red}{definir paralelep\'ipedo rectangular B en R3. }
\textcolor{red}{definir particion de un  paralelep\'ipedo rectangular. quien es $\Delta V_{ijk}$  quien es $V_{ijk}$. Explicar la notacion que aparece en la primer definicion }
\textcolor{red}{se podria hacer algun dibujo} 


\begin{definition}
Dada una funci\'on continua $f:U\to\R$, donde $U$ es un paralelep\'ipedo rectangular (una caja) en $\Rn{3}$, se define la suma de Reimann de $f$ sobre $U$, partiendo los tres lados de $U$ en $n$ partes iguales, a
\[
    S_n=\sum_{i=0}^{n-1}\sum_{j=0}^{n-1}\sum_{k=0}^{n-1}f(c_{ijk})\Delta V_{ijk},
\]  
donde $c_{ijk}\in U_{ijk}$, el $ijk$-\'esimo paralelep\'ipedo rectangular en la partici\'on de $U$, y $\Delta V_{ijk}$ es el volumen de $U_{ijk}$.
\end{definition}

\begin{definition} 
    Sean $U=[a,b]\times[c,d]\times[p,q]$ y $f:U\to\R$ una funci\'on.  Diremos que $f$ es integrable sobre $U$ si existe y es finito el  l\'imite $\lim_{n\to\infty}S_n$, en tal caso  el valor de dicho l\'imite  se llama \textbf{integral triple} de $f$ sobre $U$ y se nota 
    \[
          \iiint_U f\:dV=\iiint_U f(x,y,z)\:dxdydz.
    \]
\end{definition}

\begin{obs}
\textcolor{red}{agregar condiciones sobre f para la  existencia del limite}
\end{obs}


Para extender esta noci\'on de integral a un conjunto acotado m\'as general, esto es, conjuntos que puedan ser encerrados por una caja, definimos lo siguiente. 

\begin{definition}
Dada $f:U\to\R$, se define la funci\'on $f^*$ tal que
\[
    f^*(x,y,z)=
    \begin{dcases*}
        f(x,y,z) & si $(x,y,z)\in W$ \\[.2cm]
        0        & si $(x,y,z)\notin W.$
    \end{dcases*}
\]
Entonces si $U$ es una caja que contiene a $W$ y $\partial W$ est\'a formada por las gr\'aficas de un n\'umero finito de funciones continuas, $f^*$ ser\'a integrable.  Se define
\[
    \iiint_W f\:dV=\iiint_U f^*\:dV.  
\]
\end{definition}

\begin{obs} 
   Esta definici\'on es independiente de la selecci\'on de $U$.
\end{obs}

\begin{propertie}
    Sean $f,\;g$ dos funciones integrables en una regi\'on $W\subset\Rn{3}$, entonces:
    \begin{enumerate}
        \item[i.] $\alpha f+\beta g$ es integrable en $W$, $\forall\;\alpha,\;\beta\in\R$ y adem\'as
        \[
            \iiint_W \left(\alpha f+\beta g\right)dV=\alpha\iiint_W f\:dV+\beta\iiint_W g\:dV.
        \]
        \item[ii.] El producto $fg$ es integrable en $W$.
        \item[iii.] Si $|g(x,y,z)|\geq k>0\;\forall(x,y,z)\in W$, el cociente $f/g$ es integrable en $W$.
        \item[iv.] Si $f\geq 0 $ en $W$, $\iiint_W f\:dV\geq0$.
        \item[v.]Si $f\leq g$ en $W$, $\iiint_W f\:dV\leq\iiint_W g\:dV.$
        \item[vi.]Si $|f|$ es integrable en $W$, entonces 
        \[
            \left|\iiint_W f\:dV\right|\leq\iiint_W|f|\:dV.  
        \]    
        \item[vii.] Si $W=W_1\cup W_2, y \;W_1\cap W_2\neq\varnothing,$ es una partici\'on de $W$, $f$ es integrable en $W$ $\iff$ $f$ es integrable en $W_1$ y $W_2$. En este caso 
        \[
            \iiint_W f\:dV=\iiint_{W_1} f\:dV+\iiint_{W_2}f\:dV.    
        \]
    \end{enumerate}
\end{propertie}

\begin{definition}
    Se define el volumen de un conjunto $W\subset\Rn{3}$  y se nota $\text{Vol}(W)$, a
    \[
        \text{Vol}(W)=\iiint_W1\:dV.
    \]
\end{definition}

\begin{theorem}
    \textbf{Teorema del valor medio}. Sea $f:W\subseteq\Rn{3}\to\Rn{3}$ continua y $W$ es un conjunto elemental. Entonces existe $(x_0,y_0,z_0)$ en $W$ tal que 
    \[
        \iiint_W f\:dV=f(x_0,y_0,z_0)\text{Vol}(W).
    \]
\end{theorem}

\begin{definition}  \textcolor{red}{No queda claro quien es el conjunto D.  }
    \textbf{Conjunto elemental en }$\Rn{3}$.
    Sea $W\subseteq\Rn{3}$. Sean $\phi_1,\;\phi_2:[a,b]\to\R$ funciones y sean $\gamma_1,\;\gamma_2:D\to\R$ funciones continuas.  Un conjunto $W\subseteq\Rn{3}$ se llama elemental, si puede ser escrito  \textcolor{red}{de algunas de las siguientes 6 maneras: es largo pero escribirlas todas, ojo con la notacion} como el conjunto de puntos $(x,y,z)$ que satifacen
    \begin{equation*}
        x\in[a,b], \qquad \phi_1(x)\leq y\leq\phi_2(x), \qquad \gamma_1(x,y)\leq z\leq\gamma_2(x,y). 
    \end{equation*}
    O bien, intercambiando el orden de las variables.
\end{definition}


 \textcolor{red}{Ilustracion}


\begin{theorem}
    \textbf{Teorema del cambio de variables para integrales triples}. Sea $f:W\subseteq\Rn{3}\to\R$  un campo escalar integrable. Sea $\mathbf{T}:W^*\subseteq\Rn{3}\to W\subseteq\Rn{3}$ continua, de clase $\mathcal{C}^1$ en $\interior{(W^*)}$,  inyectiva en $\interior{(W^*)}$ y tal que $\mathbf{T}(W^*)=W$. Entonces
    \[
        \iiint_W f\:dV=\iiint_{W^*}(f\circ\mathbf{T})J_\mathbf{T}\:dV.
    \]
\end{theorem}
