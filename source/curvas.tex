\begin{definition}
    Definimos una curva $\Gamma\in\Re^n$ como la imagen de una trayectoria $\boldsymbol{\sigma}:I\subseteq\Re\to\Re^n$ continua y de clase $C^1$ a trozos. Si $I=[a,b]$ a los puntos $\boldsymbol{\sigma}(a)$ y $\boldsymbol{\sigma}(b)$ se los llama extremos de la curva.\final
\end{definition}

\begin{center}
\begin{tikzpicture}

    % Real number line on the left
    \draw[semithick,-Stealth] (-2,0) -- (2,0);
    \node[label=below:$a$] at (-1.25,0) {$[$}; 
    \node[label=below:$b$] at (1.25,0) {$]$}; 
    \node[label=right:$\mathbb{R}$] at (2,0) {};
    
    % Curved arrow in the middle
    \draw[semithick,->] (2,1) to[out=45,in=135,looseness=0.8] (8,1);
    \node[label=below:$\boldsymbol{\sigma}$] at (5,2) {};
    
    % Plane R^2 on the right
    \begin{scope}[xshift=8cm]
        \draw[semithick] (0,0) .. controls (1,2) and (2,-1) .. (3,1);
        \node at (0,0) [circle,fill,inner sep=1pt] {};
        \node at (3,1) [circle,fill,inner sep=1pt] {};
        \node[below left] at (0,0) {$\boldsymbol{\sigma}(a)$};
        \node[above right] at (3,1) {$\boldsymbol{\sigma}(b)$};
    \end{scope}

\end{tikzpicture}
\end{center}

\begin{center}
\begin{tikzpicture}

    % Real number line on the left
    \draw[semithick,-Stealth] (-2,0) -- (2,0);
    \node[label=below:$a$] at (-1.25,0) {$[$}; 
    \node[label=below:$b$] at (1.25,0) {$]$}; 
    \node[label=right:$\mathbb{R}$] at (2,0) {};

    % Curved arrow in the middle
    \draw[semithick,->] (2,1) to[out=45,in=135,looseness=0.8] (8,1);
    \node[label=below:$\boldsymbol{\sigma}$] at (5,2) {};

    % Plane R^2 on the right
    \begin{scope}[xshift=8cm, yshift=0.4cm]
        \draw[semithick] 
            (0,0) .. controls (4.5,-1) and (-1.5,-2.5) .. (3,1);
        \node at (0,0) [circle,fill,inner sep=1pt] {};
    \node at (3,1) [circle,fill,inner sep=1pt] {};
    \node[below left] at (0,0) {$\boldsymbol{\sigma}(a)$};
    \node[above right] at (3,1) {$\boldsymbol{\sigma}(b)$};
    \end{scope}

\end{tikzpicture}
\end{center}

\textcolor{red}{desp agrego las leyendas}

Si $\boldsymbol{\sigma}(a)=\boldsymbol{\sigma}(b)$ entonces $\Gamma$ se llama curva cerrada.

\begin{center}
\begin{tikzpicture}

    % Real number line on the left
    \draw[semithick,-Stealth] (-2,0) -- (2,0);
    \node[label=below:$a$] at (-1.25,0) {$[$}; 
    \node[label=below:$b$] at (1.25,0) {$]$}; 
    \node[label=right:$\mathbb{R}$] at (2,0) {};
    
    % Curved arrow in the middle
    \draw[semithick,->] (2,1) to[out=45,in=135,looseness=0.8] (8,1);
    \node[label=below:$\boldsymbol{\sigma}$] at (5,2) {};
    
    % Plane R^2 on the right
    \begin{scope}[xshift=8.2cm, yshift=0.5cm]
        \draw[semithick]
            plot[smooth,tension=1.2] coordinates {(1,0.6) (3,0) (2.5,-1.5) (1,-1) (0,0) (1,0.6)};
    
            % \foreach \point in {(1,0.6), (3,0), (2.5,-1.5), (1,-1), (0.2,0.1), (1,0.6)} {
            % \node at \point [circle,fill,inner sep=1pt] {};
            % }
        \node at (0,0) [circle,fill,inner sep=1pt] {};
        \node[below left] at (0,0) {$\boldsymbol{\sigma}(a)=\boldsymbol{\sigma}(b)$};
    \end{scope}

\end{tikzpicture}
\end{center}

Adem\'as, si $\boldsymbol{\sigma}$ es inyectiva en $I$, salvo tal vez en sus bordes, entonces se llama a $\Gamma$ curva simple.

\begin{example}
    Sea la trayectoria $\boldsymbol{\sigma}:[0,2\pi]\to\Gamma$, tal que $\boldsymbol{\sigma}(t)=(\cos t, \sen t)$.

    \begin{center}
    \begin{tikzpicture}

        % Real number line on the left
        \draw[semithick,-Stealth] (-2,0) -- (2,0);
        \node[label=below:$0$] at (-1.25,0) {$[$}; 
        \node[label=below:$2\pi$] at (1.25,0) {$]$}; 
        \node[label=right:$\mathbb{R}$] at (2,0) {};
    
        % Curved arrow in the middle
        \draw[->, semithick] (2,1) to[out=45,in=135,looseness=0.8] (6,1);
        \node[label=below:$\boldsymbol{\sigma}$] at (4,1.7) {};
    
        % Plane R^2 on the right
        \begin{scope}[xshift=8.5cm]
            % Circle
            \draw[semithick] (0,0) circle [radius=2];
            % Axes
            \draw[-Stealth, thin] (-2.5,0) -- (2.5,0) node[right] {$x$};
            \draw[-Stealth, thin] (0,-2.5) -- (0,2.5) node[above] {$y$};
            \node at (-2,-2) [align=center, below] {
                $\Gamma=\text{Im}(\boldsymbol{\sigma})$
            };
            \node at (2,0) [circle,fill,inner sep=1pt] {};
            \node[below left] at (2,0) {$\boldsymbol{\sigma}(0)=\boldsymbol{\sigma}(2\pi)$};
        \end{scope}
    
    \end{tikzpicture}
    \end{center}
    
\end{example}

\begin{definition}
    Continuidad a trozos. [...] \textcolor{red}{No est\'a en el Tromba y no s\'e como definirlo formalmente}.\final
\end{definition}

Cada curva simple $\Gamma$ tiene asociadas dos orientaciones o sentidos posibles. Si los puntos $P$ y $Q$ son los extremos de la curva entonces podemos considerar a $\Gamma$ con orientaci\'on desde $P$ hacia $Q$ o bien desde $Q$ hacia $P$.

\begin{definition}
    Parametrizaci\'on de una curva. Una parametrizaci\'on de una curva simple $\Gamma\in\Re^n$ es una trayectoria $\boldsymbol{\sigma}:I\subseteq\Re\to\Re^n$ continua, de clase $C^1$ a trozos, inyectiva en $\text{int}\:(I)$ y $\text{Im}\:(\boldsymbol{\sigma})=\Gamma$.\final
\end{definition}

\begin{definition}
    Una paremetrizaci\'on $\boldsymbol{\sigma}:I\subset\Re\to\Re^n$ se llama regular si $\boldsymbol{\sigma}'(t)\neq0$.\final
\end{definition}