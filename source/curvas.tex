\begin{definition}
    Definimos una curva $\Gamma\in\Re^n$ como la imagen de una trayectoria $\boldsymbol{\sigma}:I\subseteq\Re\to\Re^n$ continua y de clase $C^1$ a trozos. Si $I=[a,b] a los puntos \boldsymbol{\sigma}(a)$ y $\boldsymbol{\sigma}(b)$ se los llama extremos de la curva.

    Si $\boldsymbol{\sigma}(a)=\boldsymbol{\sigma}(b)$ entonces $\Gamma$ se llama curva cerrada.

    Adem\'as, si $\boldsymbol{\sigma}$ es inyectiva en $I$ salvo tal vez en sus bordes entonces se llama a $\Gamma$ curva simple.
\end{definition}

\textcolor{red}{Ejemplos?}

\begin{definition}
    Continuidad a trozos. [...]
\end{definition}

Cada curva simple $\Gamma$ tiene asociadas dos orientaciones o sentidos posibles. Si los puntos $P$ y $Q$ son los extremos de la curva entonces podemos considerar a $\Gamma$ con orientaci\'on desde $P$ hacia $Q$ o bien desde $Q$ hacia $P$.

\begin{definition}
    Parametrizaci\'on de una curva. Una parametrizaci\'on de una curva simple $\Gamma\in\Re^n$ es una trayectoria $\boldsymbol{\sigma}:I\subseteq\Re\to\Re^n$ continua, de clase $C^1$ a trozos, inyectiva en $\text{int}\:(I)$ y $\text{Im}\:(\boldsymbol{\sigma})=\Gamma$. 
\end{definition}

\begin{definition}
    Una paremetrizaci\'on $\boldsymbol{\sigma}:I\subset\Re\to\Re^n$ se llama regular si $\boldsymbol{\sigma}'(t)\neq0$.
\end{definition}