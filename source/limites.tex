\label{sec:limites}
\begin{definition} [L\'imite] \label{def:limite}
Sean $f:A \subset \Re^n \to \Rn{m}$ una funci\'on, $L \in \Rn{m}$ y $x_0$ un punto de acumulaci\'on de $A$. Decimos que \emph{el l\'imite de $f$ para $x$ tendiendo a $x_o$ es $L$} o que \emph{$f$ tiende a $L$ cuando $x$ tiende a $x_o$}
si
\[
 \forall \varepsilon > 0 \, \exists \delta > 0 / f(x) \in B_{\varepsilon}(L) \, \forall 
 x \in B_{\delta}^*(x_o) \cap A.
\]

En este caso, utilizamos la notaci\'on
\[
 f(x) \xrightarrow[x \to x_o]{} L \quad \text{o} \quad \lim_{x \to x_o} f(x) = L.
\]
\begin{obs} Si $f$ es un \emph{campo escalar} (es decir, si en la definici\'on \eqref{def:limite} es $m = 1$), la condici\'on 
 \[
 \forall \varepsilon > 0 \, \exists \delta > 0 / f(x) \in B_{\varepsilon}(L) \, \forall 
 x \in B_{\delta}^*(x_o) \cap A
\]
es equivalente a 
\[
 \forall \varepsilon > 0 \, \exists \delta > 0 / \abs{ f(x) - L } < \varepsilon \, \forall 
 x \in B_{\delta}^*(x_o) \cap A.
\]
\end{obs}

\begin{obs} La definici\'on de l\'imite puede expresarse equivalentemente en t\'erminos de distancias o de normas:
\begin{itemize} 
 \item Sean $f:A \subset \Re^n \to \Rn{m}$ una funci\'on, $L \in \Rn{m}$ y $x_0$ un punto de acumulaci\'on de $A$. Decimos que \emph{el l\'imite de $f$ para $x$ tendiendo a $x_o$ es $L$} o que \emph{$f$ tiende a $L$ cuando $x$ tiende a $x_o$} si
\[
 \forall \varepsilon > 0 \, \exists \delta > 0 / \dis_m {(f(x),L)} < \varepsilon \, \forall 
 x\in A : 0 < \dis_n {(x,x_o)} < \delta,
\] 
donde $\dis_n$ y $\dis_m$ son las distancias en $\Re^n$ y $\Rn{m}$, respectivamente.

 \item Sean $f:A \subset \Re^n \to \Rn{m}$ una funci\'on, $L \in \Rn{m}$ y $x_0$ un punto de acumulaci\'on de $A$. Decimos que \emph{el l\'imite de $f$ para $x$ tendiendo a $x_o$ es $L$} o que \emph{$f$ tiende a $L$ cuando $x$ tiende a $x_o$} si
\[
 \forall \varepsilon > 0 \, \exists \delta > 0 / \norm{f(x) - L}_m < \varepsilon \, \forall 
 x\in A : 0 < \norm{x - x_o}_n < \delta,
\] 
donde $\norm{\cdot}_n$ y $\norm{\cdot}_m$ son las normas en $\Re^n$ y $\Rn{m}$, respectivamente.
\end{itemize}
\end{obs}

\end{definition}

\begin{theorem}[Unicidad del l\'imite] \label{teo:unicidad_limite}
\mbox{}

Sean $f:A \subset \Re^n \to \Rn{m}$ una funci\'on, $x_0$ un punto de acumulaci\'on de $A$ y $L_1, L_2 \in \Rn{m}$ tales que 
\[
 f(x) \xrightarrow[x \to x_o]{} L_1 \quad \wedge \quad f(x) \xrightarrow[x \to x_o]{} L_2.
\]
Entonces $L_1 = L_2$.
\begin{proof}
\mbox{}

Las ideas utilizadas en la demostraci\'on de este teorema son las que siguen.\\
Como 
\[
 f(x) \xrightarrow[x \to x_o]{} L_1,
\]
dado $\varepsilon > 0$ existe alg\'on $\delta_1 > 0$ tal que cada punto $x \in B_{\delta_1}^*(x_o) \cap A$ se aplica a trav\'es de $f$ en alg\'on punto de la bola de centro $L_1$ y radio $\varepsilon$.

\def\Ro{2.0}
\def\Rb{1.0}
\def\Ra{3.0}
\def\sep{1.0*\Ro}

\input{../figs/fig1}


De la misma manera, como 
\[
 f(x) \xrightarrow[x \to x_o]{} L_2,
\]
dado $\varepsilon > 0$ existe alg\'on $\delta_2 > 0$ tal que cada punto $x \in B_{\delta_2}^*(x_o) \cap A$ se aplica a trav\'es de $f$ en alg\'on punto de la bola de centro $L_2$ y radio $\varepsilon$.

\input{../figs/fig2}

Si suponemos que $L_1 \ne L_2$, por propiedad de la distancia es $\dis(L_1,L_2) > 0$ y podemos elegir 
\[
 \varepsilon = \frac{\dis{(L_1,L_2)}}{2}.
\]
Para este valor de $\varepsilon$ podemos encontrar un punto $x$ que se aplica por $f$ en alg\'on punto de la bola de centro $L_1$ y radio $\varepsilon$ \textbf{y tambi\'on} se aplica por $f$ en alg\'on punto de la bola de centro $L_2$ y radio $\varepsilon$, lo cual es absurdo, pues estas dos bolas son disjuntas.

\input{../figs/fig3}

Ahora s\'i, veamos la demostraci\'on formal del teorema.\\
 Supongamos, por el absurdo, que $L_1 \ne L_2$, y sea $\varepsilon = \dis(L_1,L_2)/2 > 0.$ \\ 
 Como 
 \[
  f(x) \xrightarrow[x \to x_o]{} L_1,
 \]
podemos elegir $\delta_1 > 0$ tal que $x \in B_{\delta_1}^*(x_o) \cap A \then f(x) \in B_{\varepsilon}(L_1)$.\\
Como 
\[
 f(x) \xrightarrow[x \to x_o]{} L_2,
\]
 podemos elegir $\delta_2 > 0$ tal que $x \in B_{\delta_2}^*(x_o) \cap A \then f(x) \in B_{\varepsilon}(L_2)$.\\
 Notemos que, como sugiere la figura, 
 \[
  B_{\varepsilon}(L_1) \cap B_{\varepsilon}(L_2) = \emptyset.
 \]
 En efecto, de no ser as\'i, sea $z \in B_{\varepsilon}(L_1) \cap B_{\varepsilon}(L_2)$. Tenemos que:
 \[
  2 \varepsilon = \dis(L_1,L_2) \le \dis(L_1,z) + \dis(z,L_2)
  < \varepsilon + \varepsilon = 2 \varepsilon \then \varepsilon < \varepsilon. \text{ Absurdo.}
 \]

 Sea $\delta = \min\{ \delta_1, \delta_2 \}$. \\
 Como $B_{\delta}^*(x_o) \subset B_{\delta_1}^*(x_o)$ y $B_{\delta}^*(x_o) \subset B_{\delta_2}^*(x_o)$, si tomamos alg\'on $x \in B_{\delta}^*(x_o) \cap A$\footnote{Notemos que $B_{\delta}^*(x_o) \cap A \ne \emptyset$ porque $x_o$ es un punto de acumulaci\'on de $A$.}, vale que $f(x) \in B_{\varepsilon}(L_1)$ y tambi\'on $f(x) \in B_{\varepsilon}(L_2)$, por lo que $f(x) \in B_{\varepsilon}(L_1) \cap B_{\varepsilon}(L_2)$. Absurdo, pues $B_{\varepsilon}(L_1) \cap B_{\varepsilon}(L_2) = \emptyset$. El absurdo provino de suponer que $L_1 \ne L_2$, por lo cual debe ser $L_1 = L_2$.

\end{proof}

\end{theorem}
\begin{propertie}[\'algebra de l\'imites] \label{prop:alg_lim} Sean $f,g:A \subset \Re^n \to \Rn{m}$ funciones y $x_0$ un punto de acumulaci\'on de $A$. Supongamos que 
  \begin{align*}
   &\lim_{x \to x_o} f(x) = L_1 \in \Rn{m} \\
   &\lim_{x \to x_o} g(x) = L_2 \in \Rn{m}.
  \end{align*}
  Entonces:
  \begin{enumerate} %[i.]
   \item Linealidad
      \begin{enumerate} %[(a)]
       \item $\lim_{x \to x_o} (f + g)(x) = L_1 + L_2$
       \item $\lim_{x \to x_o} (\alpha f)(x) = \alpha L_1 \, \forall \alpha \in \R$
      \end{enumerate}
  \item $\lim_{x \to x_o} (f \cdot g) (x) = L_1 \cdot L_2$\footnote{Si $m \ge 2$, ``$\cdot$'' representa el producto escalar en $\Rn{m}$.}
  \item Si $m = 1$ (i.e., si $f \text{ y } g$ son campos escalares), $g$ es no nula en alg\'on entorno de $x_o$ y $L_2 \ne 0$, 
  \[
   \lim_{x \to x_o} \frac{f(x)}{g(x)} = \frac{L_1}{L_2}.
  \]
  \end{enumerate}
\end{propertie}

\begin{propertie} \label{prop:lim_x_comp}
 Sea $F: A \subset \Re^n \to \Rn{m}$ un campo vectorial y $x_0$ un punto de acumulaci\'on de $A$. Sean $f_j: A \subset \Re^n \to \R \, (j = 1,2, \dots, m)$ campos escalares tales que $F(x) = \left( f_1(x), f_2(x), \dots, f_m(x) \right) \, \forall x \in A$\footnote{Los campos escalares $f_j$ quedan determinados un\'ivocamente por $F$ de la siguiente manera: 
 \begin{align*}
     f_j &: A \subset \Re^n \to \R \\
     &f_j(x) = F(x) \cdot e_j,
 \end{align*} donde $e_j$ es el $j$-\'esimo vector can\'onico de $\Rn{m}$.}.\\
 Sea $L = \left( L_1, L_2, \dots, L_m \right) \in \Rn{m}$, $L_j \in \R \, \forall j = 1,2, \dots, m$. Entonces:
 \[
  \lim_{x \to x_o} F(x) = L \iff \lim_{x \to x_o} f_j(x) = L_j \, \forall j = 1,2, \dots, m.
 \]

\end{propertie}


  \begin{propertie} \label{prop:lim_1}
    Sean $f:A \subset \Re^n \to \R$ una funci\'on y $x_0$ un punto de acumulaci\'on de $A$. Las siguientes proposiciones son equivalentes:
 \begin{enumerate} %[i.]
  \item $\lim_{x \to x_o} f(x) = 0$.
  \item $\lim_{x \to x_o} | f(x) | = 0$.
 \end{enumerate}
 \begin{proof}
 \mbox{}
 
 \begin{enumerate} %[(a)]
  \item Veamos que (I) implica (II). Como 
  \[
   \lim_{x \to x_o} f(x) = 0,
  \]
  por definici\'on de l\'imite, dado $\varepsilon > 0$ podemos tomar $\delta > 0$ tal que 
  \[
   \abs{f(x) - 0} < \varepsilon \, \forall x \in B^*_{\delta}(x_o) \cap A.
  \]
  Para este valor de $\delta$ se cumple que
  \[
   \Big| \abs{f(x)} - 0 \Big| = \Big| \abs{f(x)} \Big| = \abs{f(x)} = \abs{f(x) - 0} < \varepsilon \, 
   \forall x \in B^*_{\delta}(x_o) \cap A,
  \]
  de modo que, por definici\'on, 
  \[
   \lim_{x \to x_o} \abs{ f(x) } = 0.
  \]
  \item Veamos que (II) implica (I). Como
  \[
   \lim_{x \to x_o} \abs{ f(x) } = 0,
  \]
  por definici\'on de l\'imite, dado $\varepsilon > 0$ podemos tomar $\delta > 0$ tal que 
  \[
   \Big| \abs{f(x)} - 0 \Big| < \varepsilon \, \forall x \in B^*_{\delta}(x_o) \cap A.
  \]
  Para este valor de $\delta$ se cumple que
  \[
   \abs{f(x) - 0} = \abs{f(x)} = \Big| \abs{f(x)} \Big| = \Big| \abs{f(x)} - 0 \Big| < \varepsilon \, 
   \forall x \in B^*_{\delta}(x_o) \cap A,
  \]
  de modo que, por definici\'on, 
  \[
   \lim_{x \to x_o} f(x) = 0.
  \]
 \end{enumerate}

  
 \end{proof}

\end{propertie} 

\begin{propertie} \label{prop:sandwich} %[Principio de intercalaci\'on]
Sean $f, g, h : A \subset \Re^n \to \R$ funciones y $x_0$ un punto de acumulaci\'on de $A$. Supongamos que $\exists \delta_o > 0$ tal que 
\[
 f(x) \le g(x) \le h(x) \, \forall x \in B_{\delta_o}(x_o) \cap A.
\]
Si $\lim_{x \to x_o}f(x) = \lim_{x \to x_o}h(x) = L$, entonces
\[
 \lim_{x \to x_o}g(x) = L.
\]
 
\end{propertie}

\begin{propertie} \label{prop:cero_x_acotada}
 Sean $f, g : A \subset \Re^n \to \R$ funciones y $x_0$ un punto de acumulaci\'on de $A$. Si $\exists \delta_o > 0$ tal que $f$ es acotada en $B_{\delta_o}(x_o) \cap A$ (i.e.: $\exists M > 0 \text{ tal que } \abs{f(x)} \le M \forall x \in  B_{\delta_o}(x_o) \cap A$) y $\lim_{x \to x_o}g(x) = 0$, entonces
\[
 \lim_{x \to x_o}(f \cdot g)_{(x)} = 0.
\]
\begin{proof}
\mbox{}

Supongamos que, para cierto $\delta_o > 0$, $f$ es acotada en $B_{\delta_o}(x_o) \cap A$, y sea $M > 0 \text{ tal que } \abs{f(x)} \le M \quad \forall x \in  B_{\delta_o}(x_o) \cap A$.
Como 
\[
\lim_{x \to x_o}g(x) = 0, 
\]
dado $\varepsilon > 0 $, por definici\'on de l\'imite, podemos elegir $\delta_1 > 0$ tal que 
\[
 \abs{g(x)} < \frac{\varepsilon}{M} \quad \forall x \in  B^*_{\delta_1}(x_o) \cap A.                                                                                               
\]
Sea $\delta = \min{ \left\{ \delta_o, \delta_1 \right\} }$. \\
Se cumple que
\[
 \abs{ (f \cdot g)_{(x)} - 0 } = \abs{ f(x) \cdot g(x) } = \abs{f(x)}\abs{g(x)} \le 
 M \abs{g(x)} < M \frac{\varepsilon}{M} = \varepsilon \quad \forall x \in  B^*_{\delta}(x_o) \cap A,
\]
Por lo tanto, por definici\'on de l\'imite,
\[
 \lim_{x \to x_o}(f \cdot g)_{(x)} = 0.
\]

\end{proof}
\end{propertie}


\begin{propertie} [L\'imite de la composici\'on] \label{prop:lim_comp}
\mbox{}

Sean $f, g$ funciones 
\begin{align*}
 g &:A \subset \Re^n \to B \subset \Rn{m} \\
 f &:B \subset \Rn{m} \to \Rn{p},
\end{align*}
y puntos $a \in A', b \in B'$ tales que 
\[
 \lim_{x \to a} g(x) = b \quad \wedge \quad \lim_{u \to b} f(u) = L.
\]
Entonces
\[
 \lim_{x \to a} \left( f \circ g \right)_{(x)} = L.
\]
\end{propertie}
\begin{example}
 Veamos que 
 \[
  \lim_{(x,y) \to (0,0)} \frac{\sin(x^2 + y^2)}{x^2 + y^2} = 1.
 \]
 En efecto, sabemos que 
  \[
    \lim_{u \to 0} \frac{\sin(u)}{u} = 1,
  \]
  y que
  \[
    \lim_{(x,y) \to (0,0)} x^2 + y^2 = 0.
  \]  
  Por lo tanto, si definimos 
  \begin{align*}
   f&: \R - \{0\} \to \R \\
    &f(u) = \frac{\sin(u)}{u} \\
   g&: \Rn{2} - \{(0,0)\} \to \R \\
    &g(x,y) = x^2 + y^2,
  \end{align*}
  por la propiedad \eqref{prop:lim_comp} es
  \[
   \lim_{(x,y) \to (0,0)} \left( f \circ g \right)_{(x,y)} = 
   \lim_{(x,y) \to (0,0)} \frac{\sin(x^2 + y^2)}{x^2 + y^2} = 1.
  \]
\end{example}

Una consecuencia directa de la propiedad \eqref{prop:lim_comp} es el siguiente corolario: 
\begin{corollary} [L\'imites por curvas] \label{cor:lim_curva}
  Sean $f: A \subset \Rn{2} \to \R, \, (x_o,y_o)$ un punto de acumulaci\'on de $A$ y $L \in \R$ tales que 
  \[
   \lim_{(x,y) \to (x_o,y_o)} f(x,y) = L.
  \]

  Entonces, para cualesquiera funciones continuas $g_1, \, g_2 : I \subset \R \to \R$ tales que, para alg\'on $t_o \in I'$
  \[
    \lim_{t \to t_o} \left( g_1(t),g_2(t) \right) = \left( x_o , y_o \right)
  \]
  y, adem\'as, $\left( g_1(t), g_2(t) \right) \in A \, \forall t \in \left( I - \{t_o\} \right),$ se cumple que
  
  \[
   \lim_{t \to t_o} f\left( g_1(t),g_2(t) \right) = L.
  \]
\end{corollary}
\begin{example}
 \begin{align*}
   f: &\left\lbrace (x,y) \in \Rn{2} : x \ne y \right\rbrace \to \R \\
   &f(x,y) = \frac{x^2}{x - y}.
  \end{align*} 
  Veamos que $f$ no tiene l\'imite para $(x,y) \to (0,0)$.
  \begin{enumerate} [(a)]
   \item  Sean $g_1(t) = t, \, g_2(t) = t + t^2$.
  \[
   f(g_1(t),g_2(t)) = f(t,t + t^2) = \frac{t^2}{-t^2} \to -1 \text{ cuando } t \to 0,
  \]
  de modo que, \emph{si $f$ tiene l\'imite}, el l\'imite es -1, por \eqref{cor:lim_curva} y la unicidad del l\'imite.
  \item  Sean $g_1(t) = t, \, g_2(t) = t - t^2$.
  \[
   f(g_1(t),g_2(t)) = f(t,t + t^2) = \frac{t^2}{t^2} \to 1 \text{ cuando } t \to 0,
  \]
  de modo que, \emph{si $f$ tiene l\'imite}, el l\'imite es 1, por \eqref{cor:lim_curva} y la unicidad del l\'imite.
  \end{enumerate}
  Por lo tanto, por la unicidad del l\'imite (teorema \eqref{teo:unicidad_limite}), \emph{si $f$ tiene l\'imite $L$}, $L = 1 = -1$, lo que es un absurdo. El absurdo provino de suponer que $f$ tiene l\'imite, por lo cual $f$ no puede tener l\'imite.
\end{example}

\begin{propertie}[L\'imites iterados] \label{prop:lim_ite}
  Sean $f: A \subset \Rn{2} \to \R$, $(x_o,y_o)$ un punto de acumulaci\'on de $A$ y $L \in \R$ tales que
  \[
   \lim_{(x,y) \to (x_o,y_o)} f(x,y) = L.
  \]
  \begin{enumerate} %[I.]
   \item Si para cada $x \ne x_o$ existe el l\'imite
  \[
   \lim_{y \to y_o} f(x,y) = g(x)
  \]
  y adem\'as existe el l\'imite
  \[
   \lim_{x \to x_o} g(x) = \lim_{x \to x_o} \left( \lim_{y \to y_o} f(x,y) \right) = L_1,
  \]  
  entonces $L = L1$.
  \item Si para cada $y \ne y_o$ existe el l\'imite
  \[
   \lim_{x \to x_o} f(x,y) = h(y)
  \]
  y adem\'as existe el l\'imite
  \[
   \lim_{y \to y_o} h(y) = \lim_{y \to y_o} \left( \lim_{x \to x_o} f(x,y) \right) = L_2, 
  \]
  entonces $L = L_2$.
  \end{enumerate}
\end{propertie}
% \emph{Ejemplos:
% \begin{itemize}
%  \item existan los iterados, sean distintos
%  \item existan los iterados, sean iguales, pero no hay l\'imite
%  \item existan los iterados, sean iguales, probar por def que existe el l\'imite
%  \item ejemplo en el que exista el l\'imite doble pero no los iterados
% \end{itemize}
% }

\begin{propertie}[L\'imites por conjuntos] \label{prop:lim_x_conj}
   Sea $f : A \subset \Re^n \to \Rn{m}$ una funci\'on, con $A = A_1 \cup A_2$.\\
   Sea $f_1$ la \emph{restricci\'on de $f$ a $A_1$}:
 \begin{align*}
  f_1 &: A_1 \subset \Re^n \to \Rn{m} \\
  &f_1(x) = f(x).
 \end{align*}
Sea $f_2$ la \emph{restricci\'on de $f$ a $A_2$}:
 \begin{align*}
  f_2 &: A_2 \subset \Re^n \to \Rn{m} \\
  &f_2(x) = f(x).
 \end{align*}
 Sean $x_o \in A_1' \cap A_2'$ y $L \in \Rn{m}$. Entonces
 \[
  \lim_{x \to x_o} f(x) = L \quad \iff 
  \lim_{x \to x_o} f_1(x) = L \wedge \lim_{x \to x_o} f_2(x) = L.
 \]
\end{propertie}
\begin{example}
\mbox{}

 Sea 
  \[
   f(x,y) = 
     \begin{cases}
        \frac{e^x - 1}{x}  & \text{ si } x \ne 0  \\
         1                 & \text{ si } x = 0
     \end{cases}
  \]
 Veamos que $\lim_{(x,y) \to (0,0)} f(x,y) = 1$.\\
 Sean $A_1 = \{ (x,y) \in \Rn{2} : x \ne 0\}, \, A_2 = \{ (x,y) \in \Rn{2} : x = 0\}$ y 
 \begin{align*}
  f_1 &: A_1 \to \R \\
  &f_1(x,y) = \frac{e^x - 1}{x}, \\
  f_2 &: A_2 \to \R \\
  &f_2(x,y) = 1.
  \end{align*}
 Como 
 \[
    \lim_{t \to 0} \frac{e^t - 1}{t} = 1
 \]
 y 
 \[
    \lim_{(x,y) \to (0,0)} x = 0,
 \]
 por la propiedad \eqref{prop:lim_comp} es 
 \[
    \lim_{(x,y) \to (0,0)} f_1(x,y) = 1.
 \]
Como adem\'as $\lim_{(x,y) \to (0,0)} f_2(x,y) = 1$ y $A_1 \cup A_2 = \Rn{2}$, por la propiedad \eqref{prop:lim_x_conj} es
 \[
  \lim_{(x,y) \to (0,0)} f(x,y) = 1.
 \]
\end{example}

\begin{definition}[L\'imite infinito] \label{def:lim_inf}
  Sean $f : A \subset \Re^n \to \R$ una funci\'on y $x_0$ un punto de acumulaci\'on de $A$. 
  \begin{enumerate} %[I.]
   \item Decimos que \emph{el l\'imite de $f$ para $x$ tendiendo a $x_o$ es $+\infty$} o que \emph{$f$ tiende a $+\infty$ cuando $x$ tiende a $x_o$} si
    \[
      \forall M > 0 \, \exists \delta > 0 / f(x) > M \, \forall 
	  x \in B_{\delta}^*(x_o) \cap A.
    \]
    En este caso, utilizamos la notaci\'on
    \[
      f(x) \xrightarrow[x \to x_o]{} +\infty \quad \text{o} \quad \lim_{x \to x_o} f(x) = +\infty.
    \]
   \item Decimos que \emph{el l\'imite de $f$ para $x$ tendiendo a $x_o$ es $-\infty$} o que \emph{$f$ tiende a $-\infty$ cuando $x$ tiende a $x_o$} si
    \[
      \forall M > 0 \, \exists \delta > 0 / f(x) < -M \, \forall 
	  x \in B_{\delta}^*(x_o) \cap A.
    \]
    En este caso, utilizamos la notaci\'on
    \[
      f(x) \xrightarrow[x \to x_o]{} -\infty \quad \text{o} \quad \lim_{x \to x_o} f(x) = -\infty.
    \]
   \item Decimos que \emph{el l\'imite de $f$ para $x$ tendiendo a $x_o$ es $\infty$} o que \emph{$f$ tiende a $\infty$ cuando $x$ tiende a $x_o$} si
    \[
      \forall M > 0 \, \exists \delta > 0 / \abs{f(x)} > M \, \forall 
	  x \in B_{\delta}^*(x_o) \cap A.
    \]
    En este caso, utilizamos la notaci\'on
    \[
      f(x) \xrightarrow[x \to x_o]{} \infty \quad \text{o} \quad \lim_{x \to x_o} f(x) = \infty.
    \]
  \end{enumerate}
\end{definition}

\begin{definition}[L\'imite en el infinito] \label{def:lim_x_inf}
Sean $f : \Re^n \to \R$ una funci\'on y $L \in \R$. Decimos que
%la manera de presentar esta definici\'on no es consistente con las anteriores
\[
 f(x) \xrightarrow[\norm{x} \to +\infty]{} L \quad \text{o} \quad \lim_{\norm{x} \to +\infty} f(x) = L
\]
si 
\[ 
 \forall \varepsilon > 0 \, \exists \delta > 0 / f(x) \in B_{\varepsilon}(L) \, \forall 
 x / \norm{x} > \delta.
\]
\end{definition}


\iffalse
\begin{propertie} \label{prop:lim_2}
 Sean $f:A \subset \Rn{m} \to \Rn{m}$, $g:A \to \R$ y $x_0$ un punto de acumulaci\'on de $A$, tales que 
 \[
  \lim_{x \to x_o} g(x) = L_g \quad \wedge \quad \lim_{x \to x_o} \left( (f + g)_{(x)} \right) = L.
 \]
 Entonces, existe el l\'imite $\lim_{x \to x_o} f(x)$ y, adem\'as, $\lim_{x \to x_o} f(x) = L - L_g$.
\end{propertie}
\fi
