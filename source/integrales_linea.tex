
\textcolor{red}{Falta decir quies es $I_i$}



\begin{definition}
    Sea $\Gamma$ una curva simple en $\Rn{n}$ y sea $\boldsymbol{\sigma}:I\subseteq\R\to\Rn{n}$ una parametrizaci\'on regular de $\Gamma$. Se define la \textbf{longitud de} $\Gamma$, y se nota  Long($\Gamma$), a 
    \[
        \text{Long}(\Gamma)=\sum_{i=1}^{m}\int_{I_i}\|\boldsymbol{\sigma}'(t)\|dt,    
    \]
    donde $\boldsymbol{\sigma}\lvert_{I_i}$ es de clase $\mathcal{C}^1$.
\end{definition}


\textcolor{red}{Ejemplos: puede ser la circunsferencia y un segmento.  Ilustrar}

\begin{definition}
    Sea $\boldsymbol{\sigma}:I=[a,b]\to\Rn{n}$ una trayectoria continua y de clase $\mathcal{C}^1(\interior{I})$ y sea $f$ un campo escalar tal que la funci\'on compuesta $f\circ\boldsymbol{\sigma}(t)$ sea continua en $I$. Se define la \textbf{integral de trayectoria}, o \textbf{integral de} $f$ \textbf{a lo largo de la trayectoria} $\boldsymbol{\sigma}$, y se la nota $\int_{\boldsymbol{\sigma}}f\:ds$, a
    \[
        \int_{\boldsymbol{\sigma}}f\:ds=\int_a^b (f\circ\boldsymbol{\sigma})\|\boldsymbol{\sigma}'(t)\|\:dt.
    \]
\end{definition}

Si $\boldsymbol{\sigma}(t)$ es s\'olo $\mathcal{C}^1$ a trozos o $f\circ\boldsymbol{\sigma}(t)$ es continua a trozos, se define $\int_{\boldsymbol{\sigma}}f\:ds$ sobre una partici\'on de $[a,b]$, donde sobre cada intervalo de la partici\'on $(f\circ\boldsymbol{\sigma})\|\boldsymbol{\sigma}'(t)\|$ sea continua, y sumando las integrales sobre cada intervalo de la partici\'on. 

\begin{definition}
    Sea $\Gamma$ una curva simple en $\Rn{n}$ y sea $\boldsymbol{\sigma}:[a,b] \subseteq\R\to\Rn{n}$ una parametrizaci\'on regular de $\Gamma$.  Sea $f:A\subseteq\Rn{n}\to\R$ continua con $\Gamma \subset A$.  Se define la integral de l\'inea del campo $f$ sobre la curva $\Gamma$, y se nota por $\int_{\Gamma}f\:ds$ a
    \[
        \int_{\Gamma}f\:ds=\sum_{i=1}^{m}\int_{I_i}f\circ\boldsymbol{\sigma}(t)\|\boldsymbol{\sigma}'(t)\|\:dt,  
    \]
    donde $\boldsymbol{\sigma}\lvert_{I_i}$ es de clase $\mathcal{C}^1$.
\end{definition}

\textcolor{red}{Falta decir quies es $I_i$}

\begin{obs} \textcolor{red}{La definicion anterior no depende de la parametrizacion }
\end{obs} 


\textcolor{red}{Ejemplos.}

\begin{obs} 
 Sea $f:A\subseteq\Rn{2} \to\R$   tal que  $f(x,y)>0 \; \forall (x,y) \in A$ y   sea $\Gamma \subset A$ una curva simple en $\Rn{2}$.   La integral de l\'inea tiene de $f$ sobre $\Gamma$  se puede interpretar geometricamente como el ``\'area de la valla de base $\Gamma$ y altura $f$". 
\textcolor{red}{ilustracion de la aplicacion}
\end{obs}

\textcolor{red}{Ejemplo.}

\begin{definition}
 \textcolor{red}{Copiar exactamente la estructura y orden  de las definiciones para campos escalares.}     Integral de un campo vecrorial a lo largo de una curva simple.
    Sea $\mathbf{F}:A\subseteq\Rn{n}\to\Rn{n}$ un campo vectorial continuo sobre una curva simple orientada $\Gamma\subset A$. Sea $\boldsymbol{\sigma}:[a,b]\to A\subseteq\Rn{n}$ una parametrizaci\'on de $\Gamma$ que preserva su orientaci\'on. Se define la \textbf{integral de} $\mathbf{F}$ \textbf{sobre la curva simple} $\boldsymbol{\Gamma}$ como
    \[
        \int_{\Gamma}\mathbf{F}\cdot d\mathbf{s}=\int_{\boldsymbol{\sigma}}\mathbf{F}\cdot d\mathbf{s} 
    \]
\end{definition}

\textcolor{red}{Definicion. poner referencia a la definicion necesaria} Si $\Gamma$  es cerrada,  se suele notar a la integral como
\[
    \oint_{\Gamma}\mathbf{F}\cdot d\mathbf{s},  
\]  y se la llama integral de circulaci\'on de $\mathbf{F}$ sobre $\Gamma$.





\begin{definition}
    Sean $\mathbf{F}:A\subseteq\Rn{n}\to\Rn{n}$ un campo vectorial y $\boldsymbol{\sigma}:[a,b]\to A\subseteq\Rn{n}$ una trayectorial $\mathcal{C}^1$ a trozos tales que $\mathbf{F}\circ\boldsymbol{\sigma}:[a,b]\to\Rn{n}$ es una funci\'on continua. Se define la \textbf{integral de de l\'inea del campo vecorial} $\mathbf{F}$ \textbf{a lo largo de la trayectoria} $\boldsymbol{\sigma}$ como
    \[
        \int_{\Gamma}\mathbf{F}\cdot d\mathbf{s}=\int_a^b\mathbf{F}\circ\boldsymbol{\sigma}(t)\cdot\boldsymbol{\sigma}'(t)\:dt. 
    \]
\end{definition}

Se usa la notaci\'on $\int_{\boldsymbol{\sigma}}\mathbf{F}\cdot d\mathbf{s}=\int_{\boldsymbol{\sigma}}\mathbf{F}\cdot\hat{\mathbf{t}}\:ds$,  donde $\hat{\mathbf{t}}$ es versor tangencial a la curva  $\Gamma = Img(\boldsymbol{\sigma}).$
